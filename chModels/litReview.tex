\section{Background Literature}
\label{chModels.sec.litReview}

%The priciple foundation literature on which this Thesis builds
%includes the following:
%
The foundation of this Thesis is a set of good ideas and facts cherry
picked from the long history of research into modeling rigid-body
dynamics.
%
Those ideas and facts are summarized in this Chapter.
%
Readers requiring more information are referred to the following texts
for the reasons listed.
%
Taylor's {\it Classical Mechanics}\cite{Taylor2005} provides an
excellent introduction to modeling rigid-body motion using
non-inertial reference frames.
%
In \cite{Bullo2004}, Bullo and Lewis provide both a rigorous
development of and a compelling case for utilizing differential
geometry techniques in nonlinear control theory.
%
Chapters 5, 6, and 7 from \cite{Chirikjian2000}, by Chirikjian and
Kyatkin, and \cite{murray&li&sastry}, by Murray, Li and Sastry,
elucidate the representation of rigid-body motion in the groups SO(3)
and SE(3).
%
{\it A Short Introduction to Applications of Quaternions}, by
A. G. Rawlings\cite{RawlingQuaternions}, presents different coordinate
systems used to represent rigid-body rotations.
%
Rawlings's text also presents the intuition behind the individual
coordinate systems and the interconnections between those
representations.
%
{\it Guidance and Control of Ocean Vehicles} by
Fossen\cite{fossen} and {\it Advances in Six-Degree-of-Freedom
  Dynamics and Control of Underwater Vehicles} by
Martin\cite{martin.thesis} present the details of modeling \acf{UV}
dynamics.


