\section{6-\acs{DOF} \acs{UV} Dynamics Model}
\label{chModels.sec.UVSE3plant}

Using the state representation conventions of Section
\ref{chModels.sec.SE3}, we represent the pose and velocity of an
\ac{UV} with $H\in\SE3$ and $v\in\mathbb{R}^6$ respectively.
%
We model a \ac{UV} as a rigid-body with added hydrodynamic mass,
quadratic drag, gravitational force, and buoyancy torque moving under
the influence of external torques $\tau \in \mathbb{R}^{3}$ and forces
$f \in \mathbb{R}^{3}$.
%Using the
%functions defined in Section \ref{chModels.sec.functionDefn} and state
%definitions from Section \ref{chModels.sec.SO3}, 
The commonly accepted second-order finite-dimensional lumped parameter
dynamic model for fully submerged rigid-body \ac{UV}, written in the
body-frame, is

\begin{align} \label{chModels.eq.UVSE3plant}
\dot{H}=&H \widehat{v}
\nonumber \\
  M\dot{v}=&\ad_v^T\left(M v\right)+\left(\sum_{i=1}^6 |v_i|D_i
           \right)v+\mathcal{G}(R)+u 
\end{align}

\noindent where 
%
%\begin{equation}
%mathcal{G}(R)=\left[ \begin{array}{c} g R^T e_3   
%    \\ \mathcal{J}(b)R^T e_3 \\ \end{array}\right],
%\end{equation}
%TM_FOR_SPACE_COULD_MOVE_INLINE
%
%\noindent 
$\mathcal{G}(R)=\left[ \begin{array}{c} g R^T e_3 \\ \mathcal{J}(b)R^T
    e_3 \\ \end{array}\right]$,
$e_3=\left[ \begin{array}{c} 0\\ 0\\
    1\\ \end{array}\right]$, $u=\left[ \begin{array}{c} f \\ \tau
    \\ \end{array}\right]$, $M\in \mathbb{R}^{6 \times 6}$ is the
vehicle's mass matrix, the set $D_i\in \mathbb{R}^{6 \times 6}$
($i=1\cdots6$) are the 6 \ac{DOF} fully-coupled quadratic drag
coefficients, $g\in \mathbb{R}$ is the net vertical force acting on
the vehicle due to gravity and buoyancy (i.e. the net buoyancy), and
$b\in\mathbb{R}^3$ is the torque applied to the vehicle due to the
relative positions of the \ac{COG} and \ac{COB} (which will vary as a
function of the vehicle's roll and pitch)\cite{fossen}.

Although the model (\ref{chModels.eq.UVSE3plant}) is derived
empirically, its structure is well established in the literature
\cite{fossen}.  Previous studies have demonstrated this model's
capacity to approximate \ac{UV} dynamics following typical operational
trajectories and justified this model's exclusion of a linear drag
term\cite{martinID_ICRA13}.  The parameters $M$, $D$, $g$, and $b$ are
expected to have the following properties:
\begin{itemize}
\item the mass matrix $M$ is \ac{SPD}, the sum of the vehicle's
  rigid-body mass matrix and its hydrodynamic added-mass matrix;
\item the scalar $g$ is the net difference of the
  force of gravity and force of buoyancy on the vehicle and is reported in newtons;
\item the vector $b\in\mathbb{R}^3$ is the
  body-frame \ac{COB} position multiplied by the force of buoyancy
  added to the body-frame \ac{COG} position multiplied by the force of
  gravity and is reported in newton meters; and
\item the quadratic drag $D$ is dissipative (i.e. $v^T \left(\sum_{i=1}^6
    |v_i|D_i\right)v\leq 0$ for all $v\in\mathbb{R}^6$ or,
  equivalently, the symmetric part of $\sum_{i=1}^6 |v_i|D_i$ is
  negative definite).
\end{itemize}
%
%Note that (\ref{chModels.eq.UVSE3plant}) is linear in $M$, $D$, $g$
%and $b$, thus it will be convenient to define a single parameter
%vector, $\theta\in\realSpace{241}$. Let $\theta_M\in\realSpace{21}$ be
%the unique values in the symmetric matrix $M$. Then there exists a
%unique $P\in\realSpace{36\times 21}$ such that $M^S=P\theta_M$, where
%$M^S$ is the stack operation on the matrix $M$\cite{kron}. The
%plant parameter vector $\theta$ is defined as

An alternate form of \ac{UV} dynamics can be developed using a
regressor matrix since the second equality in
(\ref{chModels.eq.UVSE3plant}) is linear in $M$, $D$, $g$, and $b$.
Let $\{m_1,\;\cdots\;,m_{21}\}$ be the 21 unique scalar values in $M$
such that
%
\begin{equation}
\ssp
  M=\left[\begin{array}{cccccc}
      m_{1} & m_{2} & m_{3} & m_{4} & m_{5} & m_{6} \\
      m_{2} & m_{7} & m_{8} & m_{9} & m_{10} & m_{11} \\
      m_{3} & m_{8} & m_{12} & m_{13} & m_{14} & m_{15} \\
      m_{4} & m_{9} & m_{13} & m_{16} & m_{17} & m_{18} \\
      m_{5} & m_{10} & m_{14} & m_{17} & m_{19} & m_{20} \\
      m_{6} & m_{11} & m_{15} & m_{18} & m_{20} & m_{21} 
\end{array}\right].  
\end{equation}
%
Defining the matrix $P_M$ as
%
\begin{equation}
\ssp
  P_M=\left[ \begin{array}{ccccccccccccccccccccc}
      {\color{red}1} & 0 & 0 & 0 & 0 & 0 & 0 & 0 & 0 & 0 & 0 & 0 & 0 & 0 & 0 & 0 & 0 & 0 & 0 & 0 & 0 \\
      0 & {\color{red}1} & 0 & 0 & 0 & 0 & 0 & 0 & 0 & 0 & 0 & 0 & 0 & 0 & 0 & 0 & 0 & 0 & 0 & 0 & 0 \\
      0 & 0 & {\color{red}1} & 0 & 0 & 0 & 0 & 0 & 0 & 0 & 0 & 0 & 0 & 0 & 0 & 0 & 0 & 0 & 0 & 0 & 0 \\
      0 & 0 & 0 & {\color{red}1} & 0 & 0 & 0 & 0 & 0 & 0 & 0 & 0 & 0 & 0 & 0 & 0 & 0 & 0 & 0 & 0 & 0 \\
      0 & 0 & 0 & 0 & {\color{red}1} & 0 & 0 & 0 & 0 & 0 & 0 & 0 & 0 & 0 & 0 & 0 & 0 & 0 & 0 & 0 & 0 \\
      0 & 0 & 0 & 0 & 0 & {\color{red}1} & 0 & 0 & 0 & 0 & 0 & 0 & 0 & 0 & 0 & 0 & 0 & 0 & 0 & 0 & 0 \\
      0 & {\color{red}1} & 0 & 0 & 0 & 0 & 0 & 0 & 0 & 0 & 0 & 0 & 0 & 0 & 0 & 0 & 0 & 0 & 0 & 0 & 0 \\
      0 & 0 & 0 & 0 & 0 & 0 & {\color{red}1} & 0 & 0 & 0 & 0 & 0 & 0 & 0 & 0 & 0 & 0 & 0 & 0 & 0 & 0 \\
      0 & 0 & 0 & 0 & 0 & 0 & 0 & {\color{red}1} & 0 & 0 & 0 & 0 & 0 & 0 & 0 & 0 & 0 & 0 & 0 & 0 & 0 \\
      0 & 0 & 0 & 0 & 0 & 0 & 0 & 0 & {\color{red}1} & 0 & 0 & 0 & 0 & 0 & 0 & 0 & 0 & 0 & 0 & 0 & 0 \\
      0 & 0 & 0 & 0 & 0 & 0 & 0 & 0 & 0 & {\color{red}1} & 0 & 0 & 0 & 0 & 0 & 0 & 0 & 0 & 0 & 0 & 0 \\
      0 & 0 & 0 & 0 & 0 & 0 & 0 & 0 & 0 & 0 & {\color{red}1} & 0 & 0 & 0 & 0 & 0 & 0 & 0 & 0 & 0 & 0 \\
      0 & 0 & {\color{red}1} & 0 & 0 & 0 & 0 & 0 & 0 & 0 & 0 & 0 & 0 & 0 & 0 & 0 & 0 & 0 & 0 & 0 & 0 \\
      0 & 0 & 0 & 0 & 0 & 0 & 0 & {\color{red}1} & 0 & 0 & 0 & 0 & 0 & 0 & 0 & 0 & 0 & 0 & 0 & 0 & 0 \\
      0 & 0 & 0 & 0 & 0 & 0 & 0 & 0 & 0 & 0 & 0 & {\color{red}1} & 0 & 0 & 0 & 0 & 0 & 0 & 0 & 0 & 0 \\
      0 & 0 & 0 & 0 & 0 & 0 & 0 & 0 & 0 & 0 & 0 & 0 & {\color{red}1} & 0 & 0 & 0 & 0 & 0 & 0 & 0 & 0 \\
      0 & 0 & 0 & 0 & 0 & 0 & 0 & 0 & 0 & 0 & 0 & 0 & 0 & {\color{red}1} & 0 & 0 & 0 & 0 & 0 & 0 & 0 \\
      0 & 0 & 0 & 0 & 0 & 0 & 0 & 0 & 0 & 0 & 0 & 0 & 0 & 0 & {\color{red}1} & 0 & 0 & 0 & 0 & 0 & 0 \\
      0 & 0 & 0 & {\color{red}1} & 0 & 0 & 0 & 0 & 0 & 0 & 0 & 0 & 0 & 0 & 0 & 0 & 0 & 0 & 0 & 0 & 0 \\
      0 & 0 & 0 & 0 & 0 & 0 & 0 & 0 & {\color{red}1} & 0 & 0 & 0 & 0 & 0 & 0 & 0 & 0 & 0 & 0 & 0 & 0 \\
      0 & 0 & 0 & 0 & 0 & 0 & 0 & 0 & 0 & 0 & 0 & 0 & {\color{red}1} & 0 & 0 & 0 & 0 & 0 & 0 & 0 & 0 \\
      0 & 0 & 0 & 0 & 0 & 0 & 0 & 0 & 0 & 0 & 0 & 0 & 0 & 0 & 0 & {\color{red}1} & 0 & 0 & 0 & 0 & 0 \\
      0 & 0 & 0 & 0 & 0 & 0 & 0 & 0 & 0 & 0 & 0 & 0 & 0 & 0 & 0 & 0 & {\color{red}1} & 0 & 0 & 0 & 0 \\
      0 & 0 & 0 & 0 & 0 & 0 & 0 & 0 & 0 & 0 & 0 & 0 & 0 & 0 & 0 & 0 & 0 & {\color{red}1} & 0 & 0 & 0 \\
      0 & 0 & 0 & 0 & {\color{red}1} & 0 & 0 & 0 & 0 & 0 & 0 & 0 & 0 & 0 & 0 & 0 & 0 & 0 & 0 & 0 & 0 \\
      0 & 0 & 0 & 0 & 0 & 0 & 0 & 0 & 0 & {\color{red}1} & 0 & 0 & 0 & 0 & 0 & 0 & 0 & 0 & 0 & 0 & 0 \\
      0 & 0 & 0 & 0 & 0 & 0 & 0 & 0 & 0 & 0 & 0 & 0 & 0 & {\color{red}1} & 0 & 0 & 0 & 0 & 0 & 0 & 0 \\
      0 & 0 & 0 & 0 & 0 & 0 & 0 & 0 & 0 & 0 & 0 & 0 & 0 & 0 & 0 & 0 & {\color{red}1} & 0 & 0 & 0 & 0 \\
      0 & 0 & 0 & 0 & 0 & 0 & 0 & 0 & 0 & 0 & 0 & 0 & 0 & 0 & 0 & 0 & 0 & 0 & {\color{red}1} & 0 & 0 \\
      0 & 0 & 0 & 0 & 0 & 0 & 0 & 0 & 0 & 0 & 0 & 0 & 0 & 0 & 0 & 0 & 0 & 0 & 0 & {\color{red}1} & 0 \\
      0 & 0 & 0 & 0 & 0 & {\color{red}1} & 0 & 0 & 0 & 0 & 0 & 0 & 0 & 0 & 0 & 0 & 0 & 0 & 0 & 0 & 0 \\
      0 & 0 & 0 & 0 & 0 & 0 & 0 & 0 & 0 & 0 & {\color{red}1} & 0 & 0 & 0 & 0 & 0 & 0 & 0 & 0 & 0 & 0 \\
      0 & 0 & 0 & 0 & 0 & 0 & 0 & 0 & 0 & 0 & 0 & 0 & 0 & 0 & {\color{red}1} & 0 & 0 & 0 & 0 & 0 & 0 \\
      0 & 0 & 0 & 0 & 0 & 0 & 0 & 0 & 0 & 0 & 0 & 0 & 0 & 0 & 0 & 0 & 0 & {\color{red}1} & 0 & 0 & 0 \\
      0 & 0 & 0 & 0 & 0 & 0 & 0 & 0 & 0 & 0 & 0 & 0 & 0 & 0 & 0 & 0 & 0 & 0 & 0 & {\color{red}1} & 0 \\
      0 & 0 & 0 & 0 & 0 & 0 & 0 & 0 & 0 & 0 & 0 & 0 & 0 & 0 & 0 & 0 & 0 & 0 & 0 & 0 & {\color{red}1} \\
   \end{array}\right],
\end{equation}
%
and $\theta_M$ as
%
%\begin{equation}
%\theta_M=\left[m_{1}\;m_{2}\;m_{3}\;m_{4}\;m_{5}\;m_{6}\;m_{7}\;m_{8}\;m_{9}\;m_{10}\;
%              m_{11}\;m_{12}\;m_{13}\;m_{14}\;m_{15}\;m_{16}\;m_{17}\;m_{18}\;m_{19}\;m_{20}\;
%              m_{21}\right]^T
%\end{equation} 
%
a vector of the unique scalar terms $m_i$ from $M$ in numerical order,
%
we have the relation 
%
\begin{equation}
M^S=P_M \theta_M.
\end{equation}
%
Let the $UV$ subscript denotes 6-\ac{DOF} \ac{UV} dynamics and define $\theta_{UV}$ as
%
\begin{equation}\label{chModels.eq.paramVecUV}
\theta_{UV}=\left[\begin{array}{cccccccccc}\theta_M^T&(D_1^S)^T&(D_2^S)^T&(D_3^S)^T
&(D_4^S)^T&(D_5^S)^T&(D_6^S)^T&g&b^T\end{array}\right]^T.
\end{equation}
%
Then the second equality in (\ref{chModels.eq.UVSE3plant}) can be factored as 
%
\begin{align}  
  u(t)&=\left[\begin{array}{ccc} \Mreg(v,\dot{v})&\Dreg{}(v)&\gbreg(R)\end{array}\right]\theta
\nonumber \\
      &=\UVreg(v,\dot{v},R)\theta_{UV}
\label{chModels.eq.UVSE3_LS} 
\end{align}
%
using the following definitions
%
\begin{itemize}
\item $\Mreg(v,\dot{v})=\left(\dot{v}(t)^T \otimes \mathbb{I}-v^T\otimes \ad^T(v)\right)P_M$,
\item $\Dreg{i}(v)=-|v_i|v^T \otimes \mathbb{I}$ for all $i \in \{1,\cdots,6\}$, 
\item $\Dreg{}(v)=\left[\begin{array}{cccccc} \Dreg{1}(v)&\Dreg{2}(v)&\Dreg{3}(v)&\Dreg{4}(v)&\Dreg{5}(v)&\Dreg{6}(v)\end{array}\right]$,
\item $\gbreg(R)=\left[\begin{array}{cc} -R^Te_3 & 0_{3\times 3}\\ 
                                   0_{3 \times 1} & \mathcal{J}(R^T e_3)\end{array}\right]$, and 
\item $\UVreg=\left[\begin{array}{ccc} \Mreg(v,\dot{v})&\Dreg{}(v)&\gbreg(R)\end{array}\right]$.
\end{itemize}
%



%TM_FIT_IN
%
%scrub of talking about linearity and instead talk about factoring...
%MATCH LANGUAGE FROM OKC MODEL SECTION.
%
%Introduce P_M
%
%
% A least squares estimate of the parameters in
% (\ref{chModels.eq.UVSE3plant}) can be obtained using the signals
% $\dot{v}(t)$, $v(t)$, $R(t)$ and $u(t)$. Let
% $\theta_M\in\mathbb{R}^{21}$ be the unique values in the symmetric
% matrix $M$. Then there exists a unique $P\in\mathbb{R}^{36\times 21}$
% such that $M^S=P\theta_M$, where $M^S$ is the stack operation on the
% matrix $M$\cite{kron}.  Note that $P$ is composed only of ones and
% zeros, and only has a single one per row.  Using the Kronecker product
% \cite{kron} we define the following functions



% \noindent where $\mathbb{I}\in\mathbb{R}^{6\times 6}$ is the identity
% matrix. Using the plant parameter vector
% $\theta=\left[\begin{array}{cccccc}
%     \theta_M^T&(D_1^S)^T&\cdots&(D_6^S)^T&g&b^T\end{array}\right]^T$,
% (\ref{chModels.eq.UVSE3plant}) becomes


