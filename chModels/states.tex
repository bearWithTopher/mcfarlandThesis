\section{State Representations}

\subsection{Rotating Rigid-Body Kinematics}
%\subsection{Special Orthogonal Group}
\label{chModels.sec.SO3}

\noindent We use three state representations of rigid-body angular
position and velocity:
\begin{itemize}
\item $q\in\mathbb{R}^3$, the angular position in so(3)
exponential coordinates,
\item $\omega\in\mathbb{R}^3$, the body-frame angular velocity, and
\item $R\in\SO3$, the rotation matrix from the body-frame to the
  world-frame.
\end{itemize}
%
The relations between these state representations are given by
%
\begin{align}
R=&e^{\mathcal{J}(q)}  \\
\dot{R}=&R\mathcal{J}(\omega).
\end{align}

%------------------old version------------------------------
% We represent the angular position of the rigid body in so(3)
% exponential coordinates as $q\in\mathbb{R}^3$ and it's angular
% velocity as $\omega\in\mathbb{R}^3$.
% %
% Angular position can also be represented as a rotation matrix
% from a body-frame to a world-frame, $R\in\SO3$.
% %
% The relations between these state representations are

% \begin{align}
% R=&e^{\mathcal{J}(q)}  \\
% \dot{R}=&R\mathcal{J}(\omega).
% \end{align}



\noindent Throughout this study we make use of the relation between
the body-frame angular velocity, $\omega$, and the time derivative of
the exponential coordinate position, $\dot{q}$, which takes the form
%
\begin{equation}
\omega=\mathcal{A}(q) \dot{q}.
\end{equation}
%
The closed form for the Jacobian \funDefn{A}{\rSp{3}}{\rSp{3\times 3}}
given by
%
\begin{equation}\label{chModels.eq.A}
A(q)=I+\left(\frac{1-\cos{\|q\|}}{\|q\|}\right)\frac{\mathcal{J}(q)}{\|q\|}+
\left(1-\frac{\sin{\|q\|}}{\|q\|}\right)\frac{\mathcal{J}(q)^2}{\|q\|^2}.
\end{equation}
%
was first reported by Park in \cite{Park1991}. The inverse of this mapping,
%
\begin{equation}
\dot{q}=\mathcal{A}^{-1}(q)\omega,
\end{equation}
%
\funDefn{A^{-1}}{\rSp{3}}{\rSp{3\times 3}} also has a closed functional
  form given by
%
\begin{equation}\label{chModels.eq.Ainv}
  \mathcal{A}^{-1}(q)=I_{3\times 3}-\frac{1}{2}\mathcal{J}(q)+\left(1-\frac{\|q\|}{2}\cot{\frac{\|q\|}{2}}\right)\frac{\mathcal{J}(q)^2}{\|q\|^2}.
\end{equation}
%
This inverse exists for $\|q\|<\pi$ \cite{Park1991}.  See
\cite{Bullo2004,Chirikjian2000} for additional properties.


\subsection{Rotating and Translating Rigid-Body Kinematics}
%\subsection{Special Euclidean Group}
\label{chModels.sec.SE3}

\noindent We use seven state representations of rigid-body pose and velocity:
\begin{itemize}
\item $\psi\in\mathbb{R}^6$, the pose in se(3) exponential
coordinates,
\item $v\in\mathbb{R}^6$, the body-frame velocity,   
\item $H\in\SE3$, the homogenious transform from the body-frame to the world-frame,
\item $R\in\SO3$, the rotation matrix from the body-frame to the
  world-frame,
\item $p\in\mathbb{R}^3$, the vector representing the body-frame's
  origin in the world-frame,
\item $\nu\in\mathbb{R}^3$, the vehicle's body-frame translational
  velocity, and
\item $\omega\in\mathbb{R}^6$, the vehicle's body-frame angular
  velocity.
\end{itemize} 
%
%Many of our analytic results rely on representing the vehicle's state
%in exponential coordinates, $\psi$, and body-frame velocity, $v$.
%
The relations between these state representations are given by
%
\begin{align}
H=&e^{\widehat{\psi}}, \\
\dot{H}=&H\widehat{v}, \\
H=&\left[ \begin{array}{cc}
         R      & p  \\
    0_{1 \times 3}  & 1  \\
  \end{array} \right],~\text{and} \\
v=&\left[ \begin{array}{c} \nu \\  \omega \\ \end{array} \right].
\end{align}
% %----------------Old Version of-------------------------------- 
% We represent the pose of the rigid body in se(3) exponential
% coordinates as $\psi\in\mathbb{R}^6$ and its body-frame velocity using
% $v\in\mathbb{R}^6$. 
% %
% Rigid body pose can also be represented as a homogenious transform 
% from the body-frame to the world-frame, $H\in\SE3$.
% %
% The relations between these state representations are
% %
% \begin{align}
% H=&e^{\widehat{\psi}} \\
% \dot{H}=&H\widehat{v}.
% \end{align}
% %
% Further, note that for $R\in\SO3$, the rotation matrix from
% the body-frame to the world-frame, and $p\in\mathbb{R}^3$, the vector
% representing the location of body-frame's origin in the world-frame,
% we have the equality
% %
% $H=\left[ \begin{array}{cc}
%          R      & p  \\
%     0_{1 \times 3}  & 1  \\
%   \end{array} \right]$.
% %
% Finally, note that for $\nu\in\mathbb{R}^3$, the vehicle's
% body-frame translational velocity, and $\omega\in\mathbb{R}^6$, the
% vehicle's body-frame angular velocity, we have the equality
% %
% $v=\left[ \begin{array}{c} \nu \\  \omega \\ \end{array} \right]$.


We employ the inverse SE(3) velocity Jacobian,
\funDefn{\hat{\mathcal{A}}^{-1}}{\rSp{6}}{\rSp{6\times 6}}, as a
relation between the body-frame velocity and time derivative of
exponential coordinate pose
%
\begin{equation}\label{chModels.eq.hatAInv}
\dot{\psi}=\hat{\mathcal{A}}^{-1}(\psi) v.
\end{equation}
%
In \cite{bullo1995_SE3_PD} the authors derive a closed form expression
for this matrix valued function, reprinted in Appendix
\ref{appenJacSE3} as (\ref{appenJacSE3.eq.hatAinv}).
%
To the best of the author's knowledge a closed form expression for the
SE(3) angular velocity Jacobian, $\hat{\mathcal{A}}\left(\psi\right)$,
has not been reported.
%
Appendix \ref{appenJacSE3} provides further information on this
Jacobian, including the derivation of a simpler closed form
(\ref{appenJacSE3.eq.hatAinv2}), proof that
$\psi^T\left(\hat{\mathcal{A}}^{-T}(\psi)+\hat{\mathcal{A}}^{-1}(\psi)\right)\psi=\psi^T\psi$
(Appendix \ref{appenJacSE3.sec.PsiHatAinvPsiEquality}), and an
upper bound for $\|\hat{\mathcal{A}}^{-1}(\psi)x\|$ (Appendix
\ref{appenJacSE3.sec.normBound}).






