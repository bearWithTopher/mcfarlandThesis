\subsection{3-\acs{DOF} Rotational Dynamics Model}
\label{chModels.sec.SO3plant}


%Using the functions defined in Section \ref{chModels.sec.functionDefn} and
%state definitions from Section \ref{chModels.sec.SO3}, 
The commonly accepted model of a rigid-body rotating under the
influence of an external torque $\tau \in \mathbb{R}^{3}$ is given by

\begin{align} \label{chModels.eq.SO3plant}
\dot{R} &=R\widehat{\omega}                         \nonumber \\
\dot{\omega} &=I^{-1}\left(\mathcal{J}\left(I\omega\right)\omega+\tau\right) 
\end{align}

\noindent where $\omega \in \mathbb{R}^{3}$ is the plant's angular
velocity, $R\in\SO3$ is the plant's rotational position, and
$I\in\mathbb{R}^{3\times 3}$ is the plant's \ac{SPD} inertia
tensor\cite{Taylor2005}. 


An alternate form of plant dynamics can be developed using a regressor
matrix since the second equality in (\ref{chModels.eq.SO3plant}) is
linear $I$, the plant's inertia tensor.
%
Let
$\lambda_1,\;\cdots,\;\lambda_6\in\rSp{}$
be the six unique scalars in $I$ such that 
%
{\ssp
$I=\left[\begin{array}{ccc}
\lambda_1 & \lambda_2 & \lambda_3 \\
\lambda_2 & \lambda_4 & \lambda_5 \\
\lambda_3 & \lambda_5 & \lambda_6 
\end{array}\right]$.} 
%
%A least squares estimate of $I$ can be
%obtained using the signals $\dot{\omega}(t)$, $\omega(t)$, and
%$\tau(t)$.  
Defining the matrix $P_I$ as
%
\begin{equation}
\ssp
  P_I=\left[ \begin{array}{cccccc}
      {\color{red}1} & 0 & 0 & 0 & 0 & 0 \\
      0 & {\color{red}1} & 0 & 0 & 0 & 0 \\
      0 & 0 & {\color{red}1} & 0 & 0 & 0 \\
      0 & {\color{red}1} & 0 & 0 & 0 & 0 \\
      0 & 0 & 0 & {\color{red}1} & 0 & 0 \\
      0 & 0 & 0 & 0 & {\color{red}1} & 0 \\
      0 & 0 & {\color{red}1} & 0 & 0 & 0 \\
      0 & 0 & 0 & 0 & {\color{red}1} & 0 \\
      0 & 0 & 0 & 0 & 0 & {\color{red}1} \\
    \end{array}\right],
\end{equation}
%
and $\theta_I$ as
\begin{equation}
\theta_I=\left[\begin{array}{cccccc}
\lambda_1 &
\lambda_2 &
\lambda_3 &
\lambda_4 &
\lambda_5 &
\lambda_6 
\end{array}\right]^T
\end{equation}   
%
%\begin{equation}
%\theta_I=\left[\begin{array}{c}
%\lambda_1 \\
%\lambda_2 \\
%\lambda_3 \\
%\lambda_4 \\
%\lambda_5 \\
%\lambda_6 \\
%\end{array}\right]
%\end{equation}   
%
we have the relation 
%
\begin{equation}\label{chModels.eq.IstackToTheta}
I^S=P_I\theta_I.
\end{equation}
%
The second equality in (\ref{chModels.eq.SO3plant}) can be factored as 
%
\begin{align}
  \tau(t)&= \left(\dot{\omega}(t)^T \otimes \mathbb{I}+\omega(t)^T
    \otimes \mathcal{J}(\omega(t))\right)P_I \theta_I        \nonumber \\
  &=\Ireg(\omega,\dot{\omega}) \theta_I, \label{chModels.eq.SO3_LS}
\end{align}
%
\noindent where $\mathbb{I}\in\mathbb{R}^{3\times 3}$ is the identity
matrix and  
%
\begin{equation}
\Ireg(\omega,\dot{\omega})=\dot{\omega}(t)^T \otimes
\mathbb{I}+\omega(t)^T \otimes \mathcal{J}(\omega(t))P_I.
\end{equation}