\subsection{3-\acs{DOF} \acs{UV} Rotational Dynamics Model}
\label{chModels.sec.UVSO3plant}

Modeling a submerged rotating rigid-body requires accounting for the
surrounding fluid.
%
The most commonly employed plant models for \ac{UV} are
finite dimensional lumped-parameter approximate models with
vehicle-specific plant parameter terms including mass and added mass
parameters; quadratic drag parameters; and gravity and buoyancy
parameters.  
%
Previous studies have shown that including explicit terms
for the quadratic drag and buoyancy torque within
(\ref{chModels.eq.SO3plant}) results in a model which approximates
the dynamics of a rotating \ac{UV}\cite{martinID_ICRA13}.
%Using the functions defined in Section \ref{chModels.sec.functionDefn} and state
%definitions from Section \ref{chModels.sec.SO3}, 
Therefore, we model the rotational dynamics of an \ac{UV} as
%
\begin{align} \label{chModels.eq.UVSO3plant}
\dot{R}&=R\mathcal{J}(\omega)
\nonumber \\
  \dot{\omega}&=I^{-1}\left(\mathcal{J}(I\omega)\omega+\left(\sum_{i=1}^3 |\omega_i|C_i
    \right)\omega+\mathcal{J}(b)R^T e_3+\tau\right) 
\end{align}
%
\noindent where $I$ is the \ac{UV}'s inertia tensor,
$e_3=\left[ \begin{array}{ccc} 0& 0& 1\\ \end{array}\right]^T$,
$C_1,C_2,C_3\in \mathbb{R}^{3\times 3}$ make up a general three \ac{DOF}
coupled quadratic drag matrix, and $b\in\mathbb{R}^3$ represents the
torque applied to the vehicle due to the relative positions of the
\ac{UV}'s \ac{COG} and \ac{COB}.
%The
%length of $b$ characterizes the torque experienced by the vehicle when
%the center of buoyancy is not directly on top of the center of
%gravity.  
Quadratic drag is assumed to be dissipative
thus, $\omega^T \left(\sum_{i=1}^3
  |\omega_i|C_i\right)\omega\leq 0$ for all $\omega\in\mathbb{R}^3$
--- i.e. the symmetric part of $\sum_{i=1}^3 |\omega_i|C_i$ is negative
definite.


The second equality in (\ref{chModels.eq.UVSO3plant}) is linear in the
plant parameters. Thus, an alternate form of \ac{UV} rotational
dynamics can be developed using a regressor matrix.  Let the $UVR$
subscript denote {\it\ac{UV} rotational dynamics}
%
%a least squares estimate of thes
%parameters in  can be obtained using the
%signals $\dot{\omega}(t)$, $\omega(t)$, $R(t)$ and $u(t)$.  Note that
%(\ref{chModels.eq.UVSO3plant}) can be rewritten as
%
and define $\theta_{UVR}\in\rSp{36}$ as 
%
%\begin{equation}
%\theta_{UVR}=\left[\begin{array}{c}
%\theta_I \\
%C_1^S \\
%C_2^S \\
%C_3^S \\
%b
%\end{array}\right]
%\end{equation}
%
\begin{equation}\label{chModels.eq.paramVecUVR}
\theta_{UVR}=\left[\theta_I^T\; (C_1^S)^T\; (C_2)^T \; (C_3^S)^T \; b^T\right]^T.
\end{equation}
%
Then the second equality in (\ref{chModels.eq.UVSO3plant}) can be factored as
%
\begin{widetext}
\begin{align}
  \tau(t)&= \left[ \begin{array}{ccccc} 
 \Ireg(\omega,\dot{\omega})              &
-|\omega_1|\omega^T \otimes \mathbb{I} &
-|\omega_2|\omega^T \otimes \mathbb{I} &
-|\omega_3|\omega^T \otimes \mathbb{I} &
\mathcal{J}(R^T e_3)\\ \end{array} \right] \theta_{UVR}        \nonumber \\
      &=\UVRreg(\omega,\dot{\omega},R) \theta_{UVR}               \label{chModels.eq.UVSO3_LS} 
\end{align}
%
where
%
\begin{equation}
\UVRreg=\left[ \begin{array}{ccccc} 
 \Ireg(\omega,\dot{\omega})              &
-|\omega_1|\omega^T \otimes \mathbb{I} &
-|\omega_2|\omega^T \otimes \mathbb{I} &
-|\omega_3|\omega^T \otimes \mathbb{I} &
\mathcal{J}(R^T e_3)\\ \end{array} \right].
\end{equation}
\end{widetext}
