\section{Notation Conventions}
\label{chModels.sec.notation}

We assume the readers are familiar with SO(3), the special orthogonal
group for $\rSp{3}$, and SE(3), the special euclidean group for
$\rSp{3}$.  We also assume readers are familiar with their tangent
spaces so(3) and se(3) respectively.  Readers not familiar with these
concepts are referred to \cite{murray&li&sastry}.

\subsection{Function Definitions}
\label{chModels.sec.functionDefn}

\noindent Functions used throughout this text are defined as follows:
%
\begin{itemize}
\item $\mathcal{J}:\mathbb{R}^3\to\mathbb{R}^{3\times 3}$ is
the mapping from $\realSpace{3}$ to so(3), the tangent space of SO(3),
and is defined by
%
\begin{equation}
\mathcal{J}\left(
\left[ \begin{array}{c} 
\omega_1 \\\omega_2  \\ \omega_3 \\
\end{array} \right]\right) =
\left[ \begin{array}{ccc}
     0     & -\omega_3 &  \omega_2                               \\
   \omega_3  &    0    & -\omega_1                               \\
  -\omega_2  &  \omega_1 &    0                                  \\
\end{array} \right].
\end{equation}
%
%
\item $\widehat{\cdot}:\mathbb{R}^6\to\mathbb{R}^{4\times 4}$ is the
  mapping from $\mathbb{R}^6$ to se(3), the tangent space of SE(3),
  and (for $\nu,\omega\in\rSp{3}$) is defined by
%
\begin{equation}
\widehat{
\left[ \begin{array}{c} 
\nu \\ \omega  \\
\end{array} \right] } =
\left[ \begin{array}{cc}
     \mathcal{J}(\omega) & \nu        \\
        0_{1 \times 3}     &  0         \\
\end{array} \right].
\end{equation}
%
%
\item $\ad:\mathbb{R}^6\to\mathbb{R}^{6\times 6}$ is the se(3) adjoint
operator. For the vector 
%
\begin{equation}
v=\left[ \begin{array}{c} 
\nu \\ \omega  \\
\end{array} \right]
\end{equation} 
%
the se(3) adjoint operator is defined as
%
\begin{equation}
\label{chModels.eq.ad}
\ad_v=
\left[ \begin{array}{cc}
     \mathcal{J}(\omega) &  \mathcal{J}(\nu)        \\
      0_{3 \times 3}     & \mathcal{J}(\omega)    \\
\end{array} \right].
\end{equation}
%
%
\item $\Ad:\SE3\to\mathbb{R}^{6\times 6}$ is the se(3) adjoint map,  defined as
%
\begin{equation}
\label{chModels.eq.Ad}
\Ad\left(
\left[ \begin{array}{cc} 
R &  p \\ 
0_{1\times 3} &  1 \\
\end{array} \right]\right) =
\left[ \begin{array}{cc}
             R           & \mathcal{J}(p) R   \\
      0_{3 \times 3}        &    R  \\
\end{array} \right].
\end{equation}
%
%
\item \funDefn{\otimes}{\realSpace{m\times n}\times\realSpace{p\times q}}{\realSpace{(m*p)\times (n*q)}}
is the Kronecker product operator\cite{kron}.  
%
For matrices 
%
\begin{equation}
A=\left[ \begin{array}{cccc} 
a_{11} & a_{12} &\hdots & a_{1n}  \\
a_{21} & a_{22} &       & a_{2n}  \\
\vdots &      &\ddots & \vdots \\
a_{m1} & a_{m2} &\hdots & a_{mn}  \\
\end{array} \right]
%
\text{ and }
%
B=\left[ \begin{array}{cccc} 
b_{11} & b_{12} &\hdots & b_{1q}  \\
b_{21} & b_{22} &       & b_{2q}  \\
\vdots &      &\ddots & \vdots \\
b_{p1} & b_{p2} &\hdots & b_{pq}  \\
\end{array} \right]
\end{equation}
%
the Kronecker product is defined as
%
\vspace{5mm}
\begin{equation}
A\otimes B=\left[ \begin{array}{cccc} 
a_{11}B & a_{12}B & \hdots & a_{1n}B  \\
a_{21}B & a_{22}B &        & a_{2n}B  \\
\vdots &        & \ddots & \vdots  \\
a_{m1}B & a_{m2}B & \hdots & a_{mn}B  \\
\end{array} \right].
\end{equation}
%
%
\item \funDefn{\cdot^S}{\realSpace{m\times n}}{\realSpace{(m*n)}} is the
stack operator.  Using the definition of $A$ above, the stack operator is
defined by stacking the columns of $A$ such that  
%
\begin{equation}
A^S=\left[ \begin{array}{ccccccccccccc} 
a_{11}  &
a_{21}  &
\cdots &
a_{m1}  &
a_{12}  &
a_{22}  &
\cdots &
a_{m2}  & 
\cdots &
a_{1n}  &
a_{2n}  &
\cdots &
a_{mn}
\end{array} \right]^T.
\end{equation}
%
%\begin{equation}
%A^S=\left[ \begin{array}{c} 
%a_{11}  \\
%a_{21}  \\
%\vdots \\
%a_{m1}  \\
%a_{12}  \\
%a_{22}  \\
%\vdots \\
%a_{m2}  \\ 
%\vdots \\
%a_{1n}  \\
%a_{2n}  \\
%\vdots \\
%a_{mn}
%\end{array} \right].
%\end{equation}
\end{itemize}
%
This Thesis will also make use of 
%
\begin{itemize}
\item the SO(3) exponential map, \funDefn{\e_{\SO3}}{\so3}{\SO3} 
\item the SO(3) logarithmic map, \funDefn{\log_{SO3}}{\SO3}{\so3}
\item the SE(3) exponential map, \funDefn{\e_{\SE3}}{\se3}{\SE3} 
\item the SE(3) logarithmic map, \funDefn{\log_{SE3}}{\SE3}{\se3}
\end{itemize}
% 
See \cite{murray&li&sastry} for additional information, including
closed form functions, for these maps.

\subsection{Vector Norm, Matrix Norm, and Eigenvalue  Conventions}
\label{chModels.sec.normConventions}

Let $M\in\mathbb{R}^{n\times n}$ represent a \ac{SPD} inertia tensor
or hydrodynamic mass matrix. Note that the eigenvalues of symmetric
matrices are always real.  This Thesis employs the following
conventions for the eigenvalues of such matrices: $m_n$ is the
smallest and $m_1$ is the largest eigenvalues of the mass matrix, the
other eigenvalues are labeled such that $m_{i-1}\leq m_i\leq
m_{i+1}$ $\forall i$ such that $2\leq i\leq n-1$. This convention will also be used for any mass matrix
estimate or mass matrix error term, i.e. if $\hat{M}(t)$ is an
estimate of a true mass matrix $M$ and $\Delta M(t)=\hat{M}(t)-M$ then
these eigenvalues will be ordered such that $\hat{m}_{i-1}(t)\leq
\hat{m}_i(t) \leq \hat{m}_{i+1}(t)$ and $\Delta m_{i-1}(t)\leq \Delta
m_i(t) \leq \Delta m_{i+1}(t)$. We assume that the eigenvalues of
physically realistic mass matrices are either constant or bounded for
all time, i.e. there exists a scalar $a\in\mathbb{R}_+$ such that
$m_1<a$.



The $\ell 2$ norm, or Euclidean norm, for $x\in\mathbb{R}^n$ is
defined as $\|x\|=\left(\sum_{i=1}^n x_i^2\right)^{1/2}$.  The $\ell 2$ induced
matrix norm, also known as the spectral norm, for $M\in\mathbb{R}^{n
  \times n}$ is defined as $\|M\|_2=\max_{\|x\|=1}\|Mx\|$.  Let
$a_{ij}$ be the individual entries of $A\in\mathbb{R}^{n\times n}$,
the Frobenius norm of $A$ is defined as
$\|A\|_F=\left(\sum_{i,j=1}^n|a_{ij}|^{2} \right)^{1/2}$.  For more
information on these norms see \cite{horn&johnson}. We define the
following matrix semi-norm
%
\begin{equation}
\minNorm{M}=\min_{\|x\|=1}\|Mx\|.
\end{equation}
%
For a given mass matrix $M$, these definitions and the cross product
property $\|x_1\times x_2\|\leq \|x_1\|\|x_2\|$ give rise to the
following properties

\begin{align}\label{chModels.eq.AVO_ineq}
  m_n\|x\|&=\minNorm{M}\|x\|\leq\|Mx\|\leq\|M\|_2\|x\|=m_1\|x\|, 
   \\
  \|-Mx\|&\leq-m_n\|x\|,
   \\
  \|M^{-1/k}\|_2&=\|M^{-1}\|_2^{1/k}=\left(\frac{1}{m_n}\right)^{1/k},
   \\
  \|M L\|_2&\leq\|M\|_2\|L\|_2=m_{1} l_{1},~ \text{and} 
   \\ 
  \|\mathcal{J}(M x_1)x_2\|&\leq\|M x_1\|\|x_2\|\leq m_1\|x_1\|\|x_2\|.
\end{align}


