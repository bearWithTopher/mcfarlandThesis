
\section{UV Model-Based Control}
In this section we report a novel model-based linearizing tracking
controller for plants of the form (\ref{eq.plantROV}).

\subsection{Tracking Control of a UV subject to Drag, Gravity, 
  Buoyancy, External Forces, and Torques}\label{sec.ROVAID}

\noindent Given $\theta$ (the unique scalar parameters from $M$, $D$,
$g$ and $b$ stacked as a vector as defined in
\ref{chModels.eq.plantParamVec}), a desired reference trajectory
(${^w_d}H$, ${^d}v_d$ and ${^d}\dot{v}_d$), and a regressor matrix
$\mathbb{W}:\mathbb{R}^6\times\mathbb{R}^6\times\mathbb{R}^6\times\SE3\to
\mathbb{R}^{6 \times 241}$ defined as

\begin{equation}
\mathbb{W}({^a}\dot{v}_d,\Delta v, {^a}v_d , {^w_a}H)\theta =
    M{^a}\dot{v}_d - M\ad_{\Delta v}{^a}v_d -\ad_{{^a}v_a}^T M{^a}v_a
    -\mathcal{D}({^a}v_a)-\mathcal{G}({^w_a}H),
\end{equation}

\noindent the control law 

\begin{equation}
u=-(k_p\hat{\mathcal{A}}^{-T}(\Delta \psi)+k_d\Delta v)
+\mathbb{W}({^a}\dot{v}_d,\Delta v, {^a}v_d , {^w_a}H)\theta,
\end{equation}

provides asymptotically exact trajectory-tracking for plants of the
form (\ref{eq.plantROV}) as long as {\color{red} $\|\Delta \psi\|_2 <\pi$ }.


\subsection{Controlled System}

The controlled plant will be of the form

\begin{align}
  M{^a}\dot{v}_a=
     &\ad_{{^a}v_a}^T M{^a}v_a+\mathcal{D}({^a}v_a)+\mathcal{G}({^w_a}H)
\nonumber \\ 
      &-(k_p\hat{\mathcal{A}}^{-T}(\Delta \psi)+k_d\Delta v)
+\mathbb{W}({^a}\dot{v}_d,\Delta v, {^a}v_d , {^w_a}H)\theta
\nonumber \\
     =&M{^a}\dot{v}_d - M\ad_{\Delta v}{^a}v_d
      -(k_p\hat{\mathcal{A}}^{-T}(\Delta \psi)+k_d\Delta v)
\nonumber 
\end{align}

\noindent From equation (\ref{chUV_ATMC.eq.delta_v_dot}) we get

\begin{equation}\label{chUV_AMTC.eq.Mdelta_v_dot}
M \dot{\Delta v} =       -(k_p\hat{\mathcal{A}}^{-T}(\Delta \psi)+k_d\Delta v).
\end{equation}

\noindent If we define the system error vector $x\in\mathbb{R}^{12}$

\begin{equation}\label{chVU_AMTC.eq.error_x}
x=\left[ \begin{array}{c}
     \Delta \psi           \\
     \Delta v              \\
\end{array} \right]
\end{equation}

\noindent and the matrix valued function
$\hat{\mathbb{A}}:\mathbb{R}^6\to\mathbb{R}^{12}\times\mathbb{R}^{12}$ as

\begin{equation}\label{chUV_AMTC.eq.error_A}
\hat{\mathbb{A}}(\Delta \psi)=
\left[ \begin{array}{cc}
     0_{6\times 6}   & \hat{\mathcal{A}}^{-1}(\Delta \psi)             \\
     -k_p M^{-1}\hat{\mathcal{A}}^{-T}(\Delta \psi)   &  -k_d M^{-1}   \\
\end{array} \right]
\end{equation}

\noindent then from (\ref{chUV_AMTC.eq.delta_psi_dot}),
(\ref{chUV_AMTC.eq.Mdelta_v_dot}), (\ref{chVU_AMTC.eq.error_x}) and
(\ref{chUV_AMTC.eq.error_A}) we see that the equation for the error
dynamics of the system  is

\begin{equation} \label{chUV_AMTC.eq.err_sys}
\dot{x}=\hat{\mathbb{A}}(\Delta \psi) x.
\end{equation}


\subsection{Convergence Proof}\label{chUV_AMTC.sec.proof_MBC}

Consider the following Lyapunov function candidate

\begin{equation} \label{chUV_AMTC.eq.lyap}
V(t)=\frac{1}{2} x^T\mathcal{M}_\epsilon x
\end{equation}

\noindent where 

\begin{equation}\label{chUV_AMTC.eq.lyapM}
\mathcal{M}_\epsilon=\left[ 
\begin{array}{cc}
  k_p\mathbb{I}_{6\times 6}  & \epsilon M         \\
  \epsilon M               &    M               \\
\end{array} \right].
\end{equation}

\noindent Note that $\forall k_p\in\mathbb{R}_+$ there exists an
$\epsilon_0\in\mathbb{R_+}$ such that $\forall
\epsilon\in\mathbb{R}_+$ for which $\epsilon < \epsilon_0$ the
following hold for $V(t)$:

\begin{itemize}
\item $V(t)$ is positive definite
\item $V(t)$ is radially unbounded
\item $V(t)$ is equal to zero if and only if $x=\vec{0}$,
\end{itemize}

\noindent The time derivative of (\ref{chUV_AMTC.eq.lyap}) is 

\begin{equation} \label{chUV_AMTC.eq.lyap_dot}
\dot{V}(t)=\frac{1}{2}
x^T\left(\hat{\mathbb{A}}^{T}(\Delta\psi)\mathcal{M}_\epsilon 
        +\mathcal{M}_\epsilon\hat{\mathbb{A}}(\Delta\psi)\right) x
\end{equation}

\noindent From the facts
$\Delta\psi^T\left(\hat{\mathcal{A}}^{-T}+\hat{\mathcal{A}}^{-1}\right)\Delta
\psi=\Delta\psi^T\Delta\psi$ (shown in Section \ref{sec.show_A_to_I})
and $\Delta v^T \left(\hat{\mathcal{A}}^{-T} M +\hat{\mathcal{A}}^{-1}
  M\right)\Delta v \leq 2 \lambda_1 c \|\Delta v\|_2^2$ (where
$\lambda_1$ is the max eigenvalue of $M$ and $c\in\mathbb{R}_+$ is
such that $\|\hat{\mathcal{A}}^{-1}x\| \leq c\|x\|_2$ $\forall \psi$
for which $\|\psi\|<\pi$ as shown in Section \ref{sec.bound_SE3_A})
the expressions below follow

\begin{align}
\dot{V}(t) =&\frac{1}{2} x^T \left[\begin{array}{cc}
  -\epsilon k_p \left(\hat{\mathcal{A}}^{-T}+\hat{\mathcal{A}}^{-1}\right) &
 -\epsilon k_d \mathbb{I}_{6 \times 6}        \\
  -\epsilon k_d \mathbb{I}_{6 \times 6}              &  
\left(\hat{\mathcal{A}}^{-T} M +\hat{\mathcal{A}}^{-1}
M\right)- 2k_d\mathbb{I}_{6 \times 6}                \\
\end{array} \right] x
\nonumber \\
 =&\frac{1}{2} x^T \left[\begin{array}{cc}
  -\epsilon2k_p\mathbb{I}_{6 \times 6}  &
 -\epsilon k_d \mathbb{I}_{6 \times 6}        \\
  -\epsilon k_d \mathbb{I}_{6 \times 6}              &  
\left(\hat{\mathcal{A}}^{-T} M +\hat{\mathcal{A}}^{-1}
M\right)- 2k_d\mathbb{I}_{6 \times 6}                \\
\end{array} \right] x
\nonumber \\
\leq& -\epsilon  k_p \|\Delta\psi\|_2^2+\epsilon k_d\|\Delta
\psi\|_2\|\Delta v\|_2 +\frac{\epsilon}{2}\lambda_1 c \|\Delta v\|^2_2
-  k_d \|\Delta v\|^2_2
\nonumber \\
 \leq&\frac{1}{2}\left[\begin{array}{cc}
   \|\Delta \psi\| & \|\Delta v\|
   \end{array} \right] 
  \left[\begin{array}{cc}
  -\epsilon 2 k_p  & \epsilon k_d    \\
 \epsilon k_d      &   \epsilon \lambda_1 c- 2k_d\\
  \end{array} \right] 
\left[\begin{array}{c}
   \|\Delta \psi\| \\ \|\Delta v\|
   \end{array} \right]. 
\label{chUV_AMTC.eq.lyap_dot_bound}
\end{align}

\noindent Consider that as $\epsilon \to 0$ the 2-by-2 symmetric
matrix in (\ref{chUV_AMTC.eq.lyap_dot_bound}) has at least one
negative eigenvalue. Since the determinate of this matrix is equal to
the product of its eigenvalues, if the determinate is positive for
$\epsilon$ small enough, then both eigenvalues must be negative,
making the matrix negative definite.  Since the determinate equals
$\epsilon\left(-2\epsilon\lambda_1 c k_p +4k_pk_d- \epsilon k_d^2
\right)$, $\forall\epsilon$ such that

\begin{equation}
\epsilon \leq \frac{4 k_p k_d}{2 \lambda_1 k_p c +k_d^2}
\end{equation}

\noindent then the determinate is positive and, therefore, there
exists an $\epsilon$ such that $\dot{V}(t)$ is negative definite.
