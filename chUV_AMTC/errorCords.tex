
\subsection{State Error Coordinates}
\label{chUV_AMTC.sec.SE3_errCoord}

In this section we define the posistion error vector, $\Delta \psi$,
the velocity error vector, $\Delta v$ and compute a useful expression 
for $\dot{\Delta v}$. 


We define $\Delta H$ as 

\begin{equation}
\Delta H={^w_d}H^{-1}{^w_a}H.
\end{equation}

\noindent Note that $\Delta H$ is the transfrom from the actual to desired
vehicle frame since $\Delta H={^d_w}H{^w_a}H={^d_a}H$.  We use $\Delta
H$ to define the position error vector, $\Delta \psi$, as 

\begin{equation}
\Delta \psi=\log_{\SE3}\left(\Delta H \right).
\end{equation}

\noindent We define our velocity error vector, $\Delta v$, as 

\begin{equation}
\Delta v = {^a}v_a-{^a}v_d
\end{equation}
 
\noindent where ${^a}v_d=\Ad_{\Delta H^{-1}}{^d}v_d$ is defined using
the adjoint map $Ad:\SE3\to\mathbb{R}^{6 \times 6}$ which transforms
SE(3) velocity vectors between reference frames (CITE-REFERENCE OR
PREV CHAPTER).  Using the fact that $\forall v\in\mathbb{R}^6$ and 
$\forall H\in\SE3$, $H\widehat{v}H^{-1}=\widehat{\Ad_H v}$, consider 
the time derivative of $\Delta H$  

\begin{align}
\Dot{\Delta H}
  =& {^w_d}H^{-1}{^w_a}\dot{H} + {^w_d}\dot{H}^{-1}{^w_a}H  
\nonumber \\ 
  =&\Delta H \widehat{{^a}v_a} -\widehat{{^d}v_d}\Delta H   
\nonumber \\ 
  =& \Delta H\widehat{{^a}v_a} -\Delta H \Delta H^{-1}\widehat{{^d}v_d}\Delta H 
\nonumber \\ 
  =&\Delta H\left( \Delta H {^a}v_a -\Ad_{\Delta H^{-1}}{^d}v_d\right)^{\widehat{}} 
\nonumber \\ 
  =&\Delta H\widehat{\Delta v}. 
\end{align}

\noindent The previous equation implies that $\Delta \psi$ and $\Delta
v$ are consistent, thus

\begin{equation}\label{chUV_AMTC.eq.delta_psi_dot}
\dot{\Delta \psi} = \hat{\mathcal{A}}^{-1}(\Delta \psi) \Delta v
\end{equation}

\noindent where
$\hat{\mathcal{A}}^{-1}:\mathbb{R}^6\to\mathbb{R}^6\times\mathbb{R}^6$ is
the inverse of the SE(3) jacobian (CITE SOMEONE).  

Finally, lets develop an expression for $\dot{\Delta v}$ which will
turn out to be useful.  Note that $\widehat\Delta
v=\widehat{{^a}v_a}-\Delta H^{-1}\widehat{{^d}v_d}\Delta H$ and
consider the following expression for $\dot{\Delta v}$,

\begin{align}
\widehat{\dot{\Delta v}}
 =&\widehat{{^a}\dot{v}_a}
  -\Delta H^{-1}\widehat{{^d}\dot{v}_d}\Delta H
  -\dot{\Delta H}^{-1}\widehat{{^d}v_d}\Delta H
  -\Delta H^{-1}\widehat{{^d}v_d}\dot{\Delta H}
\nonumber \\
 =&\widehat{{^a}\dot{v}_a}
  -\Delta H^{-1}\widehat{{^d}\dot{v}_d}\Delta H
  +\widehat{\Delta v}\Delta H^{-1}\widehat{{^d}v_d}\Delta H
  -\Delta H^{-1}\widehat{{^d}v_d}\Delta H\widehat{\Delta v}
\nonumber \\
 =&\widehat{{^a}\dot{v}_a}
  -\left(\Ad_{\Delta H^{-1}}{^d}\dot{v}_d\right)^{\widehat{}}
  +\widehat{\Delta v}\left(\Ad_{\Delta H^{-1}}{^d}v_d\right)^{\widehat{}}
  -\left(\Ad_{\Delta H^{-1}}{^d}v_d\right)^{\widehat{}}\widehat{\Delta v}.
\label{chUV_AMTC.eq.Delta_v_dot_se3}
\end{align}


\noindent Note that the last two terms of the final equality of
(\ref{chUV_AMTC.eq.Delta_v_dot_se3}) are the Lie bracket $\Delta v$
and ${^a}v_d$, and that using matrix commutator,
$\ad:\mathbb{R}^6\to\mathbb{R}^{6\times 6}$, with the $\mathbb{R}^6$
representations of velocities is analytically equivilent to the Lie
bracket operation on their $\se3$ representations, i.e. $\forall
x,y\in\mathbb{R}^6$ we know $\widehat{\ad_x
  y}=\widehat{x}\widehat{y}-\widehat{y} \widehat{x}$. Therefore, based
on (\ref{chUV_AMTC.eq.Delta_v_dot_se3}), we have the following
expression

\begin{equation}\label{chUV_ATMC.eq.delta_v_dot}
\dot{\Delta v} = {^a}\dot{v}{_a}-{^a}\dot{v}{_d}+\ad_{\Delta v} {^a}v_d.
\end{equation}


\subsection{Parameter Error Coordinates}
\label{chUV_AMTC.sec.param_errCoord}

talk about $\Delta \theta$...