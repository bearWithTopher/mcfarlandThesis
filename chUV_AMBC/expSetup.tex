
\subsection{Experimental Setup}
\label{chUV_AMBC.sec.expSetup}


We employed the Johns Hopkins University Hydrodynamic Test Facility
(Appendix \ref{appenJHUHTF}) to evaluate \ac{UV} \ac{AMBC}.
%
To compare the trajectory-tracking performance of controllers we
consider the \ac{MNE} for vectors such as the position error ($\Delta
\psi$) and velocity error ($\Delta v$).
%
We also report the \ac{MAE} between the measured and reference
values for the 12 position and velocity signals.
%
A smaller \ac{MNE} or \ac{MAE} value means the controller is doing a
better job tracking the reference trajectory.
%
As with the adaptively identified \ac{UV} models in Sections
\ref{chUV_AID.sec.UVSO3exp} and \ref{chUV_AID.sec.UVSE3exp},
identified \ac{UV} models are evaluated by error between simulated
model performance and the experimentally observed performance.
%
\ac{MAE} values between the simulated plant states and
experimentally measured states are reported.
%
Appendix \ref{appenJHUHTF} provides further details about
our hydrodynamic test facility and algorithm evaluation methods.


The fully-coupled model of \ac{UV} dynamics used in
(\ref{chUV_AMBC.eq.UVSE3plant}) requires 241 independent parameter
values.
%
To simplify and clarify the experimental analysis of \ac{UV} \ac{AMBC}
in the presence of unmodeled thruster dynamics, we have implemented a
controller which employs an uncoupled model using 16 scalar terms: 6
hydrodynamic mass, 6 quadratic drag terms (one for each \ac{DOF} which
we will label as $m_i$ and $d_i$ respectively), and the 4
gravitational parameters $g$ and $b$.
%
For a diagonal parameter adaptation gain matrix $K_\theta$, we can
label the individual mass parameter gains as $\gamma_{m_i}$, the
individual drag parameter gains as $\gamma_{d_i}$, the individual
buoyancy parameter gains as $\gamma_{b_i}$ and the gravitational
parameter gain as $\gamma_{g}$.
%
Both the \ac{AMBC} control process and parameter update process
where implemented as a discrete time approximation of the
continuous time algorithm.
%
Every 100ms the commanded torque and commanded force were recalculated
using (\ref{chUV_AMBC.eq.AMBC_controlLaw}) as well as the most recent
state measurements, reference state values, and parameter estimates.
%
These inputs signals are therefore piecewise constant.
%
Euler integration of (\ref{chUV_AMBC.eq.AMBC_paramUpdateLaw}) for
100ms time steps provided the time series of parameter estimates.
%
The 100ms \ac{JHUROV} control system cycle period is one to two
orders-of-magnitude smaller than the \ac{JHUROV} state variation rate
of 1 second or greater observed during dynamic operation.
%
In practice the discrete time approximations were seen to provide
similar performance to the continuous time algorithm implemented in
simulation.


Sinusoidal reference trajectories are used in this
study. 
%
Table \ref{chUV_AMBC.tb.expStat} lists the frequencies and amplitudes
for the 6-\ac{DOF} reference trajectories used.  


\begin{table}[htbp]
\ssp
\caption{Reference Trajectory Information}
\begin{center}
\begin{tabular}{cccc}
 & & RefTraj1& RefTraj2 \\
\hline
\multicolumn{2}{c}{Reference Trajectory}  &Trajectory Control      &  Parameter \\
\multicolumn{2}{c}{Purpose}              & Evaluation &  Cross-Validation \\
\hline
\ac{DOF}      & {\it Excitation} & \multicolumn{2}{c}{\it Trajectory-Tracking} \\
world x  &  Cos Freq  & 0.242 rad/sec & 0.185 rad/sec \\ 
         &  Cos Amp  &    0.50 m     &     0.50 m    \\ 
\hline
\ac{DOF}      & {\it Excitation} & \multicolumn{2}{c}{\it Trajectory-Tracking} \\
world y  &  Cos Freq  & 0.210 rad/sec & 0.286 rad/sec \\ 
         &  Cos Amp  &    0.50 m     &     0.50 m    \\ 
\hline
\ac{DOF}      & {\it Excitation} & \multicolumn{2}{c}{\it Trajectory-Tracking} \\
world z  &  Cos Freq  & 0.185 rad/sec & 0.242 rad/sec \\ 
         &  Cos Amp  &    0.30 m     &     0.30 m    \\ 
\hline
\ac{DOF}      & {\it Excitation} & {\it Trajectory-Tracking} & {\it Torque Input} \\
roll     &  Cos Frequency  & 0.5 rad/sec & 0.55 rad/sec \\ 
         &  Cos Amplitude  &    6.9$^\circ$    &     35 N m     \\ 
\hline
\ac{DOF}      & {\it Excitation} & {\it Trajectory-Tracking} & {\it Torque Input} \\
pitch    &  Cos Freq  & 0.6 rad/sec & 0.65 rad/sec \\ 
         &  Cos Amp  &    8.6$^\circ$    &     30 N m     \\ 
\hline
\ac{DOF}      & {\it Excitation} & \multicolumn{2}{c}{\it Trajectory-Tracking} \\
heading  &  Cos Freq  & 0.210 rad/sec & 0.0824 rad/sec \\ 
         &  Cos Amp  &    135$^\circ$  &   135$^\circ$   \\  
\hline \end{tabular}
\end{center}
\label{chUV_AMBC.tb.expStat}
%\vspace*{-5mm}
\end{table}