\subsection{\acs{UV} \acs{MBC}}
%\label{sec.MBCder}
\label{chUV_AMBC.sec.MBC}

This Section reports a linearizing tracking controller for mechanical
plants of the form (\ref{chUV_AMBC.eq.UVSE3plant}). Note that this
controller requires exact {\it a priori} parameter knowledge. 
%
A local stability analysis is also included. 
%
Section \ref{chUV_AMBC.sec.MBC_goal} explicitly states the goal for \ac{UV}
\ac{MBC}.
%
The proof that Theorem \ref{chUV_AMBC.theo.UV_MBC} satisfies these
conditions is provided in two parts.
%
First, in Section \ref{chUV_AMBC.sec.UV_MBC_controlledPlant}, the
dynamics of the controlled plant is developed.
%
Then, in Section \ref{chUV_AMBC.sec.proof_MBC}, we prove the
result.
%
Section \ref{chUV_AMBC.sec.AMBC} reports an adaptive extension to
the \ac{MBC} algorithm reported in this Section.



\subsubsection{\acs{UV} \acs{MBC} Goal}
\label{chUV_AMBC.sec.MBC_goal}

Given a desired reference trajectory
$\{{^w_d}H,\;{^d}v_d,\;{^d}\dot{v}_d \}$ for a plant of the form
(\ref{chUV_AMBC.eq.UVSE3plant}), where the signals
$\{{^w_a}H,\;{^a}v_a,\; u\}$ are accessible and the parameters
$\{M,\;D,\; g,\; b\}$ are known, our goal is to develop a control law
\funDefn{u}{\SE3\times\realSpace{6}\times\SE3\times\realSpace{6}\times\realSpace{6}}
{\realSpace{6}} which guarantees that all signals remain bounded and
provides asymptotically exact reference trajectory tracking, i.e.
$\lim_{t\to \infty}\Delta \psi(t)=\vec{0}$ and $\lim_{t\to
  \infty}\Delta v(t)=\vec{0}$.

\subsubsection{\acs{UV} \acs{MBC} Theorem}


\begin{UV_MBC}
\label{chUV_AMBC.theo.UV_MBC}

For plants of the form (\ref{chUV_AMBC.eq.UVSE3plant}), %
when the plant parameters % 
 $\{M,D,g,b\}$
 are known, the control law
%
\begin{equation}
u\left( {^w_a}H, {^a}v_a, {^w_d}H, {^d}v_d, {^d}\dot{v}_d\right)
=-(k_p\hat{\mathcal{A}}^{-T}(\Delta \psi)+k_d\Delta v)+ 
\MBCreg({^a}\dot{v}_d,\Delta v, {^a}v_d , {^w_a}H)\theta_{UV},
\label{chUV_AMBC.eq.MBC_controlLaw}
\end{equation}
%
\noindent where the regressor matrix
$\MBCreg:\mathbb{R}^6\times\mathbb{R}^6\times\mathbb{R}^6\times\SE3\to
\mathbb{R}^{6 \times 241}$ is defined such that
%
\begin{equation}
\MBCreg({^a}\dot{v}_d,\Delta v, {^a}v_d , {^w_a}H)\theta_{UV} =
    M{^a}\dot{v}_d - M\ad_{\Delta v}{^a}v_d -\ad_{{^a}v_a}^T M{^a}v_a
    -\mathcal{D}({^a}v_a)-\mathcal{G}({^w_a}H),
\label{chUV_AMBC.eq.MBC_regressor}
\end{equation}
%
provides asymptotically stable trajectory tracking in the
sense of Lyapunov, i.e.  $\lim_{t\to \infty}\Delta \psi(t)=\vec{0}$
and $\lim_{t\to \infty}\Delta v(t)=\vec{0}$, if the following
conditions are met:

\begin{itemize}
\item the signals $\{{^w_d}H,{^d}v_d,{^d}\dot{v}_d
  \}\in\{\SE3,\realSpace{6},\realSpace{6}\}$ are continuous and bounded 
\item $k_d,\; k_p\in\mathbb{R}_+$
\item $\|x(t_0)\|<\sqrt{
                  \frac{ {_\epsilon}\lambda_{12} }
                       { {_\epsilon}\lambda_1   }
                       } \pi\quad$
%
for the state error vector $x$ defined in (\ref{chVU_AMBC.eq.error_x}) 
\end{itemize}

\noindent where, as per the eigenvalue ordering conventions from
Section \ref{chModels.sec.normConventions}, ${_\epsilon}\lambda_{i}$
are the eigenvalues of
%
$\mathcal{M}_\epsilon$ from (\ref{chUV_AMBC.eq.lyapM}).
\end{UV_MBC}

%----------------------------------------------------------------
%----------------------------------------------------------------
%originally from 2013_McFarland_Thesis/chUV_AMBC/errorCords.tex
%\subsubsection{Velocity Error Time Derivative}



%----------------------------------------------------------------
%----------------------------------------------------------------
%originally from 2013_McFarland_Thesis/chUV_AMBC/MBC.tex
\subsubsection{\acs{UV} \acs{MBC} Controlled Plant}
\label{chUV_AMBC.sec.UV_MBC_controlledPlant}

The controlled plant is of the form
%
\begin{align}
  M{^a}\dot{v}_a=
     &\ad_{{^a}v_a}^T M{^a}v_a+\mathcal{D}({^a}v_a)+\mathcal{G}({^w_a}H)
%\nonumber \\ 
      -(k_p\hat{\mathcal{A}}^{-T}(\Delta \psi)+k_d\Delta v)
\nonumber \\ 
     &+\MBCreg({^a}\dot{v}_d,\Delta v, {^a}v_d , {^w_a}H)\theta_{UV}
\nonumber \\
     =&M{^a}\dot{v}_d - M\ad_{\Delta v}{^a}v_d
      -(k_p\hat{\mathcal{A}}^{-T}(\Delta \psi)+k_d\Delta v)
%\nonumber 
\end{align}
%
From (\ref{chUV_AMBC.eq.delta_v_dot}) we get
%
\begin{equation}\label{chUV_AMBC.eq.Mdelta_v_dot}
M \Delta\dot{ v} =       -(k_p\hat{\mathcal{A}}^{-T}(\Delta \psi)+k_d\Delta v).
\end{equation}
%
Using the system error vector $x\in\mathbb{R}^{12}$ defined in
(\ref{chVU_AMBC.eq.error_x}) as
%
\begin{equation}
x=\left[ \begin{array}{c}
     \Delta \psi           \\
     \Delta v              \\
\end{array} \right]
\end{equation}
%
and the matrix valued function
$\hat{\mathbb{A}}:\mathbb{R}^6\to\mathbb{R}^{12}\times\mathbb{R}^{12}$ defined as
%
\begin{equation}\label{chUV_AMBC.eq.error_A}
\hat{\mathbb{A}}(\Delta \psi)=
\left[ \begin{array}{cc}
     0_{6\times 6}   & \hat{\mathcal{A}}^{-1}(\Delta \psi)             \\
     -k_p M^{-1}\hat{\mathcal{A}}^{-T}(\Delta \psi)   &  -k_d M^{-1}   \\
\end{array} \right]
\end{equation}
%
then from (\ref{chUV_AMBC.eq.delta_psi_dot}),
(\ref{chUV_AMBC.eq.Mdelta_v_dot}), (\ref{chVU_AMBC.eq.error_x}), and
(\ref{chUV_AMBC.eq.error_A}) we see that the equation for the error
dynamics of the system  is
%
\begin{equation} \label{chUV_AMBC.eq.err_sys}
\dot{x}=\hat{\mathbb{A}}(\Delta \psi) x.
\end{equation}

%----------------------------------------------------------------
%----------------------------------------------------------------
%originally from 2013_McFarland_Thesis/chUV_AMBC/MBC.tex
\subsubsection{\acs{UV} \acs{MBC} Stability Proof}\label{chUV_AMBC.sec.proof_MBC}

Consider the following Lyapunov function candidate
%
\begin{equation} \label{chUV_AMBC.eq.lyap}
V_1(t)=\frac{1}{2} x^T\mathcal{M}_\epsilon x
\end{equation}
%
where 
%
\begin{equation}\label{chUV_AMBC.eq.lyapM}
\mathcal{M}_\epsilon=\left[ 
\begin{array}{cc}
  k_p\mathbb{I}_{6\times 6}  & \epsilon M         \\
  \epsilon M               &    M               \\
\end{array} \right].
\end{equation}
%
Consider that $\forall x$
%
\begin{equation} \label{chUV_AMBC.eq.lyap_bound}
V_1(t)\geq\frac{1}{2}\left[\begin{array}{cc}
   \|\Delta \psi\| & \|\Delta v\|
   \end{array} \right] 
  \left[\begin{array}{cc}
       k_p             & -\epsilon \lambda_1  \\
  -\epsilon \lambda_1  &  \lambda_6           \\
  \end{array} \right] 
\left[\begin{array}{c}
   \|\Delta \psi\| \\ \|\Delta v\|
   \end{array} \right]
\end{equation}
%
%TM_ADD_BELOW
%\noindent Consider that as $\epsilon \to 0$ the 2-by-2 symmetric
%matrix in (\ref{chUV_AMBC.eq.lyap_bound}) has two positive eigenvalues $\forall
%k_p\in\mathbb{R}_+$.  Since the determinate of this matrix is equal to
%the product of its eigenvalues an eigenvalue can cross zero only if
%the determinant crosses zero.  Since the determinate equals
%$k_p\lambda_6-\epsilon^2\lambda_1^2$ we know for
%$\forall\epsilon \in \mathbb{R}_+$ such that 
%
where 
%
 $  \left[\begin{array}{cc}
       k_p             & -\epsilon \lambda_1  \\
  -\epsilon \lambda_1  &  \lambda_6           \\
  \end{array} \right]$  
%
is \ac{SPD} if 
%
\begin{equation}
\epsilon \leq \sqrt{\frac{k_p \lambda_6}{\lambda_1^{2}}}.
\end{equation}
%
Thus
%
\begin{itemize}
\item $V_1(t)$ is positive definite and
%\item $V_1(t)$ is radially unbounded, and
\item $V_1(t)$ is equal to zero if and only if $x=\vec{0}$.
\end{itemize}


The time derivative of (\ref{chUV_AMBC.eq.lyap}) is 
%
\begin{equation} \label{chUV_AMBC.eq.lyap_dot}
\dot{V}_1(t)=\frac{1}{2}
x^T\left(\hat{\mathbb{A}}^{T}(\Delta\psi)\mathcal{M}_\epsilon 
        +\mathcal{M}_\epsilon\hat{\mathbb{A}}(\Delta\psi)\right) x.
\end{equation}
% 
Note that $\exists c\in\mathbb{R}_+$ for which
$\|\hat{\mathcal{A}}^{-1}(\psi) x\| \leq c\|x\|\quad\forall\psi\in\rSp{6}$ such that $\|\psi\|<\pi$ (See Appendix
\ref{appenJacSE3.sec.normBound}).  Therefore $\forall \Delta \psi$
such that $\|\Delta\psi\|<\pi$ we have
%$c$ and $\lambda_1$ (the max eigenvalue of $M$)
%
%
\begin{equation}
\Delta
v^T \left(\hat{\mathcal{A}}^{-T}(\Delta\psi) M +M \hat{\mathcal{A}}^{-1}(\Delta \psi)
  \right)\Delta v \leq 2 \lambda_1 c \|\Delta v\|^2.
\end{equation}
%
%
Using this bound and the equality
$\Delta\psi^T\left(\hat{\mathcal{A}}^{-T}+\hat{\mathcal{A}}^{-1}\right)\Delta
\psi=\Delta\psi^T\Delta\psi$ (shown in Appendix
\ref{appenJacSE3.sec.PsiHatAinvPsiEquality}), we have
%
%\begin{widetext}
\begin{align}
\dot{V}_1(t) =&\frac{1}{2} x^T \left[\begin{array}{cc}
  -\epsilon k_p \left(\hat{\mathcal{A}}^{-T}+\hat{\mathcal{A}}^{-1}\right) &
 -\epsilon k_d \mathbb{I}_{6 \times 6}        \\
  -\epsilon k_d \mathbb{I}_{6 \times 6}              &  
\left(\hat{\mathcal{A}}^{-T} M +M \hat{\mathcal{A}}^{-1}
\right)- 2k_d\mathbb{I}_{6 \times 6}                \\
\end{array} \right] x
\nonumber \\
 =&\frac{1}{2} x^T \left[\begin{array}{cc}
  -\epsilon2k_p\mathbb{I}_{6 \times 6}  &
 -\epsilon k_d \mathbb{I}_{6 \times 6}        \\
  -\epsilon k_d \mathbb{I}_{6 \times 6}              &  
\left(\hat{\mathcal{A}}^{-T} M +M \hat{\mathcal{A}}^{-1}
\right)- 2k_d\mathbb{I}_{6 \times 6}                \\
\end{array} \right] x
\nonumber \\
\leq& -\epsilon  k_p \|\Delta\psi\|^2+\epsilon k_d\|\Delta
\psi\|\|\Delta v\| +\frac{\epsilon}{2}\lambda_1 c \|\Delta v\|^2
-  k_d \|\Delta v\|^2
\nonumber \\
 \leq&\frac{1}{2}\left[\begin{array}{cc}
   \|\Delta \psi\| & \|\Delta v\|
   \end{array} \right] 
  \left[\begin{array}{cc}
  -\epsilon 2 k_p  & \epsilon k_d    \\
 \epsilon k_d      &   \epsilon \lambda_1 c- 2k_d\\
  \end{array} \right] 
\left[\begin{array}{c}
   \|\Delta \psi\| \\ \|\Delta v\|
   \end{array} \right]. 
\label{chUV_AMBC.eq.lyap_dot_bound}
\end{align}
%\end{widetext}
%
$\forall \Delta \psi$ such that $\|\Delta\psi\|<\pi$.
%\noindent Consider that as $\epsilon \to 0$ the 2-by-2 symmetric
%matrix in (\ref{chUV_AMBC.eq.lyap_dot_bound}) has at least one
%negative eigenvalue. Since the determinate of this matrix is equal to
%the product of its eigenvalues, if the determinate is positive for
%$\epsilon$ small enough, then both eigenvalues must be negative,
%making the matrix negative definite.  Since the determinate equals
%$\epsilon\left(-2\epsilon\lambda_1 c k_p +4k_pk_d- \epsilon k_d^2
%\right)$, the  $\forall\epsilon$ such that
Since
%
$ \left[\begin{array}{cc}
  -\epsilon 2 k_p  & \epsilon k_d    \\
 \epsilon k_d      &   \epsilon \lambda_1 c- 2k_d\\
  \end{array} \right]$ 
%
is negative definite symmetric if
%
\begin{equation}
\epsilon \leq \frac{4 k_p k_d}{2 \lambda_1 k_p c +k_d^2},
\end{equation}
%
and $\dot{V}_1(t)$ is negative definite if this condition on
$\epsilon$.
%
We are free to choose $\epsilon$ such that
$\epsilon\leq\min\left(\frac{4 k_p k_d}{2 \lambda_1 k_p c
    +k_d^2},\sqrt{\frac{k_p \lambda_6}{\lambda_1^{2}}} \right)$, thus
we conclude that the system is locally asymptotically stable, i.e.
$\lim_{t\to \infty}\Delta \psi(t)=\vec{0}$ and $\lim_{t\to
  \infty}\Delta v(t)=\vec{0}$.



This result is local because we require $\|\Delta\psi(t)\|<\pi\quad \forall t \geq t_0$. To justify this assumption note that
$\forall t\in\rSp{}$ we know $\|\Delta\psi(t)\|\leq\|x(t)\|$ and
consider the condition $\|x(t_0)\|<\sqrt{ \frac{
    {_\epsilon}\lambda_{12} } { {_\epsilon}\lambda_1 } } \pi$.
%
Since ${_\epsilon}\lambda_{12}\leq{_\epsilon}\lambda_1$, this
condition implies $\|x(t_0)\|< \pi$.  By the Lyapunov stability proof
above, $V_1(t)\leq V_1(t_0)$ for all $t\geq t_0$, thus
%
\begin{align}
{_\epsilon}\lambda_{12}\|x(t)\|^2
    \leq&x(t)^T \mathcal{M}_\epsilon x(t)
\nonumber \\
    \leq&V_1(t)
\nonumber \\
    \leq&V_1(t_0)
\nonumber \\
    \leq&x(t_0)^T \mathcal{M}_\epsilon x(t_0)
\nonumber \\  
    \leq&{_\epsilon}\lambda_1\|x(t_0)\|^2
\nonumber \\
     <&{_\epsilon}\lambda_1\left(\sqrt{ \frac{ {_\epsilon}\lambda_{12} } {
    {_\epsilon}\lambda_1 } } \pi  \right)^2.    
\end{align}
%
\noindent This inequality implies $\|\Delta \psi(t)\|\leq\|x(t)\|<\pi$
for all $t \geq t_0$.  The assumption is justified and local asymptotic
stability is proven.


