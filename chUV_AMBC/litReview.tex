\section{Literature Review}
 
Adaptive controllers for linear plants are
well understood 
\cite{ksn&anu.book}.
%\cite{ksn&anu.book,sastry&bodson.book,astrom.book}.
% 
Adaptive reference trajectory-tracking is well
understood for several types of second-order holonomic nonlinear
plants whose parameters enter linearly into the plant equations of
motion, e.g. robot manipulator arms
\cite{Craig&hsu&sastry.ijrr87,slotine&li.ijrr87,horowitz&sadegh.ijrr90},
 spacecraft \cite{kod.cdc85a,slotine1990},
and  rigid-body rotational plants \cite{Chaturvedi2006}.
% rigid-body rotational plants \cite{Chaturvedi2006}, and general
%mechanical systems \cite{Lain1997}.
%
Comparative experimental evaluations of \ac{AMBC} for robot manipulator
arms have been reported, e.g. \cite{slotine.performance,whitcomb&kod.tra93}.



The structure of the \ac{MBC} algorithm reported herein was
inspired by the proportional derivative tracking control algorithm for
rigid-body motion in free space reported by Bullo and
Murray\cite{bullo1995_SE3_PD}.
%
Our controller can be seen as a specialization of this result for
\ac{UV} control; we use their error coordinate structure with a
fully-coupled lumped-parameter plant model of \ac{UV} dynamics,
(\ref{chModels.eq.UVSE3plant}).
%
Fully-coupled lumped-parameter plant models for \acp{UV} use a finite
set of plant parameters which have been shown empirically to be a good
approximation for the complex dynamics of a rigid-body and
associated fluid-vehicle interaction.
%
Lumped-parameter models of \ac{UV} dynamics are used in previously
reported \ac{MBC} algorithms.
%
In \cite{fossen}, Fossen summarizes lumped-parameter modeling of
\ac{UV} dynamics and \ac{UV} \ac{MBC} using a lumped-parameter
modeling and traditional error coordinates.
%
\cite{smallwood2004JOE} reports and experimentally evaluates 
single \ac{DOF} \ac{UV} \ac{MBC} algorithms for control of the 
x, y, depth, and heading \ac{DOF}.
%
\cite{martinControl_ICRA13} reports and experimentally evaluates a
\ac{UV} \ac{MBC} algorithm assuming a fully-coupled lumped-parameter
model; this comparative experimental evaluation of \ac{MBC} and
\ac{PDC} included reference trajectories requiring simultaneous motion
in all \ac{DOF} and was conducted using the Johns Hopkins University
Hydrodynamic Test Facility (Appendix \ref{appenJHUHTF}).



Previous studies have utilized a lumped-parameter \ac{UV} models in
the development of tracking control algorithms which are robust to
model parameter uncertainty.
%
In \cite{yoerger&slotine.JOE85} a sliding mode controller and
numerical simulations of performance in X, Y, and heading is reported.
%
In \cite{Yuh1990} a discrete time parameter adaptation algorithm is
reported with a numerical simulation study.  
%
In \cite{Fossen1991} the authors report a hybrid (adaptive and
sliding) nonlinear \ac{UV} controller which explicitly handles
multiplicative uncertainty in the input mapping.


Experimental evaluations of \ac{AMBC} algorithms have also been
reported.
%
Yoerger et al. reported the first experimental evaluation of nonlinear
adaptive sliding-mode control on an \ac{UV} \cite{yoerger.icra91}.
%
In \cite{yuhICRA1999} an experimental evaluation of \ac{UV} \ac{AMBC}
performance in the presence of noisy position readings is reported.
%
In \cite{antonelli&sarkar.cst2001} an experimental evaluation of an
\ac{AMBC} is reported for simultaneous motion in the translational
\acp{DOF}.
%
Comparative experimental evaluations of linear controllers,
model-based controllers, and their adaptive extensions for \ac{UV} single
\ac{DOF} motion have been reported \cite{smallwood2004JOE, maalouf2013pd}.
%are the subject of a compartive experimental evaluation using the JHU
%ROV.
%
In \cite{zhao2005experimental} a comparative experimental evaluation
in the presence of a common external disturbance for proportional
integral derivative control, disturbance observer control and the
adaptive extensions of both is reported.
%
%In \cite{} an \ac{AMBC} algorithm for \ac{UV} depth control is
%presented and implemented on a test vehicle.
%
In each of these previously reported experimental evaluations at most
two \ac{DOF} had a non-zero reference velocity at a given moment, and
in all cases at most set-point regulation was reported in the pitch and
roll \ac{DOF}.
