\subsection{Two-Step \ac{AMBC}}\label{chUV_AMBC.sec.twoStepMethod}

In the sequel we will find it convenient to adaptively identify
subsets of plant parameters. Note that
(\ref{chUV_AMBC.eq.MBC_regressor}) is linear in the parameters, thus
there exists a set of functions \funDefn{\MBCregi}
{\realSpace{6}\times\realSpace{6}\times\realSpace{6}\times\SE3}
{\realSpace{6}}
%
such that 
%
\begin{equation}
%\hspace*{-3mm}
\sum_{i=1}^{241}\MBCregi({^a}\dot{v}_d,\Delta v, {^a}v_d , {^w_a}H)\theta_{UV_i}
=\MBCreg({^a}\dot{v}_d,\Delta v, {^a}v_d , {^w_a}H)\theta_{UV}.
\end{equation}
%
\noindent If a subset of parameter values in $\theta_{UV}$ is known
{\it a priori} then, based on Theorems \ref{chUV_AMBC.theo.UV_MBC} and
\ref{chUV_AMBC.theo.UV_AMBC}, an \ac{AMBC} can be developed to
estimate the remaining unknown parameters.  Any \acp{AMBC} of this
form achieve asymptotically exact trajectory-tracking canceling the
contribution to vehicle dynamics due to known parameters (as shown in
Section \ref{chUV_AMBC.theo.UV_MBC}) and use parameter update laws to
adaptively estimate of the unknown parameters (as shown in Section
\ref{chUV_AMBC.theo.UV_AMBC}). If $K_\theta$ is assumed to be
diagonal, each \ac{AMBC} of this form will be equivalent to using the
\ac{AMBC} from Theorem \ref{chUV_AMBC.theo.UV_AMBC}, with each known
parameter initialized to its known value and its associated parameter
gain within $K_\theta$ set to zero.  A two-step \ac{AMBC} algorithm presented
herein independently estimates two disparate sets of parameters in two
successive experimental trials.


The structure of (\ref{chUV_AMBC.eq.UVSE3plant}) allows the
identification of dynamic and gravitational parameters to be conducted
in a two-step process.  The first step is identifying $g$ and $b$
estimates during quasi-static motion in pitch and roll.
%
For quasi-static motion (i.e. nearly zero velocity and acceleration)
in these \ac{DOF}, the following model is sufficient to describe
vehicle dynamics
%
\begin{equation} \label{chUV_AMBC.eq.plantQS}
  0=\mathcal{G}({^w_a}H)+u 
\end{equation}
%
where all inertial and drag terms are assumed to be negligible because
they are either quadratic in velocity or linear in acceleration.  
%
The validity of this model implies using the adaptive tracking
controller from Theorem \ref{chUV_AMBC.theo.UV_AMBC} to track a slow
reference trajectory in pitch and roll will identify estimates of $g$
and $b$ without requiring the simultaneous estimation of mass and drag
parameters.

The second step \ac{AMBC} algorithm estimates the inertial and drag
parameters (i.e. $M$ and $D$) while using the gravitational parameter
values identified in step one.  Using the controller from Theorem
\ref{chUV_AMBC.theo.UV_AMBC} with constant $\hat{b}$ and $\hat{g}$ set
to their previously identified values, this controller can be used to
track any reference trajectory while identifying estimates of the
vehicle's mass and drag.  This partitioning of the parameter
identification process will be important in the sequel when we
consider the effects of unmodeled thruster dynamics on \ac{AMBC}.

