%\section{INTRODUCTION} \label{sec.intro} 


This Chapter addresses the problem of adaptive model-based trajectory
tracking control of \acp{UV} for dynamic 6-\ac{DOF} motion.
%
The approach employed herein, \ac{AMBC}, estimates plant parameters
during the trajectory-tracking control process.
%
The adaptive controller estimates parameters for a rigid-body plant
such as vehicle mass and added hydrodynamic mass parameters; quadratic
drag parameters; and gravitational force and buoyancy parameters that
arise in the dynamic models of rigid-body \acp{UV}.
%
We report a non-adaptive model-based control algorithm for
trajectory-tracking control of fully-actuated rigid-body underwater
vehicles, its adaptive extension, and mathematical analysis of the
stability of the resulting closed-loop systems.
%
We report a comparative experimental evaluation of \ac{AMBC} and
\ac{PDC} in full scale vehicle trials utilizing the Johns Hopkins
University Hydrodynamic Test Facility and \ac{JHUROV} (Appendix
\ref{appenJHUHTF.sec.hydrolab}).
%
The experimental evaluation shows that \ac{AMBC} provides better
position tracking performance (30\%) and marginally worse velocity
tracking performance (8\%) over \ac{PDC}.
%
To the best of the authors' knowledge, this is the first experimental
comparison of \ac{AMBC} and \ac{PDC} \acp{UV} during simultaneous
motion in all six \ac{DOF}.


This Chapter is in two parts. 
%
Section \ref{chUV_AMBC.sec.theory} reports a non-adaptive \ac{MBC} algorithm and
an \ac{AMBC} algorithm for \acp{UV}.
%
Section \ref{chUV_AMBC.sec.unmodeledActDyn} reports results indicating 
that unmodeled thruster dynamics can destabilize parameter
adaptation and a two-step \ac{AMBC} algorithm which is robust to 
unmodeled thruster dynamics. 


The results from Sections \ref{chUV_AMBC.sec.twoStepMethod} and
\ref{chUV_AMBC.sec.unmodeledActDyn} are reported in a paper submitted
to the 2014 International Conference on Robotics and Automation
\cite{mcfarland.icra2014}.

%----------------------------------------------------------------
%----------------------------------------------------------------

% However 
% %
% %DON'T USE SUCH A COMPLEX SENTENCE STRUCTURE
% Although this paper reports a novel MBAC for \ac{UV}, the
% main objective of this text is presenting practial {\color{red}
%   difficulties} to \ac{UV} \ac{AMBC} and solutions to those {\color{red}
%   difficulties}.
% %
% As such, ...


% %----- somewhere in here---------------
% In this paper we propose using Adaptive Tracking Control algorithm

% from \ref{theo.UV_AMBC} assuming the gravitational parameters ($g$
% and $b$) are known.  In Section \ref{sec.fullAnalysisFailure} we show
% \ac{UV} Full-Parameter \ac{AMBC} fails due to thruster stiction during pitch
% and/or roll actuation.  In Section \ref{sec.twoStepExp} we demonstrate
% experimentially that \ac{UV} Dynamic-Only-Parameter \ac{AMBC} provides
% asmtotically improving trajectory-tracking, a good model of vehicle
% performance.  It is therefore robust to the thruster stiction failure
% mode seen in the Full-Parameter 


% Section ____ presents two experiments where \ac{AMBC} is used to follow a pitch only referece trajectory.  
% %
% The range of motion is changed such that one pitch only reference
% trajectroy requires thrust reversals and the other pitch only
% reference trajectory does not.
% %
% A comparative analysis of the parameter adaption process of these
% single \ac{DOF} experiments shows how a small amount of unmodeled dynamics
% near thrust reversals can cause unstable parameter adaptation.
% %
% Section _____ shows how a straightforward implmentation of the \ac{AMBC}
% presented exhibits unstable parameter adaptation in the presence of
% unmodeled thruster dynamics during thrust reversals.
% %
% Section _____ shows the two-step alogithm provides an \ac{AMBC} which,
% while still in the presence of unmodeled dynamics near thrust
% reversals, still provides improved trajectory-tracking and parameters
% estimats which are physically realistic.
