%\subsection{Background and Paper Contribution}

% In addition to quantifying expected performance gains, it is also
% important that experimental evaluations highlight any unfounded
% assumptions made during the controller's formulation.
% %
% As is shown in Section \ref{sec.fullAnalysisFailure}, the assumption
% that pitch and roll torque applied was close enough to torque
% commanded was not a justified assumption.
% %
% Like the \ac{JHUROV}, many vehicles currently deployed in the open ocean
% are equipped with actuators which can not guarantee the exact force
% commanded will be applied in all operating conditions.
% %
% \ac{UV} \acp{AMBC} have yet to achieve widespread use due to their perceived
% fragility in real world operating conditions.
% %
% We anticipate that this thrust reversal modeling error is one of
% several \ac{UV} \ac{AMBC} failure modes which contribute to this perception.
% %
% However, documenting one such failure mode in laboratory conditions,
% and presenting a modification to \ac{AMBC} which is robust to the
% underlying cause of failure, is an important step to developing \ac{AMBC}
% algorithms which are ready for the open ocean.



%\ac{UV} \acp{AMBC} have yet to achieve widespread use due to their perceived
%fragility in real world operating conditions.
%%
%The \ac{JHUROV}, like many vehicles currently deployed in the open ocean, 
%is equipped with actuators which can not guarantee the exact force
%commanded will be applied in all operating conditions.
%%
%In Section \ref{sec.fullAnalysisFailure} we investigate how modeling
%inaccuracies near thrust reversals can cause unstable parameter
%adaptation.
%%
%We anticipate that this thrust reversal modeling error is one of
%several \ac{UV} \ac{AMBC} failure modes which contribute to the perception of 
%\ac{AMBC} fragility.  
%%
%\hide{We feel documenting this failure mode in laboratory conditions,
%and presenting a modification to \ac{AMBC} which is robust to the
%underlying cause of failure, is an important step to developing \ac{AMBC}
%algorithms which are ready for the open ocean.}


\subsection{Unmodeled Thruster Dynamics within \\
                the \acs{UV} Control Process}

The \ac{JHUROV} control system uses the common assumption of {\it
steady-state} thruster operation when calculating the actuator
commands (see Appendix \ref{appenJHUHTF.sec.hydrolab}).
%
In {\it steady-state} operation at zero advance velocity the axial
thrust of a bladed-propeller marine thruster is linearly proportional
to the applied shaft torque, and is also linearly proportional to the
signed-square of the shaft angular-velocity
\cite{pona.book}.  
%
The parameters of these steady-state thruster models
cannot be determined analytically, but are easily estimated with
simple steady-state experiments.
%
Research has shown that the {\it transient} performance of marine
thrusters can be accurately approximated by a finite-dimensional
second-order plant model of propeller-fluid interaction.
%
The plant parameters of these dynamic thruster models cannot be
determined analytically, and are difficult to estimate experimentally
because such identification requires highly instrumented measurements
of the thruster thrust, prop angular velocity, and fluid flow velocity
in unsteady operation
\cite{yoerger.oe90,healey.joe95,bachmayer&whitcomb&grosenbaugh.joe2000,kim2006accurate}.


Because unsteady thruster model parameters are difficult to obtain
experimentally, in the design of marine vehicle control systems it is
common practice to employ easily-obtained steady-state thruster models.
%
This approach works extremely well for steady-state or slowly
time-varying vehicle motion, but results in the presence of unmodeled 
thruster dynamics during highly dynamic vehicle maneuvering.
%
In 1984, Rohrs et al. famously showed that stable adaptive controllers
for linear time-invariant plants can be destabilized by the presence
of unmodeled plant dynamics \cite{rohrs.1984}. % ,middleton1988design} 
%
To the best of our knowledge, this is the first observation of
unmodeled thruster dynamics resulting in the destabilization of
\ac{AMBC}. %model-based adaptive tracking control of an \ac{UV}.
