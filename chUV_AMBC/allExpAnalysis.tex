\subsection{The Effects of Unmodeled Thruster Dynamics on \ac{AMBC}}

The experiment reported in Section
\ref{chUV_AMBC.sec.fullAnalysisFailure} shows, curiously, a clear
differentiation in parameter adaptation performance; unstable
parameter adaption occurred only in the parameter estimates associated
with pitch and roll dynamics.
%
To further explore this effect, in Section
\ref{chUV_AMBC.sec.singleDOFAnalysisFailure} we investigated parameter
adaptation during pitch-only reference trajectory excitation.
%
These data show thrust reversals cause parameter instability.
%
Further, the thruster angular velocity data indicate a difference
between the actual and commanded torques applied to the vehicle. 
%
Based on our knowledge of the \ac{JHUROV} control system and thruster
design, the data from these pitch-only experiments suggest that
unmodeled thruster dynamics are present during thrust reversals.
%
Without further experimental analysis we can not specify if the
specific mechanism causing unstable parameter adaptation is unmodeled
thruster mechanical dynamics, fluid dynamics, mechanical friction
during thrust reversals, or some combination of these mechanisms.
%
Regardless of the underlying source of modeling error, these
experiments suggest that unstable parameter adaptation will occur in
parameters associated with a given \ac{DOF} if the following three
conditions are met:
%
\begin{itemize}
\item mass and gravitational parameter estimates are adapting, 
\item there exists of a single attractive stability point for that \ac{DOF}, and
\item unmodeled thruster dynamics are present.
\end{itemize}


The success of two-step parameter adaptation supports this hypothesis.
%
Implementing the parameter estimation process in two steps removed
the need for simultaneous adaptation of the mass and buoyancy terms.
%
By separating parameter adaptation in this way, an ambiguity in the
adaptation process was removed.
%
Note the three factors listed imply that both the buoyancy and
mass parameter estimates will be affected in the same way by unmodeled
thruster dynamics during thrust reversals.  
%
From the perspective of the \ac{AMBC} algorithm, the deviations from
the position and velocity reference trajectories caused by unmodeled
thruster dynamics are indistinguishable from the deviations which
would occur if either of these parameter estimates (the buoyancy
torque estimate or the inertia estimate) were too large.
%
The effects of the inertia tensor and buoyancy torque parameters on
vehicle position and velocity are also coupled.
%
For instance there will be similarities between the dynamics of a
\ac{UV} with a large mass and large buoyancy torque and a \ac{UV} with
a small mass and small buoyancy torque for a proper scaling of these
properties.
%
During each period of unmodeled thruster dynamics, the simultaneous
unstable adaptation of the inertia and buoyancy estimates was
difficult for the \ac{AMBC} algorithm to overcome because the estimate
of vehicle dynamics was only slightly degraded by the physically
unrealistic changes in these parameter estimates.
%
Setting the buoyancy torque estimate to a fixed value during dynamic
maneuvers removes the possibility of simultaneous adaptation to
physically unrealistic values.






%  might still be close to values which are
% still good at approximating vehicle dynamics.
% %
% However, the difference tween
 
% adapt twoards physically unrealistic parameter values togeather, the
% difference in vehicle dynamnics

% Although these parameters play structrually
% different roles in UV dynamics their effects on the

% their effects for small angles are coupled,  that much
% of the UV modeling errors both having physically unrealstic values can


%  a different form when entering
% into the dynamics equation



%  or the if the
% estimated pitch inertia were too large both parameter update laws will
% experience a large negitive spike since the pitch or roll state
% lingering near zero

% Although   
