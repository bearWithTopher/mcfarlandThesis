%probStatment.tex

%\section{Problem Statement}
%\label{chUV_AMBC.sec.probStatement}

\subsection{Error Coordinates}
\label{chUV_AMBC.sec.errCoord}

The tracking control algorithms presented herein use the error
coordinates $\Delta H$, $\Delta \psi$, $\Delta v$, and $\Delta \theta$.
We define $\Delta H$ as
%
\begin{equation}
\Delta H={^w_d}H^{-1}{^w_a}H.
\end{equation}
%
Note that $\Delta H$ is the transform from the actual
  to desired vehicle frame since $\Delta H={^d_w}H{^w_a}H={^d_a}H$. 
We define the position tracking error ($\Delta
\psi$), velocity tracking error ($\Delta v$), and parameter error
vector ($\Delta \theta$) as, respectively, 
%
\begin{align}
\Delta \psi=&\log_{\SE3}\left(\Delta H \right),
\label{chUV_AMBC.eq.deltaPsi} \\
\Delta v =& {^a}v_a-{^a}v_d,
\label{chUV_AMBC.eq.deltav} \\
x=&\left[ \begin{array}{c}
     \Delta \psi           \\
     \Delta v              \\
\end{array} \right],  ~\text{and}
\label{chVU_AMBC.eq.error_x} \\
\Delta \theta =& \hat{\theta}_{UV}-\theta_{UV}
\label{chUV_AMBC.eq.deltatheta} 
\end{align}
%
where ${^a}v_d=\Ad_{\Delta H^{-1}}{^d}v_d$ is defined using the
adjoint map, $Ad:\SE3\to\mathbb{R}^{6 \times 6}$ defined in
(\ref{chModels.eq.Ad}), which transforms SE(3) velocity vectors
between reference frames.  Using the fact that $\forall
v\in\mathbb{R}^6$ and $\forall H\in\SE3\quad H\widehat{v}H^{-1}=\widehat{\Ad_H v}$ the time derivative of $\Delta
H$ is
%
\begin{align}
\Delta\dot{H}
  =& {^w_d}H^{-1}{^w_a}\dot{H} + {^w_d}\dot{H}^{-1}{^w_a}H  
\nonumber \\ 
  =&\Delta H \widehat{{^a}v_a} -\widehat{{^d}v_d}\Delta H   
\nonumber \\ 
  =& \Delta H\widehat{{^a}v_a} -\Delta H \Delta H^{-1}\widehat{{^d}v_d}\Delta H 
\nonumber \\ 
  =&\Delta H\left( {^a}v_a -\Ad_{\Delta H^{-1}}{^d}v_d\right)^{\widehat{}} 
\nonumber \\ 
  =&\Delta H\widehat{\Delta v}. 
\label{chUV_AMBC.eq.DeltaH_dot}
\end{align}
%
The equality (\ref{chUV_AMBC.eq.DeltaH_dot}) implies that  
%
\begin{equation}\label{chUV_AMBC.eq.delta_psi_dot}
\Delta \dot{ \psi} = \hat{\mathcal{A}}^{-1}(\Delta \psi) \Delta v
\end{equation}
%
where \funDefn{\hat{\mathcal{A}}^{-1}}{\rSp{6}}{\rSp{6\times 6}} is
the inverse SE(3) velocity Jacobian.  The reader is directed to Appendix
\ref{appenJacSE3} for further information about this Jacobian.




We now develop a useful expression for $\Delta \dot{v}$.
Note from (\ref{chUV_AMBC.eq.deltav}) that $\widehat\Delta
v=\widehat{{^a}v_a}-\Delta H^{-1}\widehat{{^d}v_d}\Delta H$. Consider
the following expression for $\Delta\dot{ v}$,
%
\begin{align}
\widehat{\Delta\dot{ v}}
 =&\widehat{{^a}\dot{v}_a}
  -\Delta H^{-1}\widehat{{^d}\dot{v}_d}\Delta H
  -\Delta\dot{ H}^{-1}\widehat{{^d}v_d}\Delta H
%\nonumber \\
  -\Delta H^{-1}\widehat{{^d}v_d}\Delta\dot{ H} 
\nonumber \\
 =&\widehat{{^a}\dot{v}_a}
  -\Delta H^{-1}\widehat{{^d}\dot{v}_d}\Delta H
  +\widehat{\Delta v}\Delta H^{-1}\widehat{{^d}v_d}\Delta H
%\nonumber \\
  -\Delta H^{-1}\widehat{{^d}v_d}\Delta H\widehat{\Delta v}
\nonumber \\
 =&\widehat{{^a}\dot{v}_a}
  -\left(\Ad_{\Delta H^{-1}}{^d}\dot{v}_d\right)^{\widehat{}}
  +\widehat{\Delta v}\left(\Ad_{\Delta H^{-1}}{^d}v_d\right)^{\widehat{}}
%\nonumber \\
  -\left(\Ad_{\Delta H^{-1}}{^d}v_d\right)^{\widehat{}}\widehat{\Delta v}.
\label{chUV_AMBC.eq.Delta_v_dot_se3}
\end{align}
%
Note that the last two terms (\ref{chUV_AMBC.eq.Delta_v_dot_se3}) are
the Lie bracket of $\Delta v$ and ${^a}v_d$. Using the se(3)
adjoint operator, $\ad:\mathbb{R}^6\to\mathbb{R}^{6\times 6}$ defined
in (\ref{chModels.eq.ad}), with $\mathbb{R}^6$ representation of
velocities is analytically equivalent to the Lie bracket operation on
their $\se3$ representations, i.e. $\forall x,y\in\mathbb{R}^6$ we
know $\widehat{\ad_x y}=\widehat{x}\widehat{y}-\widehat{y}
\widehat{x}$. Thus, from (\ref{chUV_AMBC.eq.Delta_v_dot_se3}), we have
%
\begin{equation}\label{chUV_AMBC.eq.delta_v_dot}
\Delta\dot{ v} = {^a}\dot{v}{_a}-{^a}\dot{v}{_d}+\ad_{\Delta v} {^a}v_d.
\end{equation}

