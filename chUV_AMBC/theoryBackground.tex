
Recent advances in enabling technologies for \acp{UV} including {\it in-situ}
sensing, power storage, and communication modalities have enabled the
development of \acp{UV} which can accomplish missions previously thought
impractical or impossible.
%
Many of these missions, such as seafloor surveying and
environmental monitoring, can depend on tracking a specified trajectory as
closely as possible.
%
%Instrumentation placement is critical, requiring precise control of \ac{UV}
%orientation as well as \ac{UV} location and speed.
%
To facilitate these missions, novel \ac{UV} control algorithms may
provide improved trajectory-tracking precision.
%greater than that of traditional \ac{PDC}.
%
\ac{UV} \ac{MBC} has been shown experimentally to provide significant
performance gains over \ac{PDC} \cite{martinControl_ICRA13}, however \ac{MBC}
requires model parameters to be known {\it a priori}.
% %
% however the parameters required for \ac{MBC} must be experimentally identified;
% the drag parameters and mass parameters 
% %(which include both the
% %characteristics of the vehicle's mass and those of the ambient fluid
% %surrounding the vehicle) 
% cannot be computed analytically.
%
\ac{UV} \acf{AMBC} removes the need for a previously identified model.
%
In this Section we report novel \ac{MBC} and \ac{AMBC} algorithms for a
fully-coupled \ac{UV} plant model.


This Section is organized as follows: Sections
\ref{chUV_AMBC.sec.states}, \ref{chUV_AMBC.sec.UV_dynamics},
and \ref{chUV_AMBC.sec.errCoord} present the state
representations, \ac{UV} plant model, and error coordinates used
through the rest of this Section.
%
Section \ref{chUV_AMBC.sec.MBC} reports a \ac{UV} \ac{MBC} algorithm.
%
Section \ref{chUV_AMBC.sec.AMBC} reports a novel \ac{UV} \ac{AMBC}
algorithm.
%
Both assume a model of rigid-body \ac{UV} dynamics parametrized by
hydrodynamic mass parameters; quadratic drag parameters; gravitational
force and buoyancy parameters.
%
For both results, a local stability proof is reported showing that the
vehicle position and velocity states asymptotically converge to the
desired reference trajectory.
%
For the \ac{AMBC} algorithm reported, the parameter estimates are shown
to be stable and converge asymptotically to values that provide
input-output model behavior identical to that of the actual plant.
%
Section \ref{chUV_AMBC.sec.twoStepMethod} reports how the \ac{MBC} and
\ac{AMBC} algorithms reported herein can be used to identify subsets
of plant parameters if other parameters are known.  
%
This Section also reports a two-step \ac{AMBC} algorithm shown
experimentally in Section \ref{chUV_AMBC.sec.twoStepExp} to be robust
to actuator modeling errors observed in our experimental vehicle.


This Section omits explicit notation of variable dependence on time
except where such dependence is required to discuss the initial
condition of a controller.
