%roadmap.tex
\section{Thesis Outline}


The main contributions of this Thesis are presented in Chapters
\ref{chSMS_ID}, \ref{chUV_AID}, and \ref{chUV_AMBC}.
%
A stability analysis for each novel result is included.


\noindent{\bf Chapter \ref{chModels} - Modeling Second-Order
  Mechanical Systems:} This Chapter defines the notation, functions,
state representations, and second-order plant models used in this
Thesis.


\noindent{\bf Chapter \ref{chSMS_ID} - State Estimation
  and Parameter Identification for Simple Mechanical Systems (\acs{SMS}):}
The standard models for \ac{UV} dynamic operation evolve on the set of
rigid-body transformations, SE(3).  Thus, the first third of this
Thesis is focused on utilizing the group structure of SO(3) and SE(3)
in the development of one state observer and two \ac{AID} algorithms
for \ac{SMS}.
%
These results are the theoretical antecedents of the \ac{UV} results
reported in Chapters \ref{chUV_AID} and \ref{chUV_AMBC}.
%
The plant of the angular velocity observer is a 3-\ac{DOF}
second-order rigid-body rotating under the influence of an externally
applied torque.
%
Numerical simulations of the novel angular velocity observer and two
previously reported observers are included.
%
The simulation results show similar performance of all three observers
for most angular position trajectories and inertia tensors.
%
A novel \ac{AID} algorithm is reported for a 3-\ac{DOF}
second-order rigid body rotating under the influence of an externally
applied torque, along with a simulation study.
%
An \ac{AID} algorithm for \acp{OKC} is also reported.


\noindent{\bf Chapter \ref{chUV_AID} - Adaptive Identification (\acs{AID}) for
  Underwater Vehicles (\acs{UV}):}
The \ac{UV} \ac{AID} algorithms reported herein estimate the plant
parameters for hydrodynamic mass, quadratic drag, gravitational force,
and buoyancy for a second-order rigid-body plant under the
influence of actuator forces and torques.
%
An experimental comparison of \ac{AID} and conventional
\ac{LS} %least squares parameter identification 
is reported.
%
The experimental results corroborate the analytic stability analysis,
showing that the adaptively estimated plant parameters are stable and
converge to values that provide plant-model input-output behavior
closely approximating the input-output behavior of the actual
experimental \ac{UV}.
%
The \acp{AIDPM} are shown to be similar to the \acp{LSPM} in their
ability to match the actual vehicle's input-output characteristics.



\noindent{\bf Chapter \ref{chUV_AMBC} - Adaptive Model-Based Control
  For Underwater Vehicles :}
This Chapter reports two tracking controllers: a \ac{UV} \ac{MBC}
algorithm which provides asymptotically exact trajectory-tracking, and
a novel \ac{UV} \ac{AMBC} algorithm which provides asymptotically
exact trajectory-tracking while estimating the parameters for
hydrodynamic mass, quadratic drag, gravitational force, and buoyancy
parameters for a fully-coupled second-order rigid-body \ac{UV} plant
model.

A comparative experimental evaluation of \ac{PDC} and
\ac{AMBC} for \acp{UV} is reported.
%
To the best of the author's knowledge, this is the first such
evaluation of model-based adaptive tracking control for underwater
vehicles during simultaneous dynamic motion in all 6-\ac{DOF}.
%
This experimental evaluation revealed the presence of unmodeled
thruster dynamics arising during reversals of the vehicle's thrusters,
and that the unmodeled thruster dynamics
%these short duration, small magnitude unmodeled deviations from the
%  commanded input 
can destabilize parameter adaptation.
%
The three major contributions of this experimental evaluation are: an
experimental analysis of how unmodeled thruster dynamics can
destabilize parameter adaptation, a two-step adaptive model-based
control algorithm which is robust to the thruster modeling errors
present, and a comparative experimental evaluation of \ac{AMBC} and
\ac{PDC} for fully-actuated underwater vehicles
preforming simultaneous 6-\ac{DOF} trajectory-tracking.



\noindent{\bf Chapter \ref{chConc} - Conclusion:}
Thesis results summarized and directions for future work presented.



\noindent{\bf Appendix \ref{appenJHUHTF} - Johns Hopkins University
  Hydrodynamic Test Facility:}
The details of the Johns Hopkins University Hydrodynamic Test Facility
and our experimental testing methodologies are presented.
%
This facility includes the \ac{JHUROV}, a fully actuated \ac{UV} used
for the experimental trials reported in Chapters \ref{chUV_AID} and
\ref{chUV_AMBC}.



\noindent{\bf Appendix \ref{appenJacSE3} - Special Euclidean Group
  Velocity Jacobian:}
Details of the SE(3) Velocity Jacobian are reported.  These results
are required for the \ac{UV} \ac{MBC} stability proof (Theorem
\ref{chUV_AMBC.theo.UV_MBC}).
%


%say something like: 

%Using  currently being preformed
%between PD control, non-adaptive model-based control and adaptive
%model-based control for a number of trajectories including set-point
%control, single degree-of-freedom motion, planar motion and
%simultaneous six-degree-of-freedom motion.  To the best of my
%knowledge, this will be the first set of experiments to identify
%classes of maneuvers where \ac{UV} operators can expect to see
%performance gains from adaptive model based control.
