This Thesis reports several novel algorithms for state observation,
parameter identification, and control of second-order plants.
%
A stability proof for each novel result is included.
%
The primary contributions are adaptive algorithms for \ac{UV}
plant parameter identification and model-based control.
%
Where possible, comparative experimental evaluations of the novel
\ac{UV} algorithms were conducted using the Johns Hopkins University
Hydrodynamic Test Facility.


The \ac{UV} \ac{AID} algorithms reported herein estimate the plant
parameters (hydrodynamic mass, quadratic drag, gravitational force,
and buoyancy parameters) of second-order rigid-body \ac{UV} plants
under the influence of actuator forces and torques.
%
Previous adaptive parameter identification methods have focused on
model-based adaptive tracking controllers; however, these approaches
are not applicable when the plant is either uncontrolled, under
open-loop control, or using any control law other than a specific
adaptive tracking controller.
%
The \ac{UV} \ac{AID} algorithms reported herein do not require
simultaneous reference trajectory-tracking control, nor do they require
instrumentation of linear acceleration or angular acceleration.
%
Thus, these results are applicable in the commonly occurring cases of
uncontrolled vehicles, vehicles under open-loop control, vehicles
using control methods prescribed to meet other application-specific
considerations, and vehicles not instrumented to measure angular
acceleration.
% 
In comparative experimental evaluations, \acp{AIDPM} were shown to
accurately model experimentally measured \ac{UV} performance.


The \ac{UV} \ac{MBC} and \ac{AMBC} algorithms reported herein provide
asymptotically exact trajectory-tracking for fully coupled
second-order rigid-body \ac{UV} plants.
%
In addition, the \ac{AMBC} algorithm estimates the plant parameters
(hydrodynamic mass, quadratic drag, gravitational force, and buoyancy
parameters) for this class of plants.
%
A two-step \ac{AMBC} algorithm is also reported which first identifies
gravitational plant parameters to be used in a separate \ac{AMBC}
algorithm for trajectory-tracking.
%
% A two-step \ac{AMBC} algorithm is reported which
% %
% \begin{itemize}
% \item identifies the gravitational plant parameters
% %without requiring knowlege of the mass and drag parameters
% using a special class of quasi-static reference trajectories and
% %
% \item uses the identified gravitational plant parameters in a seperate
% \ac{AMBC} algorithm which estimates the mass and drag plant parameters.
% \end{itemize}
%
%
%An experimental analysis investigates the destabilizing effects of
%unmodeled thruster dynamics on this \ac{AMBC} for certain classes of
%reference trajectories. 
%
We report a comparative experimental analysis of \ac{PDC} and \ac{AMBC}
during simultaneous motion in all degrees-of-freedom. This analysis shows:
\begin{itemize}
\item \ac{AMBC} (i.e. simultaneous adaptation of all plant parameter estimates)
can be unstable in the presence of unmodeled thruster dynamics
% 
\item two-step \ac{AMBC} is robust to the presence of unmodeled
thruster dynamics, and
%
\item two-step \ac{AMBC} provides 30\% better position tracking
performance and 8\% worse velocity tracking performance over \ac{PDC}.
\end{itemize}
%
To the best of our knowledge, the reported comparative experimental
evaluation of \ac{AMBC} and \ac{PDC} is the first to consider
trajectory-tracking performance during simultaneous motion in all
degrees-of-freedom.
