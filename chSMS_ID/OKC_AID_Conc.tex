

\subsection{Open Kinematic Chain Adaptive Identification Conclusion}
%TM-CHECK do you still like the persistent excitation sentence below.

In this Section an \ac{AID} algorithm for a robotic manipulator is
reported.
%Adaptive identification is the only method to date which
%does not require access to acceleration data and is guaranteed to
%provide a physically feasible model.  
The \ac{AID} algorithm presented herein has the advantages of being
intuitive, having no requirement for joint acceleration, and providing
physically feasible plant parameter estimates.  However, an
experimental comparison with proven linear regression techniques, such
as the method proposed by Khalil et al. \cite{Khalil2007}, would be
required to understand the comparative performance of this adaptive
identifier. We are interested in exploring \ac{AID} of coupled \ac{UV}
\ac{OKC} systems with applications to work-class \ac{ROV} deployments.
With only a brief discussion of persistent excitation for \acl{OKC}
\acl{AID} algorithms reported thus far
\cite{hsu&bodson&sastry&paden.icra87}, we feel further consideration
of persistent excitation in the context of manipulator \acl{AID} would
clarify the comparative strengths and weaknesses between \acl{AID} and
the more established methods.
