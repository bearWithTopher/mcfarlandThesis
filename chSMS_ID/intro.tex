%\section{Introduction}
%\label{chSMS_ID.sec.intro}

This Thesis reports advances in state estimation, parameter
identification, and control algorithms applicable to \acfp{UV}.
Chapters \ref{chUV_AID} and \ref{chUV_AMBC} present the development
and experimental evaluation of \ac{UV} algorithms.  However, the
theoretical antecedents of these algorithms are state and parameter
estimation algorithms for a broader class of \acfp{SMS}.  This
Chapter presents three separate algorithms: an angular velocity
observer for a second-order rotating rigid-body plant (Section
\ref{chSMS_ID.sec.SO3_velObs}); an \acl{AID} algorithm for the inertia
tensor of a second-order rotating rigid-body plant (Section
\ref{chSMS_ID.sec.SO3_AID}); and an \acl{AID} algorithm for the
dynamic parameters of an \ac{OKC} (Section
\ref{chSMS_ID.sec.OKC_AID}).



The angular velocity observer and the inertia tensor \ac{AID}
algorithm address 3-\ac{DOF} second-order rotating rigid-body plants.
%
As such, they are applicable for a number of space, air,
and marine vehicle applications.
%
The \ac{AID} algorithm for second-order rotational
plants preforms dynamic estimation of the inertia tensor from
input-output signals.
%
In a variety of undersea, space, and air vehicle missions, the vehicle
inertia tensor varies dynamically as consumables and payload vary over
the duration of a mission.  Thus dynamic inertia tensor estimation
could facilitate forward modeling and model-based control with such
vehicles.
%
%Continuous parameter monitoring of the inertia tensor may also enable
%the detection of unexpected changes that indicate system failures.




The \ac{OKC} \ac{AID} algorithm estimates the plant model dynamic
parameters that enter linearly into a general class of nonlinear
second-order holonomic plants, including robotic manipulators.
%
Dynamic parameters that enter linearly into the plant model such as
mass, inertia, and friction coefficients can be estimated.  
%
Most previously reported \ac{AID} algorithms methods for this
class of plants have focused on model-based adaptive tracking
controllers.
%
However these approaches are not applicable when the manipulator is
either uncontrolled, under open-loop control, under actuated, or is
using any control law other than a specific adaptive tracking
controller.
%
The \ac{AID} reported herein does not require any
particular control algorithm and is also suitable for uncontrolled
plants.
%
In the case of both \ac{AID} algorithms presented in
this Chapter, continuous parameter monitoring of plant parameters may
enable the detection of unexpected changes that indicate system
failures.




The \ac{AID} algorithm presented in Section
\ref{chSMS_ID.sec.SO3_AID_Der} was originally reported in 2012\cite{mcfarland2012}.
