\subsection{Angular Velocity Observer Conclusions}  
\label{chSMS_ID.sec.AVO_conc}


This Section reports a comparative analysis of three angular velocity
observers for second-order rotational plants of the form
(\ref{chModels.eq.SO3plant}) for which the inertia tensor is known,
and the signals of angular position and torque input are available.
Two very different previously reports angular velocity observers were
reviewed \cite{Salcudean1991,Maithripala2004}.  We report one novel
angular velocity observer together with a proof of its local
asymptotic stability.  The results of a comparative numerical
simulation study of the three observers is reported.  The observers
were seen to provide similar performance over a range of inertia
tensors, angular position profiles, and feedback gains.  For the range
of inertia tensors used, each of the three observers were analytically
guaranteed to converge to the correct angular velocity estimate.
%
However, these analytic stability analyses do not provide information
on the rate of convergence.
%
Each simulated observer's angular velocity estimate converged to the
plant state as expected; however, the coordinate free and body-frame
observers (though more complex to implement) were not seen to display
the underdamped behavior which appeared to degrade the world-frame
observer's state estimate convergence rate for some of the simulated
comparisons in our ensemble of simulation studies.
%
For simple second-order linear systems, underdamped behavior results
from a poorly tuned combination of the proportional and derivative
gains; we would expect similar performance for each of the three
observers when state errors are small enough that the linearization
assumptions used to match the gains are valid.
%Since the structure of this
%nonlinear observer prevents tuning both the proportional and
%derivative gains, 
%
The lack of underdamped behavior in the coordinate free and body-frame
observers could indicate that the additional nonlinear terms improve
performance of both when the linearization assumptions are no longer
valid.
%
However, the difference in structure of these three nonlinear
observers makes it difficult to analytically prove the existence of
such benefits.
%
Exploring these differences, as well as any link between the extra
coordinate free and body-frame terms, is a topic for further research.
