\subsection{\acs{OKC} State Error Coordinates}

The \ac{AID} algorithm presented herein uses the joint velocity
estimate $v\in \mathbb{R}^{n}$ and the parameter vector estimate
$\OKCparamEst\in\mathbb{R}^{r}$ as its state variables. Let
$\mathfrak{J}\subset\rSp{n}$ be the \ac{OKC}'s joint space and note
estimated plant terms $\hat{M}$, $\hat{C}$, and $\hat{\OKCg}$ can be
factored as
%
\begin{equation}
\hat{M}(a)b+\hat{C}(a,c)d+\hat{\OKCg}(a)=\OKCreg(b,a,c,d)\OKCparamEst
\end{equation}
%
for all $a,b,c,d\in\rSp{n}$ where
\funDefn{\OKCreg}{\rSp{n}\times\mathfrak{J}\times\rSp{n}\times\rSp{n}}{\rSp{n\times
    r}} is often termed the {\it regressor}.
%
We employ the following joint velocity and inertia tensor
error coordinates
%
\begin{align}
\Delta v&=v(t)-\dot{\OKCpos}(t)           \label{chSMS_ID.eq.OKC_AID_deltaVel} \\
\Delta \OKCparam&=\OKCparamEst(t) - \OKCparam.    \label{chSMS_ID.eq.OKC_AID_deltaPar} 
\end{align}
%
%Talk about eigenvalue conventions and vector-matrix norms here if
%required 
%
Our goal is to develop an update law for $v$ and
$\OKCparamEst$ which, for initial values of $\OKCparamEst(t_0)$ in the
neighborhood of $\OKCparam$, guarantees that $\lim_{t\to
  \infty}\Delta v=\vec{0}$ and $\lim_{t\to \infty}\Delta
\OKCparamDot=\vec{0}$. Considering torque as the system input and
joint velocity as the system output, the two limits above imply the
input-output behavior of the estimator asymptotically converges to the
input-output behavior of the plant.


\subsection{Adaptive Identifier Description}



\begin{OKC_AID} \label{chSMS_ID.theo.OKC_AID}
  Consider the following \ac{AID} algorithm for plants of the form
  (\ref{chModels.eq.OKCplant}):
%
  \begin{align}
    \dot{v}&=\hat{M}^{-1}(\OKCpos) \left(\OKCinput - \hat{C}(\OKCpos,\dot{\OKCpos}) v
      -\hat{\OKCg}(\OKCpos)- K\Delta v\right)
                                              \label{chSMS_ID.eq.OKC_AID_estimator}\\
    \OKCparamEstDot&= \OKCreg(\dot{v},\OKCpos,\dot{\OKCpos},v)^T\Delta v,
                                              \label{chSMS_ID.eq.OKC_AID_identifier}
  \end{align}
%
  where $K\in\rSp{n\times n}$ is \ac{SPD}, with the following assumptions:
%
  \begin{itemize}
  \item $\OKCinput$, $\OKCpos$ and $\dot{\OKCpos}$ are bounded by assumption
  \item $v(t_0)=\dot{\OKCpos}(t_0)$
  \item $\exists \epsilon \in \rSp{}_+$ such that
% 
    \begin{equation}
      \label{chSMS_ID.eq.OKC_AID_massLowerBound}
      \|\Delta
      \OKCparam(t_0)\|_2\leq \frac{1}{\sqrt{a_M r}}\left(\lambda_m-\epsilon\right)
    \end{equation}
%
    where $\lambda_m$ is the smallest eigenvalue for $M(\OKCpos)$ in
    any configuration and 
%
\begin{equation}
  a_M=\max_{\hat{\theta}\in\{e_1,e_2,...,e_r\}}\left(\max_{\OKCpos\in\rSp{r}}\left(\eig(\hat{M}(\OKCpos))\right)\right)
\end{equation}
%
where the set $\{e_i\}$ are the standard unit length basis vectors of the space $\rSp{r}$
%$a_M=$ is the largest eigenvalue of
%    $\hat{M}(\OKCpos)$ for any configuration corresponding to
%    $\OKCparamEst=e_i$ for all the standard unit length basis vectors
%    $e_i$ 
(see Section \ref{chSMS_ID.sec.OKC_AID_boundM} for a
    complete development of $a_M$).
  \end{itemize}
%
  \noindent Under these conditions the joint velocity error will
  be asymptotically stable in the sense of Lyapunov, i.e., $\lim_{t\to
    \infty}\Delta v =\vec{0}$, and the model using the estimated
  parameters will be indistinguishable from the model using the true
  parameters for the given torque input, i.e., $\lim_{t\to
    \infty}\Delta \OKCparamDot=0$.
\end{OKC_AID}

\subsection{Error System}\label{chSMS_ID.sec.OKC_AID_error}

Consider (\ref{chModels.eq.OKCplant}),
(\ref{chSMS_ID.eq.OKC_AID_estimator}), (\ref{chModels.eq.OKCplantW}),
and (\ref{chSMS_ID.eq.OKC_AID_deltaPar}) as well as the fact that
%
\begin{equation*}
\left(\OKCreg(b_1,a,c,d_1)-\OKCreg(b_2,a,c,d_2)\right)\OKCparam=M(a)(b_1-b_2)+C(a,c)(d_1-d_2)
\end{equation*}
%
in the following equalities
%
\begin{align}
  0&=\OKCinput-\OKCinput  \nonumber \\
  &=\OKCreg(\dot{v},\OKCpos,\dot{\OKCpos},v)\OKCparamEst
  + K\Delta v - \OKCreg(\ddot{\OKCpos},\OKCpos,\dot{\OKCpos},\dot{\OKCpos})\OKCparam \nonumber \\
  &=\OKCreg(\dot{v},\OKCpos,\dot{\OKCpos},v)\Delta\OKCparam + K\Delta v+\left(
    \OKCreg(\dot{v},\OKCpos,\dot{\OKCpos},v)
    - \OKCreg(\ddot{\OKCpos},\OKCpos,\dot{\OKCpos},\dot{\OKCpos})\right)\OKCparam     \nonumber \\
  &=\OKCreg(\dot{v},\OKCpos,\dot{\OKCpos},v)\Delta\OKCparam + K\Delta v +
  M(\OKCpos)\Delta \dot{v}+C(\OKCpos,\dot{\OKCpos})\Delta v. 
  \label{chSMS_ID.eq.OKC_torqueTrick}
\end{align} 
%
The final equality of (\ref{chSMS_ID.eq.OKC_torqueTrick}) can be
rewritten as
%
\begin{equation}\label{chSMS_ID.eq.OKC_AID_deltaJointAcc}
\Delta\dot{v}=M^{-1}(\OKCpos)\left(-C(\OKCpos,\dot{\OKCpos})\Delta v 
   -K\Delta v-\OKCreg(\dot{v},\OKCpos,\dot{\OKCpos},v)\Delta \OKCparam\right).
\end{equation}

\subsection{Lyapunov Stability}\label{chSMS_ID.sec.OKC_AID_lyap}

Consider the following Lyapunov function candidate:
%
\begin{equation}\label{chSMS_ID.eq.OKC_AID_lyap}
V(t)=\frac{1}{2}\Delta v^{T} M(\OKCpos) \Delta v + 
     \frac{1}{2}\Delta \OKCparam^{T} \Delta \OKCparam.
\end{equation}
%
$V(t)$ is
%
\begin{itemize}
\item positive definite
\item radially unbounded
\item equal to zero if and only if $\Delta v=\vec{0}$ and
$\Delta \OKCparam=\vec{0}$.
\end{itemize}
%
Using (\ref{chSMS_ID.eq.OKC_AID_deltaJointAcc}) the time derivative of
(\ref{chSMS_ID.eq.OKC_AID_lyap}) is
%
\begin{align*}
\dot{V}(t)
  =&\Delta v^{T} M(\OKCpos) \Delta \dot{v} 
    +\frac{1}{2}\Delta v^{T} \dot{M}(\OKCpos) \Delta v 
    +\Delta \OKCparam^{T} \Delta \OKCparamDot \nonumber \\
  =&-\Delta v^{T} K \Delta v
    +\Delta v^{T}\left(\frac{1}{2}\dot{M}(\OKCpos)-C(\OKCpos,\dot{\OKCpos})\right)\Delta v
                                          \nonumber \\
   &-\Delta v^{T} \OKCreg(\Delta \dot{v},\OKCpos,\dot{\OKCpos},\Delta v)\OKCparam
    +\Delta \OKCparam^{T} \Delta \OKCparamDot %\nonumber \\
\end{align*}
%
Using the fact that $\dot{M}(\OKCpos)-2C(\OKCpos,\dot{\OKCpos})$ is skew
symmetric and the update law (\ref{chSMS_ID.eq.OKC_AID_identifier}) results in
%
\begin{equation}\label{chSMS_ID.eq.OKC_AID_lyapDot}
\dot{V}(t)=-\Delta v^{T} K \Delta v.
\end{equation} 
%
\sloppy{
$\dot{V}(t)$ is negative definite in $\Delta v$ and negative
semidefinite in the error coordinates $\Delta v$ and $\Delta \OKCparam$.
Lyapunov's theorem and (\ref{chSMS_ID.eq.OKC_AID_lyap}) -
(\ref{chSMS_ID.eq.OKC_AID_lyapDot}) imply that $\Delta v$ and $\Delta
\OKCparam$ are bounded and stable.  The structure of $\dot{V}(t)$ implies
that $\Delta v \in \mathcal{L}_2$, or equivalently
$\lim_{t\to\infty}\left( \int_0^t\Delta v^T \Delta
  v\right)^{1/2}<\infty$.  To ensure all the signals in
(\ref{chSMS_ID.eq.OKC_AID_estimator}) are bounded, we must ensure that
both $\hat{M}^{-1}(\OKCpos)$ and $\hat{M}(\OKCpos)$ remain bounded.}
%
%Note that the matrix $M(\OKCpos)$ is positive definite symmetric for all $\OKCpos$
%and that the eigenvalues of $\hat{M}(\OKCpos)$ vary continuously with changes
%in $\OKCparamEst$.  Therefore there exists a set of parameters
%$\OKCparamEst$ for which $\hat{M}(\OKCpos)$ is Symmetric Positive Definite
%for all $\OKCpos$ and further because $\Delta \OKCpos(t_o)=\vec{0}$, $0\leq
%V(t)\leq V(t_0) $ and $V(t_0)=\Delta \OKCparam(t_0)^T\Delta
%\OKCparam(t_0)=\|\Delta \OKCparam(t_0)\|^2$ then if $\hat{M}(\OKCpos)$ is Positive
%Definite Symmetric for all $\OKCpos$ and for all $\hat{\OKCparam(t_0)}$ such
%that $\|\OKCparam-\hat\OKCparam(t_0)\|\leq \|\Delta \OKCparam (t_0)\|$. For
%$\OKCparamEst(t_0)$ which start in this set the above condition
%guarantees that $\hat{M}^{-1}(\OKCpos)$ and $\hat{M}(\OKCpos)$ are positive
%definite and bounded for all time.  
%
Section \ref{chSMS_ID.sec.OKC_AID_boundM} proves both are implied by the conditions
specified in Theorem \ref{chSMS_ID.theo.OKC_AID}.
%
Since $\OKCinput$, $\OKCpos$, and $\dot{\OKCpos}$ are bounded by assumption,
bounded $\Delta v$, $\hat{M}(\OKCpos)$, and $\hat{M}^{-1}(\OKCpos)$ imply that
$\dot{\OKCpos}$ and $\OKCparamEstDot$ are bounded.  The bounded
joint velocities imply $\Delta \dot{v}$ is bounded.  Note that
bounded $\Delta \dot{v}$ and $\Delta v \in \mathcal{L}_2$
implies
%
\begin{equation}
\lim_{t\to \infty}\Delta v=\vec{0}.
\end{equation}
%
Moreover, since every signal in $\OKCparamEstDot$ is
bounded and $\lim_{t\to \infty}\Delta v=\vec{0}$ this implies
%
\begin{equation}
\lim_{t\to \infty}\OKCparamEstDot=\vec{0}.
\end{equation}
%
Thus the estimator's joint velocity asymptotically converges
to the joint velocity of the actual plant, and the estimated plant
parameters, $\OKCparamEst$, asymptotically converge to a constant
value.

The local stability of this \ac{OKC} \ac{AID} algorithm is proven if
condition (\ref{chSMS_ID.eq.OKC_AID_massLowerBound}) is sufficient to
bound the mass matrix estimate eigenvalues away from zero.



\subsection{Bound for Estimated Inertia Matrix}\label{chSMS_ID.sec.OKC_AID_boundM}

The final requirement to prove Theorem \ref{chSMS_ID.theo.OKC_AID} is
that the condition $\|\Delta \OKCparam(t_0)\|_2\leq\frac{1}{\sqrt{a_M
    r}}\left(\lambda_m-\epsilon\right)$ implies 
$\hat{M}(\OKCpos,t)$ and $\hat{M}^{-1}(\OKCpos,t)$ are bounded for all
time and all configurations.  Before showing the eigenvalues of
$\hat{M}(\OKCpos,t)$ are bounded both above and below we need to
further clarify some facts about manipulator mass matrices and define
a useful vector semi-norm.

Since $M(\OKCpos)$ is \ac{SPD} we know its largest and
smallest singular values are its largest and smallest
eigenvalues. Further, we know that for any physical manipulator every
eigenvalue of $M(\OKCpos)$ is both positive and bounded for every joint
configuration, $\OKCpos$, in the manipulator's joint space, $\mathfrak{J}$.
%
In the sequel, we will use the following definitions: 
let the
constants $\lambda_m,\lambda_M\in\rSp{}_+$ be such that
$\lambda_m=\min_{\OKCpos\in\mathfrak{J}} \left(\min_{\|x\|_2=1}x^T M(\OKCpos) x
\right)$ and $\lambda_M=\max_{\OKCpos\in\mathfrak{J}}\|M(\OKCpos)\|_2$.  Since
$M(\OKCpos)$ is assumed to be completely known up to the uncertain base
parameters, $\OKCparami$, this knowledge allows the complete
specification of a set of functions
$A_i:\mathbb{R}^n\to\mathbb{R}^{n\times n}$ such that
%
\begin{equation}
M(\OKCpos)=\sum_{i=1}^r \OKCparami A_i(\OKCpos).
\end{equation} 
%
Further, each $A_i(\OKCpos)$ is symmetric and bounded,
i.e. $\forall i\quad\exists a_i\geq 0$ such that $\|A_i(\OKCpos)\|_2\leq a_i$
$\forall \OKCpos\in \mathfrak{J}$. Based on these non-negative bounding
scalars we define the following vector semi-norm
%
\begin{equation}
\|b\|_a=\sum_{i=1}^r a_i |b_i|
\end{equation}
%
where $b=[b_1~\cdots ~ b_r]^T\in \mathbb{R}^r$.  Note that for
any two vector norms there exists a constant, $p$, such that
$\|b\|_a\leq p \|b\|_2$ \cite{horn&johnson}. To find a conservative value
for $p$ in the case of $\|\bullet\|_a$ and $\|\bullet\|_2$ we use that
$0\leq(|b_i-b_j|)^2$ implies $2|b_i||b_j|\leq |b_i|^2 + |b_j|^2$
$\forall i,j$.  Let $a_M$ be such that $\forall i\quad a_M\geq a_i$ and
consider
%
\begin{align}
  \|b\|_a^2
  &=\left(\sum_{i=1}^r a_i |b_i|\right)^2                \nonumber \\
  &=\sum_{i=1}^r a_i^2 |b_i|^2 +
   \sum_{i=1}^{r-1}\sum_{j=i+1}^{r}2a_ia_j|b_i||b_j|        \nonumber \\
  &\leq \sum_{i=1}^r a_M^2 |b_i|^2 +
   \sum_{i=1}^{r-1}\sum_{j=i+1}^{r}2a_M^2|b_i||b_j|         \nonumber \\
  &\leq a_M\left(\sum_{i=1}^r |b_i|^2 +\sum_{i=1}^{r-1} 
   \sum_{j=i+1}^{r}\left(|b_i|^2 + |b_j|^2\right)\right)  \nonumber \\ 
  &\leq a_M\left(\sum_{i=1}^r r |b_i|^2\right)           \nonumber \\ 
  &\leq a_Mr\|b\|_2^2.
\end{align}   
%
Since the square root is a strictly increasing function
$\sqrt{a_M r}$ is a conservative value for $p$ because the previous
inequality implies
%
\begin{equation}
\|b\|_a\leq \sqrt{a_M r} \|b\|_2.
\end{equation}

Now let us turn to the task of bounding $\hat{M}(\OKCpos,t)$.  Note that the facts
$0 \leq V(t) \leq V(t_0)$, $\Delta v(t_0)=\vec{0}$, and $ \Delta v
^T(t) M(\OKCpos) \Delta v (t)\geq 0$ for all $t$ imply that the following
inequality holds $\forall t \geq t_0$:
%
\begin{equation}
\|\Delta \OKCparam(t)\|_2\leq \|\Delta \OKCparam(t_o)\|_2.
\end{equation}
%
\noindent By defining Rayleigh-Ritz Theorem the minimum possible eigenvalue
of $\hat{M}(\OKCpos,t)$ at a certain time, $\hat{\lambda}_m(t)$, is given by
$\hat{\lambda}_m(t)=\min_{\OKCpos\in\mathfrak{J}}\left(\min_{\|x\|=1}x^T
  \hat{M}(\OKCpos,t)x\right)$.  Defining $\Delta M$ such that
$\OKCreg(a,b,c,d)\Delta \OKCparam=\Delta M(b)a+\Delta C(b,c)d+\Delta \OKCg(b)$ note
the implied equality of $\hat{M}(\OKCpos,t)=M(\OKCpos)+\Delta M(\OKCpos,t)$.  Consider
the following, 
%
%
\begin{align}
\hat{\lambda}_m
  &=\min_{\OKCpos\in\mathfrak{J}}\left(\min_{\|x\|=1}x^T \left(M(\OKCpos)
   +\Delta M(\OKCpos,t)\right) x \right)                        \nonumber \\
  &\geq \min_{\OKCpos\in\mathfrak{J}}\min_{\|x\|=1}x^T M(\OKCpos) x -
   \max_{\OKCpos\in\mathfrak{J}}\left(\max_{\|x\|=1}|x^T\Delta M (\OKCpos,t)x|\right)
                                                          \nonumber \\ 
  &\geq\lambda_m-\max_{\OKCpos\in\mathfrak{J}}\left(\max_{\|x\|=1}
   |\sum_{i=1}^r \Delta\OKCparami(t)x^T A(\OKCpos) x|\right)       \nonumber \\ 
  &\geq\lambda_m-\max_{\OKCpos\in\mathfrak{J}}\left(\sum_{i=1}^r \max_{\|x\|=1}
   |\Delta\OKCparami(t)| |x^T A(\OKCpos) x|\right)                \nonumber \\ 
  &\geq\lambda_m-\sum_{i=1}^r |\Delta\OKCparami(t)| a_i      \nonumber \\ 
  &\geq\lambda_m-\|\Delta\OKCparam(t)\|_a                    \nonumber \\ 
  &\geq\lambda_m-\sqrt{a_M r}\|\Delta\OKCparam(t)\|_2         \nonumber \\ 
  &\geq\lambda_m-\sqrt{a_M r}\|\Delta\OKCparam(t_0)\|_2       \nonumber \\ 
  &\geq\lambda_m-\sqrt{a_M r}\left(\frac{1}{\sqrt{a_M r}}
   \left(\lambda_m-\epsilon\right)\right)                 \nonumber \\ 
  &\geq\epsilon.
\end{align}
%
\noindent The previous inequality shows that the condition $\exists
\epsilon\in\mathbb{R}_+$ such that $\|\Delta
\OKCparam(t_0)\|_2\leq\frac{1}{\sqrt{a_M r}}\left(\lambda_m-\epsilon\right)$
implies that $M(\OKCpos)$ is invertible for all time.  Similarly,
%
\begin{align}
\hat{\lambda}_M
  &=\max_{\OKCpos\in\mathfrak{J}}\left(\max_{\|x\|=1}x^T \left(M(\OKCpos)
   +\Delta M(\OKCpos,t)\right) x \right)                        \nonumber \\
  &\leq\lambda_M +
   \max_{\OKCpos\in\mathfrak{J}}\left(\max_{\|x\|=1}|x^T\Delta M (\OKCpos,t)x|\right)
                                                          \nonumber \\ 
  &\leq\lambda_M+\|\Delta\OKCparam(t)\|_a                    \nonumber \\ 
  &\leq\lambda_M+\left(\lambda_m-\epsilon\right)
\end{align}
%
%
\noindent which implies that $\hat{M}(\OKCpos,t)$ is bounded for all time
since $\epsilon<\lambda_m$.



