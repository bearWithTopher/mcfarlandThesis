\section{Literature Review}
\label{chSMS_ID.sec.litReview}

State estimation and parameter identification algorithms for
\acfp{SMS} has been an active research area for
over three decades.  Controllers and observers that do not require
access to angular velocity have been developed for plants for the form
(\ref{chModels.eq.SO3plant}).  Lizarralde and Wen reported an attitude
controller for (\ref{chModels.eq.SO3plant}) that does not require
access to angular velocity\cite{Lizarralde1996}.
%
In \cite{Salcudean1991} Salcudean reported a stable velocity observer
for plants of the form (\ref{chModels.eq.SO3plant}) using unit
quaternion representation of plant angular position.
%
In \cite{Aghannan2003} the authors report general results for
intrinsic observers on a broad class of Lagrangian systems.  In
\cite{Maithripala2004} the authors use the general result of
\cite{Aghannan2003} to address the special case of observers for
mechanical systems on a Lie group.  In \cite{Mahony2005,Mahony2008}
the authors address the problem of developing complementary filters
for the special orthogonal group in the presence of noisy velocity data.
In \cite{Baldwin2009} the authors apply the general approach of
\cite{Mahony2005,Mahony2008} to develop observers
for second-order SE(3) plants.


% Over the past 20 years geometric control researchers have developed
% general tracking controllers for both kinematic first order systems,
% with state space $\SO3$, and dynamic second-order systems, with state
% spaces $\SO3\times \mathbb{R}^3$. To the best of our knowledge, the
% first nonlinear tracking controller for a second order rotating plant
% was reported by Koditschek \cite{Koditschek1988}.  Wen and
% Kreutz-Delgado summarize tracking controllers using unit quaternions
% and full state access in \cite{Wen1991}.  In \cite{Bullo2004} Lewis
% and Bullo report the development of tracking controllers from a
% coordinate free differential geometry approach.  Several papers have
% considered controlling plants similar to (\ref{chModels.eq.SO3plant})
% but also possessing additional features inspired by specific
% applications.  These include a tracking controller for underwater
% vehicles with unknown (but bounded) disturbance forces and moments
% \cite{Sanyal2009}, an attitude stabilizer for a spherically symmetric
% satellite with only magnetic field measurement capabilities and, most
% similar to the work presented herein, 


% Over the past 30 years researchers have developed a number of methods
% to deal with uncertainty in inertial parameters.  Improvements in
% on-board sensor packages have allowed access to more detailed state
% information, leading to the development of new parameter estimation
% techniques that take advantage of these new measurement sources. Batch
% methods such as least squares or extended Kalman filters have been
% used with sets of on-board sensor data to estimate parameters for a
% number of models for robotic systems such as subsea vehicles
% \cite{caccia.joe2000,alessandri.ca98,martin.thesis} and spacecraft
% \cite{Norman2011, Keim2006}.

% Adaptive methods for stable linear plants have been well understood
% for some time \cite{ksn&anu.book}. For plants with rotational DOF, adaptive
% tracking controllers have been used to limit the effect of inertia
% tensor uncertainty in application domains such as robotic manipulators
% \cite{Craig&hsu&sastry.ijrr87,slotine&li.ijrr87}, subsea vehicles
% \cite{Jordan2006}, spacecraft \cite{slotine1990}, and general
% mechanical systems \cite{Lain1997}. Chatervendi et all presents an
% adaptive tracking controller that also identifies the inertia tensor
% of (\ref{eq.plant}) \cite{Chaturvedi2006}. Smallwood and
% Whitcomb present an adaptive identifier for a 6 DOF underwater vehicle
% assuming decoupled dynamics \cite{smallwood2003TCST}. To our
% knowledge the method presented herein is the first to use adaptive
% methods for inertia tensor estimation without the requirement of
% taking control of the plant.

%\cite{Mahony2005,Mahony2008}


A broad class of nonlinear model-based trajectory tracking controllers
developed since the 1980's --- for example Koditschek's nonlinear
tracking controller for second-order rotating plants
\cite{Koditschek1988} and exactly linearizing model-based trajectory
tracking controllers for \acp{OKC}
\cite{freund.ijrr.83,luh&walker&paul.tran.aut.con.80} --- require
exact knowledge of the plant's kinematic and dynamic parameters.
%
Although kinematic parameters are often easily obtained, dynamic
parameters generally must be measured empirically.
%
%The difficulty of measuring point dynamics
%parameters motivated the development of adaptive model-based
%trajectory tracking controllers that estimated plant dynamic
%parameters online.
%
Most previously reported parameter identification methods for
\acp{OKC} employ one of two general approaches: $(i)$ linear
regression of experimental data or $(ii)$ adaptive model-based
trajectory tracking control.


A variety of previously reported studies have employed least-squares,
total least-squares, or extended Kalman filters to identify
plant parameters entering linearly into the plant equations of motion for 
robot manipulators \cite{khosla&kanade,an&atkeson&hollerbach},
\ac{UV} \cite{caccia.joe2000,alessandri.ca98,martin.thesis}, 
and spacecraft \cite{Norman2011,Keim2006}.  Khalil and
Dombre provide a summary of this work \cite{Khalil2002}.


The problem of adaptive model-based reference trajectory tracking is
well understood for several types of second-order holonomic nonlinear
plants whose parameters enter linearly into the plant equations of
motion, e.g., robot manipulator arms
\cite{Craig&hsu&sastry.ijrr87,slotine&li.ijrr87,horowitz&sadegh.ijrr90},
\acp{UV} \cite{Jordan2006}, spacecraft
\cite{kod.cdc85a,slotine1990}, and general mechanical systems
\cite{Lain1997}.
% 
The previously reported result most closely related to the adaptive
identifier reported herein is reported in
\cite{Chaturvedi2006}, which addresses the specific problem of
adaptive model-based tracking control of rotational plants of the form
(\ref{chModels.eq.SO3plant}).
%
\cite{hsu&bodson&sastry&paden.icra87} reports an adaptive
identification algorithm for \ac{OKC}s that employs a low pass
filter approach for the parameter update law to avoid instrumenting
joint acceleration, and reports a numerical simulation.
%
Other than \cite{hsu&bodson&sastry&paden.icra87}, although a great
variety of adaptive model-based trajectory tracking controllers have
been developed, to the best of our knowledge the corresponding
model-based \ac{AID} algorithm --- without requiring simultaneous
trajectory tracking control --- have not been reported.


Adaptive methods for parameter identification of linear plants are
well understood \cite{ksn&anu.book,sastry&bodson.book,astrom.book} and
have been employed for a few application-specific nonlinear models,
such as decoupled model for \acp{UV}
\cite{smallwood2003TCST}.
% 
To the best of our knowledge, the \ac{AID} algorithm reported in
Section \ref{chSMS_ID.sec.SO3_AID} is the first reported inertia
tensor adaptive estimation method for rotational plants of the
form (\ref{chModels.eq.SO3plant}) {\it without} the additional need to
simultaneously perform reference trajectory tracking.
%
Section \ref{chSMS_ID.sec.OKC_AID} reports an \ac{AID}
algorithm which does not require joint acceleration signals and
provides physically feasible models.
%
%These \ac{AID} algorithms may prove useful in applications with
%controlled or uncontrolled plants in which reference trajectory
%tracking is impractical or infeasible.

