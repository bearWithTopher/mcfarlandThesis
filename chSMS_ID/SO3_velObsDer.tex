\subsection{Velocity Observer from Body Frame}
\label{chSMS_ID.sec.AVO}

\newcommand{\omegaInBody}{\psi}

Consider the model of a rotating rigid-body under the influence of an
external torque, $\tau$, of the form (\ref{chModels.eq.SO3plant}) with
angular position, $R$, and angular velocity, $\omega$, repeated here
for the reader as
%
\begin{align} \label{chSMS_ID.eq.SO3plant}
\dot{R} &=R\widehat{\omega}                         \nonumber \\
\dot{\omega} &=I^{-1}\left(\mathcal{J}\left(I\omega\right)\omega+\tau\right).
\end{align}
%
Assume the plant's \ac{SPD} inertia
tensor, $I$, input torque signal, and angular position signal are known
but the angular velocity signal is unknown.  Without loss of
generality we can assume $I$ is diagonal with eigenvalues
$0<\lambda_3\leq \lambda_2\leq \lambda_1<\infty$ in some arbitrary
order along the diagonal.  
%Before we can formally state the conditions
%required to provide an asymptotically exact estimate of the plant's
%angular velocity, we must first develop a convenient form for the
%error dynamics of the observer.  
We define the observer states as $\hat{R}$ and $\hat{\omega}$,
estimates of the plant's angular position $R$ and angular velocity
$\omega$ respectively.  We define the error signals as
%
\begin{align}
\Delta R&=\hat{R}^T R                  \label{chSMS_ID.eq.AVO_deltaR} \\
\Delta q&=\log_{\SO3}(\Delta R)        \label{chSMS_ID.eq.AVO_deltaq} \\
\omegaInBody&=\Delta R^T \hat{\omega}          \label{chSMS_ID.eq.AVO_omegaInBody}    \\
\Delta\omega&=\omega-\omegaInBody.            \label{chSMS_ID.eq.AVO_deltaw} 
\end{align}
%
The remainder of this Section analyzes the stability of
%With these, we will spend the rest of this section considering
the previously unreported nonlinear angular velocity observer
%  
\begin{align}\label{chSMS_ID.eq.AVO}
\dot{\hat{R}} =&\hat{R} \mathcal{J}\left(\hat{\omega}-\Delta R\mathcal{A}(\Delta q) L \Delta q\right)
                                                          \nonumber \\
\dot{\hat{\omega}} 
  =&\Delta R \left( I^{-1}\left(\mathcal{J}\left(I \omegaInBody\right)\omegaInBody + \tau + 
    \Delta q\right)-\mathcal{J}(\omegaInBody)\mathcal{A}(\Delta q)L\Delta q\right)
\end{align}
%
where $L\in\rSp{3\times 3}$ is a observer gain matrix (possibly a
function of $\psi$) yet to be determined.


\subsubsection{Error System}

The time derivative of (\ref{chSMS_ID.eq.AVO_deltaR}) is
%
\begin{align}\label{chSMS_ID.eq.AVO_deltaRdot}
\Delta \dot{R}&=\dot{\hat{R}}^T R + \hat{R}^T \dot{R} \nonumber \\
&= \Delta R \mathcal{J}\left(\Delta \omega +
   \mathcal{A}(\Delta q) L \Delta q\right)
\end{align}
%
thus
%
\begin{align}\label{chSMS_ID.eq.AVO_deltaqdot}
\Delta \dot{q}
 &=\mathcal{A}^{-1}(\Delta q)\left(\Delta \omega +
   \mathcal{A}(\Delta q) L \Delta q\right)          \nonumber \\
 &=L\Delta q+\mathcal{A}^{-1}(\Delta q)\Delta\omega.
\end{align}
%
The time derivative of (\ref{chSMS_ID.eq.AVO_deltaw}) is
%
\begin{align}\label{chSMS_ID.eq.AVO_deltawdot}
\Delta\dot{\omega}
  =&\dot{\omega}-\Delta\dot{R}^T\hat{\omega}-\Delta
    R^T\dot{\hat{\omega}}
                                                        \nonumber  \\
  =&I^{-1}\left(\mathcal{J}\left(I\omega\right)\omega + \tau\right)+\mathcal{J}\left(\Delta \omega 
    +\mathcal{A}(\Delta q)L\Delta q\right)\omegaInBody
                                                        \nonumber  \\
   &-I^{-1}\left(\mathcal{J}\left(I \omegaInBody\right)\omegaInBody+\tau+\Delta q\right)+
    \mathcal{J}(\omegaInBody)\mathcal{A}(\Delta q) L\Delta q
                                                        \nonumber  \\
  =&I^{-1}\left(\mathcal{J}\left(I \omega\right) \omega + 
    I \mathcal{J}\left(\Delta \omega\right)\omegaInBody -\mathcal{J}\left(I\omegaInBody\right)
    \omegaInBody \right)-I^{-1} \Delta q                         \nonumber  \\
  =&-I^{-1} \Delta q + I^{-1}\left(-I\mathcal{J}(\omegaInBody)
    +\mathcal{J}\left(I \omega\right)-\mathcal{J}(\omegaInBody)I
    \right)\Delta\omega,
\end{align} 
%
where this final equality makes use of the fact that
%
\begin{align*}
\mathcal{J}\left(I\omega\right)\omega +I\mathcal{J}\left(\Delta \omega\right)\omegaInBody
-\mathcal{J}\left(I\omegaInBody\right)\omegaInBody&=\mathcal{J}\left(I\omega\right)\left(\Delta \omega +\omegaInBody\right)
-I\mathcal{J}(\omegaInBody)\Delta \omega -\mathcal{J}\left(I \omegaInBody\right)\omegaInBody    \nonumber   \\
   &= \left(\mathcal{J}\left(I\omega\right)-I\mathcal{J}(\omegaInBody)
      - \mathcal{J}(\omegaInBody)I\right)\Delta \omega.
\end{align*}
%
%
Define  
%
\begin{equation} \label{chSMS_ID.eq.skewSymB}
\mathcal{B}\left(\omega,\omegaInBody\right)=
-I\mathcal{J}(\omegaInBody)+ \mathcal{J}\left(I \omega\right)-\mathcal{J}(\omegaInBody)I,
\end{equation}  
%
clearly $\mathcal{B}^T\left(\omega,\omegaInBody\right)=-
\mathcal{B}\left(\omega,\omegaInBody\right)$ and thus
$\mathcal{B}\left(\omega,\omegaInBody\right)$ is
skew-symmetric. Defining 
%
\begin{equation}
x=\left[\begin{array}{c}\Delta q \\ \Delta\omega\end{array}\right]
\end{equation}
%
and
%
\begin{equation}
\mathbb{A}(x,\omega,\omegaInBody)=
\left[ \begin{array}{cc}
     L   & \mathcal{A}^{-1}(\Delta q)                           \\
     -I^{-1}   & I^{-1}\mathcal{B}(\omega,\omegaInBody)                       \\
\end{array} \right],
\end{equation}
%
from (\ref{chSMS_ID.eq.AVO_deltaqdot}) and
(\ref{chSMS_ID.eq.AVO_deltawdot}), the full error system takes the
form
%
\begin{equation}\label{chSMS_ID.eq.AVO_error}
\dot{x} =\mathbb{A}(x,\omega,\omegaInBody)x.
\end{equation}
%
\subsubsection{Stability Analysis}%\label{sec.obs.JHU}

\begin{SO3_AVO}
\label{chSMS_ID.theo.AVO_Theorem}
Consider the error system of the form of
  (\ref{chSMS_ID.eq.AVO_error}) with the following assumptions:
%
\begin{itemize}
\item The plant satisfies the condition $4\lambda_3>\lambda_1$, and
\item  $L(\omegaInBody)$ in the observer is set such that
$L(\omegaInBody)=-E+F(\omegaInBody)$ where $E$ is a constant positive
definite matrix, $F(\omegaInBody)=-I^{1/2}\mathcal{J}(\omegaInBody)I^{-1/2} +
I^{-1/2}\mathcal{J}(I\omegaInBody)I^{-1/2} - I^{-1/2}\mathcal{J}(\omegaInBody)I^{1/2}$
and $\omegaInBody=\Delta R^T\hat{\omega}$. %DEFINE \omegaInBody here?
\end{itemize}
%
This error system is locally asymptotically stable in the sense of
Lyapunov for a neighborhood of $x=\vec{0}$. %in the neighborhood of $\|x\|<\pi$.TM_DISAGREE
\end{SO3_AVO}
%
%The condition on the eigenvalues of $I$ is required to guarantee
%$\forall t\geq t_0 \quad \|x\|<\pi$.

\noindent Proof: 
%
Note that for $\epsilon$ small enough the Lyapunov function
%
\begin{equation}\label{chSMS_ID.eq.AVO_Lyap}
\mathfrak{L}_\epsilon =\frac{1}{2}x^T P_\epsilon x 
\end{equation}
%
is
%
\begin{itemize}
\item  positive definite
%\item radially unbounded
\item equal to zero if and only if $x =\vec{0}$
\end{itemize}
%
where 
%
\begin{equation}
P_\epsilon=
\left[ \begin{array}{cc}
     \mathbb{I}         & -\epsilon I^{1/2}  \\
     -\epsilon I^{1/2}  & I                  \\
\end{array} \right],
\end{equation}
%
$\mathbb{I}$ is the identity matrix, and $I$ is the plant's
inertia tensor.
%  Although the first %two 
%property is true $\forall x$ our local Lyapunov result will only
%require these properties for
%$\|x\|<\pi$.
%TM_DISAGREE we have not proven that $P_\epsilon$ is positive definite
% 
%Since we can assume, without loss of generality, that
%$I$ is diagonal it is easy to see that if $\epsilon$ is small enough
%$P_\epsilon$ is diagonally dominant.  Therefore for all $\epsilon$ under
%some threshold the function $\mathfrak{L}_\epsilon$ is

In the sequel the explicit functional dependencies of $\mathcal{A}$,
$\mathcal{B}$, and $L$ are no longer specified. This compact notation
improves the clarity of the equations.  However, in this and all
future equations: $\mathcal{A}$ denotes $\mathcal{A}(\Delta
q)$, $\mathcal{B}$ denotes $\mathcal{B}(\omegaInBody,\omega)$, and $L$
denotes $L(\omegaInBody)$.  Consider $\dot{\mathfrak{L}}_\epsilon$
given by
%
%\begin{widetext}
\begin{align}\label{chSMS_ID.eq.AVO_Lyapdot}
\dot{\mathfrak{L}}_\epsilon
  =&\frac{1}{2}\dot{x}^T P_\epsilon x +\frac{1}{2} x^T P_\epsilon \dot{x}         \nonumber \\
  =&\frac{1}{2}x^T\left(\mathbb{A}^T P_\epsilon+P_\epsilon \mathbb{A}\right)x      \nonumber \\
  =&\frac{1}{2}x^T\left[ \begin{array}{cc}
    L^T+L+2\epsilon I^{-1/2}   
    &-\epsilon L^{T} I^{1/2}-\mathbb{I}+\mathcal{A}^{-1}-\epsilon I^{-1/2}\mathcal{B} \\
      \mathcal{A}^{-T}+\epsilon\mathcal{B}I^{-1/2} -\epsilon I^{1/2}L -\mathbb{I}  
   & -\epsilon \mathcal{A}^{-T}I^{1/2}+\mathcal{B}^T -\epsilon I^{1/2}\mathcal{A}^{-1}+\mathcal{B} \\
    \end{array} \right]x                                                       \nonumber \\
  =&\frac{1}{2}x^T\left[ \begin{array}{cc}
      -2E+2\epsilon I^{-1/2}   
      &-\epsilon\left(-E I^{1/2}+I^{-1/2}\mathcal{J}\left(I\Delta\omega\right)\right) \\
      -\epsilon\left(-I^{1/2}E-\mathcal{J}\left(I\Delta\omega\right)I^{-1/2}\right)
      & -\epsilon\left(\mathcal{A}^{-T}I^{1/2}+I^{1/2}\mathcal{A}^{-1}\right) \\
    \end{array} \right]x.
\end{align}
%\end{widetext}
%
The equality $L^T+L+2\epsilon I^{-1/2}=-2E+2\epsilon I^{-1/2}$ is a
consequence of $F(\omegaInBody)$ being skew symmetric.  The equality
$-\epsilon \mathcal{A}^{-T}I^{1/2}+\mathcal{B}^T -\epsilon
I^{1/2}\mathcal{A}^{-1}+\mathcal{B}=
-\epsilon\left(\mathcal{A}^{-T}I^{1/2}+I^{1/2}\mathcal{A}^{-1}\right)$
is a consequence of $\mathcal{B}(\omega,\omegaInBody)$ being skew
symmetric. The derivation of (\ref{chSMS_ID.eq.AVO_Lyapdot}) also uses
the facts that $\mathcal{A}^{-1}(\Delta q)\Delta
q=\mathcal{A}^{-T}\Delta q=\Delta q$ implies
%
\begin{equation}
\Delta
q^T\left(-\mathbb{I}+\mathcal{A}^{-1}\right)\Delta \omega=0;
\end{equation} 
%
$-F(\omegaInBody)
I^{1/2}+I^{-1/2}\mathcal{B}(\omegaInBody)=I^{-1/2}\mathcal{J}\left(I\Delta
  \omega\right)$ implies
%
\begin{equation}
 \epsilon\left(L^{T} I^{1/2}+
  I^{-1/2}\mathcal{B}\right) = -\epsilon\left(-E
  I^{1/2}+I^{-1/2}\mathcal{J}\left(I\Delta\omega\right)\right);
\end{equation}
%
and
%
\begin{equation}
L^T(\omegaInBody)=-E-F(\omegaInBody).
\end{equation}


Since $\dot{\mathfrak{L}}_\epsilon$ depends only on $\Delta q$ and
$\Delta \omega$, $\dot{\mathfrak{L}}_\epsilon$ is bounded for all $x$
such that $\|x\|<\pi$.  First consider the term $-\epsilon\Delta
\omega^T\left(\mathcal{A}^{-T}I^{1/2}+
  I^{1/2}\mathcal{A}^{-1}\right)\Delta \omega$.  Noting the
closed-form structure of $\mathcal{A}^{-1}(\Delta q)$ from
(\ref{chModels.eq.Ainv}), the lower diagonal term of
(\ref{chSMS_ID.eq.AVO_Lyapdot}) obeys the following equality
%
\begin{align*}
  \mathcal{A}^{-T}I^{1/2}  +I^{1/2}\mathcal{A}^{-1}=&
2I^{1/2}+\frac{1}{2}(-\mathcal{J}(\Delta
  q)I^{1/2}+I^{1/2}\mathcal{J}(\Delta q))+        \nonumber \\
&\frac{1}{\|\Delta
    q\|^2}\left(1-\frac{\|\Delta q\|}{2}\cot\left(\frac{\|\Delta
        q\|}{2}\right)\right) 
 \left( \mathcal{J}(\Delta q)^2 I^{1/2}+I^{1/2}\mathcal{J}(\Delta
    q)^2 \right).
\end{align*}
%
The inequalities (\ref{chModels.eq.AVO_ineq}) from Section
\ref{chModels.sec.normConventions} and the fact
$\left(1-\frac{\|\Delta q\|}{2}\cot\left(\frac{\|\Delta
      q\|}{2}\right)\right)<1\quad\forall \|\Delta q\|\leq\|x\|<\pi$
imply
%
\begin{align}
-\epsilon \Delta \omega^T \left(\mathcal{A}^{-T}I^{1/2}+ I^{1/2}
\mathcal{A}^{-1} \right)\Delta\omega& \leq 
 \epsilon (\lambda_1^{1/2}\|\Delta
q\|+\lambda_1^{1/2}-2\lambda_3^{1/2})\|\Delta \omega\|^2 
\nonumber \\
\Delta q^T \left(-2E+2\epsilon I^{-1/2}\right)\Delta q 
&\leq-\left(2e_3-\epsilon\lambda_1\right)\|\Delta q\|^2
\nonumber \\
-\epsilon\Delta q^T\left(\left(-E_1I^{1/2}+I^{-1/2}\mathcal{J}
(I\Delta\omega)\right)\right)\Delta
\omega&\leq
\epsilon(e_{11}\lambda_1\|\Delta q\|\|\Delta \omega\|
+\lambda_3^{1/2}\lambda_1\|\Delta q\|\|\Delta \omega\|^2).
\nonumber 
\end{align}
%
In consequence
%
%***********BELOW SHOWS HOW TO SPLIT PARAETHESIS***********
% \begin{align}\label{chSMS_ID.eq.AVO_Lyapdot2}
% 2 \dot{\mathfrak{L}}_\epsilon
%    \leq&-\left(2e_{13}-\epsilon\lambda_1\right)\|\Delta q\|^2+
%         \epsilon \left(\lambda_1^{1/2}\|\Delta q\|+\lambda_1^{1/2}-\right.
%                                                        \nonumber \\
%        &\left. 2\lambda_3^{1/2}\right)\|\Delta \omega\|^2 +
%         2\epsilon(e_{11}\lambda_1\|\Delta q\|\|\Delta\omega\|+
%                                                        \nonumber \\
%        & \lambda_3^{1/2}\lambda_1\|\Delta q\|\|\Delta\omega\|^2)
%                                                        \nonumber \\
%    \leq&-\left(2e_{13}-\epsilon\lambda_1\right)\|\Delta q\|^2+
%         \epsilon (\lambda_1^{1/2}-2\lambda_3^{1/2})\|\Delta \omega\|^2+ 
%                                                        \nonumber \\
%        &\epsilon(2e_{11}\lambda_1+2\lambda_3^{1/2}\lambda_1\pi+
%         \lambda_1^{1/2}\pi)\|\Delta q\|\|\Delta\omega\| 
% \end{align}
%**********************************************************
%
\begin{align}\label{chSMS_ID.eq.AVO_Lyapdot2}
2 \dot{\mathfrak{L}}_\epsilon
   \leq&-\left(2e_{13}-\epsilon\lambda_1\right)\|\Delta q\|^2+
        \epsilon \left(\lambda_1^{1/2}\|\Delta q\|+\lambda_1^{1/2}-
        2\lambda_3^{1/2}\right)\|\Delta \omega\|^2 +
                                                       \nonumber \\
       &2\epsilon(e_{11}\lambda_1\|\Delta q\|\|\Delta\omega\|+
        \lambda_3^{1/2}\lambda_1\|\Delta q\|\|\Delta\omega\|^2)
                                                       \nonumber \\
   \leq&-\left(2e_{13}-\epsilon\lambda_1\right)\|\Delta q\|^2+
        \epsilon (\lambda_1^{1/2}-2\lambda_3^{1/2})\|\Delta \omega\|^2+ 
                                                       \nonumber \\
       &\epsilon(2e_{11}\lambda_1+2\lambda_3^{1/2}\lambda_1\pi+
        \lambda_1^{1/2}\pi)\|\Delta q\|\|\Delta\omega\| 
\end{align}
%
$\forall \|\Delta\omega\|\leq\|x\|<\pi$.   
%TM_DISAGREE where the final inequality relies on $\|\Delta\omega\|\leq\|x\|<\pi$.  
Define
$a_1=2e_{11}\lambda_1+2\lambda_3^{1/2}\lambda_1\pi+\lambda_1^{1/2}\pi$,
$a_2=2e_{13}$, $a_3=\lambda_1$,
$a_4=-\lambda_1^{1/2}+2\lambda_3^{1/2}$, and observe that 
%
\begin{equation}
2 \dot{\mathfrak{L}}_\epsilon\leq -z^T Q z
\end{equation}
%
where
%
\begin{equation}
 Q=\left[ \begin{array}{cc}
a_2-\epsilon a_3  &  -\frac{\epsilon}{2}a_1  \\
-\frac{\epsilon}{2}a_1 & \epsilon a_4\\
\end{array} \right] 
\end{equation}
%
and $z=\left[ \begin{array}{c}
    \|\Delta q\| \\    \|\Delta \omega \| \\
  \end{array} \right]$.
$Q$ is negative definite if $\lambda_1<4\lambda_3$  and 
$\epsilon<\frac{4 a_4 a_2}{\| 4 a_4 a_3 -a_1^2\|}$.

The local asymptotic stability of the angular velocity observer is
proven.  Note that accounting for the presence of the matrix valued
function (\ref{chSMS_ID.eq.skewSymB}) in the error dynamics
(\ref{chSMS_ID.eq.AVO_error}) as part of the stability analysis
required careful selection of the Lyapunov function
(\ref{chSMS_ID.eq.AVO_Lyap}) and matrix valued function
$F(\omegaInBody)$ from Theorem
\ref{chSMS_ID.theo.AVO_Theorem}. Although Corollary
\ref{chModels.theo.PDS3_Factorization} was not explicitly required for
the stability analysis reported, the selections of $\mathfrak{L}(t)$
and $F(\omegaInBody)$ were enabled by this Corollary.


% define

% \begin{equation}
% Q=\left[ \begin{array}{cc}
%       a_2-\epsilon a_3  &  -\frac{\epsilon}{2}a_1  \\
%       -\frac{\epsilon}{2}a_1 & \epsilon a_4\\
%     \end{array} \right]
% \end{equation}

% \noindent  then note

% \begin{equation}
% 2 \dot{\mathfrak{L}}_\epsilon\leq-\left[\|\Delta q\| \|\Delta\omega\|\right]
% Q
% \left[\begin{array}{c} \|\Delta q\| \\ \|\Delta\omega\|\end{array}\right].
% \end{equation}

% \noindent Lets look at the eigenvalues 

% \begin{align*}
% 0=&det\left(Q-\mu \mathbb{I}\right)        \nonumber     \\
%  =&\mu^2-(a_2-\epsilon a_3+\epsilon a_4)\mu + \epsilon(a_2-\epsilon a_3)a_4-\frac{\epsilon^2}{4}a_1^2 
% \end{align*}

% \noindent which leads to the eigenvalue function

% \begin{align*}
% \mu=&\frac{1}{2}(a_2-\epsilon a_3+\epsilon a_4) \pm  \nonumber  \\ 
%     &\frac{1}{2}\sqrt{(a_2-\epsilon a_3+\epsilon a_4)^2
%     -4\left(\epsilon(a_2-\epsilon a_3)a_4
%   -\frac{\epsilon^2}{4}a_1^2\right)}.
% \end{align*}

% \nonumber As long as $\epsilon$ is small enough this 
% implies  $(a_2-\epsilon a_3+\epsilon a_4)>0$.  Therefore the
% eigenvalues of $Q$ will always be positive as long as 

% \begin{align*}
%   &\frac{1}{2}(a_2-\epsilon a_3+\epsilon a_4) >  \nonumber  \\
%   &\frac{1}{2}\sqrt{(a_2-\epsilon a_3+\epsilon a_4)^2
%     -4\left(\epsilon(a_2-\epsilon a_3)a_4
%       -\frac{\epsilon^2}{4}a_1^2\right)}
% \end{align*}

% \noindent which is equivalent to 

% \begin{equation}
% -4 a_4 a_2 +\epsilon \left(4 a_4 a_3 -a_1^2\right)<0.
% \end{equation}

% \noindent Note that if $\lambda_1<4\lambda_3$  and $\epsilon$ is small
% enough then the inequality holds.  This completes the proof.