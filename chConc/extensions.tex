%extensions.tex
\section{Future Work}


In addition to the straightforward applications of \ac{UV} \ac{AID}
and \ac{UV} \ac{AMBC} discussed in Chapter \ref{chIntro}, these
algorithms could also enable complex, multifaceted \ac{UV} missions
%currently being considered by oceangraphic scientists and engineers 
by using \ac{AID} for fault detection and \ac{AMBC} for fault
compensation.


As shown in Chapter \ref{chUV_AID}, \ac{UV} \ac{AID} dynamically
estimates parameters assumed to be constant.  
%
%uses the difference
%between anticipated and measured \ac{UV} performance to identify plant
%parameters.
%
When applying \ac{UV} \ac{AID} for fault detection, 
%the disparity between the \ac{UV}'s anticipated and measured dynamic response still
%drives 
parameter adaptation is
% however, for fault detection those parameter estimates are 
monitored for changes indicative of \ac{UV} component failures.
%
Different parameters can be monitored for different types of failures.
%
For example, drag parameters could be used to detect entanglement,
mass parameters could be used to detect flooded housings or detached
components, and a general force/torque vector could be used to detect
unanticipated \ac{UV} collisions. 

 
As shown in Chapter \ref{chUV_AMBC}, \ac{AMBC} uses plant parameter
estimates for model-based trajectory-tracking and these plant
parameter estimates are iteratively improved in a process similar to
\ac{AID}.
%
Since \ac{AMBC} algorithms evolve parameter estimates in a process
similar to \ac{AID}, they are naturally robust to \ac{UV} component
failures.
% 
Designing a suite of \ac{AMBC} algorithms which compensate for
particular component failures and using a collection of \ac{AID}
algorithms to switch between these \ac{AMBC} algorithms could allow
fast, effective fault compensation.


Current \ac{UV} control systems rely on engineers to detect remotely and
compensate for vehicle component failures.
%
As \ac{UV} mission complexity has increased, this component failure
detection and compensation method has become a barrier limiting future
deployments.
%
Using \ac{AID} and \ac{AMBC} to automate failure detection and failure
compensation has the potential to {\it i}) lower the amount of time
lost due to mission aborts, {\it ii}) limit the possibility of losing
a vehicle, and {\it iii}) enable new missions which are currently
impossible due to vehicle safety concerns.  Missions benefiting could
include:
% 
\begin{itemize}
\item{\it Long Duration Unsupervised Deployments:} Some \ac{UV}s
are capable of deployments lasting from days to
weeks\cite{bellinghamTethys,kaminskiAUV2010}. The capacity to detect
and compensate for \ac{UV} component failures increases the likelihood
of successful unsupervised deployments.
%
\item{\it Semi-Autonomous \ac{UV} Operations:} Semi-autonomous
  missions use human direction and data interpretation to accomplish
  actions too complex to automate.  Automated failure detection will
  allow engineers to understand the vehicle's current state and
  predict its future capabilities.
% 
\item{\it Under Ice Operations:} The \ac{PROV} 
is currently being developed for under ice
operation\cite{PROV1,PROV2}, where surfacing in the event of a failure
is not an option.  \ac{PROV}'s thruster redundancy will allow continued
operation with multiple thruster failures.  However, utilizing this
redundancy requires detecting the failures. 
\end{itemize}


In short, this Thesis reports adaptive algorithms which enable better
utilization of current \acp{UV}.  These results are also the starting
point for algorithms which enable
%increase \ac{UV} capacity to accomplish 
complex, multifaceted missions of both the current generation of
\acp{UV} and those to come.

