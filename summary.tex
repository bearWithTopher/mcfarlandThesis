\documentclass{article}
\usepackage{amsmath}
\usepackage{amsfonts}
\usepackage{mathrsfs}
\DeclareMathOperator{\SE3}{SE(3)}
\DeclareMathOperator{\SO3}{SO(3)}
\DeclareMathOperator{\se3}{se(3)}
\DeclareMathOperator{\so3}{so(3)}
\DeclareMathOperator{\id}{id}
\DeclareMathOperator{\Ad}{Ad}
\DeclareMathOperator{\ad}{ad}
%\DeclareMathOperator{\exp}{exp}
\usepackage[pdftex]{color,graphicx}

\newcommand{\gbreg}{\mathbb{W}_{gb}}

\begin{document} 


\title{ Adaptive Identification and Control for Unmanned Underwater Vehicles:\\
Theory and Comparative Experimental Evaluations}

\author{Christopher J. McFarland}

\maketitle

This Thesis reports several novel algorithms for state observation,
parameter identification, and control of second-order plants.  The
primary applications considered are underwater vehicle adaptive
identification and adaptive tracking control. Previous adaptive model
identification methods have focused on model-based adaptive tracking
controllers; however, these approaches are not applicable when the
plant is either uncontrolled, under open-loop control, or using any
control law other than a specific adaptive tracking controller.  The
underwater vehicle adaptive identifier reported herein does not
require reference trajectory-tracking, linear acceleration, or angular
acceleration.  Thus, these results are applicable in the commonly
occurring cases of uncontrolled vehicles, vehicles under open-loop
control, vehicles using control methods prescribed to meet other
application-specific considerations, and vehicles not instrumented to
measure angular acceleration. The final contribution of this Thesis is
the development of a novel fully-coupled underwater vehicle adaptive
model-based controller and a comparative experimental evaluation of
adaptive model-based control and proportional derivative control for
simultaneous underwater vehicle motion in 6 degrees-of-freedom. A
stability proof for each new result is included.

The standard models for underwater vehicle dynamic operation evolve on the set of
rigid-body transformations, SE(3).  Thus, the first third of this
Thesis is focused on utilizing the group structure of SO(3) and SE(3)
in the development of one state observer and two adaptive identifiers
for simple mechanical systems. These results are the theoretical
antecedents of the underwater vehicle results presented later in the text.  The plant
of the angular velocity observer is a three degree-of-freedom
second-order rigid-body rotating under the influence of an externally
applied torque.  Numerical simulations of the novel angular velocity
observer and two previously reported observers are included.  The
simulation results show similar performance of all three observers for
most angular position trajectories and inertia tensors.  A novel
adaptive identification algorithm is reported for a three
degree-of-freedom second-order rigid-body rotating under the influence
of an externally applied torque, along with a simulation study. An
open kinematic-chain adaptive identifier is also reported.

The underwater vehicle adaptive identifiers presented herein estimate the
hydrodynamic mass, quadratic drag, gravitational force, and buoyancy
parameters of a second-order rigid-body plant under the influence of
actuator forces and torques.  An experimental comparison of adaptive
identification and conventional least squares parameter identification
is reported. The experimental results corroborate the analytic
stability analysis, showing that the adaptively estimated plant
parameters converge to values that provide plant-model input-output
behavior closely approximating the input-output behavior of the actual
experimental underwater vehicle. The adaptively identified model was
shown to be similar to the least squares identified model in its
ability to match the actual vehicle's input-output characteristics.

The underwater vehicle model-based adaptive tracking controller presented herein
estimates the hydrodynamic mass, quadratic drag, gravitational force,
and buoyancy parameters of a fully-coupled second-order rigid-body
plant while providing asymptotically exact trajectory-tracking.

A comparative experimental evaluation of proportional derivative and
adaptive model-based control for underwater vehicles is also reported.
%
To the best of the authors' knowledge, this is the first such
evaluation of model-based adaptive tracking control for underwater
vehicles during simultaneous dynamic motion in all 6 degrees-of-freedom.
%
This experimental evaluation revealed the presence of unmodeled
thruster dynamics arising during reversals of the vehicle's thrusters,
and that the unmodeled thruster dynamics
%these short duration, small magnitude unmodeled deviations from the
%  commanded input 
can destabilize parameter adaptation.
%
The three major contributions of this experimental evaluation are: an
experimental analysis of how unmodeled thruster dynamics can
destabilize parameter adaptation, a two-step adaptive model-based
control algorithm which is robust to the thruster modeling errors
present, and a comparative experimental evaluation of adaptive
model-based control and proportional derivative control for
fully-actuated underwater vehicles preforming simultaneous 6
degree-of-freedom trajectory-tracking.


\end{document}