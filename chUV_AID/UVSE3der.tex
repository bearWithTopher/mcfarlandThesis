\section{6-\acs{DOF} \acs{UV} \acs{AID}}
\label{chUV_AID.sec.UV_SE3_AID}

This Section reports a novel nonlinear \ac{AID} algorithm for
6-\ac{DOF} \ac{UV} plant models of the form
(\ref{chModels.eq.UVSE3plant}).
%
A local stability analysis is also included. 
%
This proof of Theorem \ref{chUV_AID.theo.UV_SE3_AID} is provided in
two parts.
%
First, in Section \ref{chUV_AID.sec.UVSO3_AID_errSys}, the error system
is developed.
%
Then, in Section \ref{chUV_AID.sec.UVSO3_AID_convProof}, we prove the result.  
%\subsection{Adaptive Identification of a UV subject to Drag, Gravity, 
%  Buoyancy, External Forces, and Torques}\label{sec.ROVAID}



\begin{UV_SE3_AID}
\label{chUV_AID.theo.UV_SE3_AID}


Consider the following \ac{AID} algorithm for plants of the form
(\ref{chModels.eq.UVSE3plant}):
%
\begin{align}
  \dot{\hat{v}}=&\hat{M}^{-1}\left(\ad_v^T(\hat{M}v)
  +\left(\sum_{i=1}^6 |v_i|\hat{D}_i\right) v + 
  \hat{\mathcal{G}}(R)+u\right) - a \Delta v   \label{chUV_AID.eq.UV_SE3_Estimator}\\
  \dot{\hat{M}}=&\frac{\gamma_1}{2}\left(-\psi_1 v^T 
    -v \psi_1^T+\Delta v \psi_2^T
  +\psi_2 \Delta v^T\right)\label{chUV_AID.eq.UV_SE3_MassId}\\
  \dot{\hat{D_i}}=&-\gamma_2|v_i|\Delta v v^T
  \label{chUV_AID.eq.UV_SE3_DragId}\\
   \dot{\hat{g}}=&-\gamma_3 \Delta \nu^T R^T e_3
  \label{chUV_AID.eq.UV_SE3_GravId} \\
   \dot{\hat{b}}=&\gamma_4\mathcal{J}(\Delta \omega) R^T e_3
  \label{chUV_AID.eq.UV_SE3_BuoyId} 
\end{align}
%
\noindent with the following definitions:
%
\begin{itemize}
\item $a,\gamma_1,\gamma_2,\gamma_3,\gamma_4\in\mathbb{R}_+$
\item $\hat{\mathcal{G}}(R)= \left[ \begin{array}{c}\hat{g} R^T e_3   
    \\ \mathcal{J}(\hat{b})R^T e_3 \\ \end{array}\right]$
\item $\psi_1=\ad_v(\Delta v)$
%\item $\psi_2=\dot{\hat{v}}+a \Delta v$
\item $\psi_2$ such that
\begin{align}
\psi_2=&\dot{\hat{v}}+a \Delta v \nonumber \\
       = &\hat{M}^{-1}\left(\ad_v^T(\hat{M}v)
  +\left(\sum_{i=1}^6 |v_i|\hat{D}_i\right) v + 
  \hat{\mathcal{G}}(R)+u\right)
%\nonumber
\end{align}
\end{itemize}
%
\noindent and the following assumptions:
%
\begin{itemize}
\item $u(t)$ is bounded by assumption
\item $\hat{M}(t_0)$ is \ac{SPD}
\item $\hat{v}(t_0)=v(t_0)$
\item $\exists \epsilon \in \mathbb{R}_+$ such that 
$\mathcal{T}(t_0)^{1/2}+\epsilon\leq \lambda_6$ where
\begin{equation} 
  \mathcal{T}(t_0)=\|\Delta M(t_0)\|_F^2 +
                    \frac{\gamma_1}{\gamma_2}\sum_{i=1}^6 \|\Delta D_i(t_0)\|_F^2 + 
                    \frac{\gamma_1}{\gamma_3}\Delta g(t_0)^2
                     + \frac{\gamma_1}{\gamma_4}\|\Delta b(t_0)\|^2.
\end{equation}
\end{itemize}
%
\noindent Under these conditions the estimated angular and linear
velocities will be asymptotically stable in the sense of Lyapunov,
i.e., $\lim_{t\to \infty}\Delta v=\vec{0}$, and the parameter
estimates will converge to constant values, i.e., $\lim_{t\to
  \infty}\dot{\hat{M}}(t)=0_{6\times 6}$, $\lim_{t\to
  \infty}\dot{\hat{D}}_i(t)=0_{6\times 6}$, $\lim_{t\to
  \infty}\dot{\hat{g}}(t)=0$, and $\lim_{t\to
  \infty}\dot{\hat{b}}(t)=\vec{0}$. 
\end{UV_SE3_AID}

This local stability result implies that the plant parameter estimates
converge to values that provide plant model input-output behavior
identical to that of the actual experimental plant for the given input
force and torque signals.  Note that since the initial mass matrix
estimate is \ac{SPD} and the parameter estimate law is symmetric, both
$\hat{M}(t)$ and $\Delta M(t)$ will be symmetric, and thus they will
have strictly real eigenvalues.

\subsection{\ac{UV} \ac{AID} Error System}\label{chUV_AID.sec.UV_SE3_error}
\label{chUV_AID.sec.UVSE3_AID_errSys}


%From the update laws (\ref{eq.rovMassId})-(\ref{eq.rovRightMomId}), the time
%derivatives of (\ref{eq.deltaM})-(\ref{eq.deltabSE3}) are
%
%\begin{align}
%\dot{\Delta M}&=\dot{\hat{M}} \nonumber \\
%              &=\frac{\gamma_1}{2}\left(\psi_1 v^T +v \psi_1^T 
%               +\Delta v \psi_2^T+\psi_2 \Delta v^T\right)
%\label{eq.deltaMdot} \\
%\dot{\Delta D_i}&=\dot{\hat{D}}_i \nonumber \\
%                &=-\gamma_2|v_i|\Delta v v^T
%\label{eq.deltaDdot} \\
%\dot{\Delta g}&=\dot{\hat{g}}  \nonumber \\
%              &=-\gamma_3 \Delta \nu^T R^T e_3
%\label{eq.deltagdot} \\
%\dot{\Delta b}&=\dot{\hat{b}}  \nonumber \\
%           &=\gamma_4\mathcal{J}(\Delta \omega) R^T e_3.
%\label{eq.deltabSE3dot} 
%\end{align}
%TM_RUNNING_OUT_OF_SPACE you could get rid of this and refer directly to 
% the ID laws at the final step to get \dot{V}(t) and cut this out...

Note the actual plant parameters are constant, thus
$\Delta \dot{M}=\dot{\hat{M}}$, $\Delta \dot{D}_i=\dot{\hat{D}}_i$,
$\Delta \dot{g}=\dot{\hat{g}}$, and $\Delta \dot{b}=\dot{\hat{b}}$.  
%\noindent 
Using $ M \hat{M}^{-1}= \mathbb{I} -\Delta M \hat{M}^{-1}$ (where
$\mathbb{I}$ is the identity matrix) and $\Delta\mathcal{G}(R)=
\left[ \begin{array}{c}\Delta g R^T e_3 \\ \mathcal{J}(\Delta b)R^T
    e_3 \\ \end{array}\right]$ the expression for velocity error
dynamics becomes
%We make use of $\mathcal{H}(v_1)v_2=-\ad(v_2)v_1$,
%$\psi_2=\hat{M}^{-1}\left( \mathcal{H}(\hat{M}v)v +\left(\sum_{i=1}^6
%    |v_i|\hat{D}_i\right) v + \hat{\mathcal{G}}(R)+u\right)$, and
% in developing the following
%velocity error dynamics expression:

\begin{align}\label{chUV_AID.eq.UV_SE3_Mdeltavdot}
M\Delta \dot{v}
 =&M\left(\dot{\hat{v}} - \dot{v}\right)
\nonumber \\
 =&-a M \Delta v
%\nonumber \\
  +M \hat{M}^{-1}\left( \ad_v^T(\hat{M}v)
  +\left(\sum_{i=1}^6 |v_i|\hat{D}_i\right) v + 
  \hat{\mathcal{G}}(R)+u\right)  
\nonumber \\
  &-\left(\ad_v^T(M v)+\left(\sum_{i=1}^6 |v_i|D_i
    \right)v+\mathcal{G}(R)+u\right)
\nonumber \\
=&-a M \Delta v   
%\nonumber \\
 +\left( \ad_v^T(\hat{M}v)
  +\left(\sum_{i=1}^6 |v_i|\hat{D}_i\right) v + 
  \hat{\mathcal{G}}(R)+u\right)
  -\Delta M \psi_2  
\nonumber \\
  &-\left(\ad_v^T(M v)+\left(\sum_{i=1}^6 |v_i|D_i
    \right)v+\mathcal{G}(R)+u\right)
\nonumber \\ 
=&-a M \Delta v-\Delta M \psi_2+\ad_v^T(\Delta M v)
%\nonumber \\
 +\left(\sum_{i=1}^6 |v_i|\Delta D_i\right) v + 
  \Delta \mathcal{G} (R)
\end{align}


\subsection{\ac{UV} \ac{AID} Convergence Proof}\label{chUV_AID.sec.UV_SE3_proof}
\label{chUV_AID.sec.UVSE3_AID_convProof}

Consider the following Lyapunov function candidate
\begin{equation}
\label{chUV_AID.eq.UV_SE3_lyap}
V(t)=\frac{1}{2}\Delta v^{T} M \Delta v  
      +\frac{1}{2\gamma_1}\tr\left(\Delta M \Delta M^{T}\right) %\label{eq.lyap} \\
     +\frac{1}{2\gamma_2}\sum_{i=1}^6\tr\left(\Delta D_i \Delta D_i^{T}\right)
     +\frac{1}{2\gamma_3}\Delta g^2    
     +\frac{1}{2\gamma_4}\Delta b^T \Delta b 
%\nonumber
\end{equation}
%
\noindent $V(t)$ is
%
\begin{itemize}
\item positive definite
%\item radially unbounded
\item equal to zero if and only if $\Delta v=\vec{0}$, $\Delta
  M=0_{6\times 6}$,$\forall i$ $\Delta D_i=0_{6\times 6}$, $\Delta
  g=0$, and $\Delta b=\vec{0}$.
\end{itemize}
%
\noindent From (\ref{chUV_AID.eq.UV_SE3_Mdeltavdot}) and the facts
that for any matrices $A$ and $B$ of appropriate dimension
$\tr(AB)=\tr(BA)$ and for any $x_1,x_2,x_3\in\mathbb{R}^3 \quad
x_1^T\mathcal{J}(x_2)x_3=-x_2^T\mathcal{J}(x_1)x_3$, the time
derivative of (\ref{chUV_AID.eq.UV_SE3_lyap}) is

\begin{align}
\dot{V}(t)
 =&\frac{1}{2}\left(\Delta \dot{v}^T M \Delta v
   +\Delta v^T M \Delta\dot{ v}\right)  
   +\frac{1}{\gamma_1}\tr\left(\Delta M \Delta\dot{ M}^{T}\right)
  \nonumber \\
  &+\frac{1}{\gamma_2}\sum_{i=1}^6\tr\left(\Delta D_i \Delta\dot{ D}_i^{T}\right)
   +\frac{1}{\gamma_3}\Delta g \Delta\dot{ g}
   +\frac{1}{\gamma_4}\Delta b^T \Delta\dot{ b}
  \nonumber  \\
 =&-a\Delta v^T M \Delta v
   +\frac{1}{2}\left(\ad_v^T(\Delta M v) -\Delta M \psi_2\right)^T \Delta v
  %\nonumber  \\
  +\frac{1}{2}\Delta v^T \left(\ad_v^T(\Delta M v) -\Delta M \psi_2\right)
  \nonumber  \\
  &+\frac{1}{\gamma_1}\tr\left(\Delta M \Delta\dot{ M}^{T}\right) 
  %\nonumber  \\
  +\Delta v^T \left(\sum_{i=1}^6 |v_i|\Delta D_i\right) v 
   +\frac{1}{\gamma_2}\sum_{i=1}^6\tr\left(\Delta D_i \Delta\dot{ D}_i^{T}\right)
  \nonumber  \\
  &+\Delta v^T\Delta \mathcal{G} (R) +\frac{1}{\gamma_3}\Delta g \Delta\dot{ g}
   +\frac{1}{\gamma_4}\Delta b^T \Delta\dot{ b}
  \nonumber  \\
 =&-a\Delta v^T M \Delta v
   +\frac{1}{2}\left(\psi_1^T \Delta M v -\Delta v^T \Delta M \psi_2\right)
  %\nonumber  \\
  +\frac{1}{2}\left(v^T \Delta M \psi_1 -\psi_2^T \Delta M \Delta v\right) 
  \nonumber  \\
  &+\frac{1}{\gamma_1}\tr\left(\Delta M \Delta\dot{ M}^{T}\right)
   +\frac{1}{\gamma_2}\sum_{i=1}^6\tr\left(\Delta D_i \Delta\dot{ D}_i^{T}\right)
  +\sum_{i=1}^6\left(|v_i|\Delta v^T \Delta D_i v\right)
  \nonumber \\
  & +\frac{1}{\gamma_3}\Delta g \Delta\dot{ g}
   +\frac{1}{\gamma_4}\Delta b^T \Delta\dot{ b}
  %\nonumber  \\
  +\left[ \begin{array}{cc}\Delta \nu^T \Delta \omega^T \\ \end{array}\right]
    \left[ \begin{array}{c}\Delta g R^T e_3   
    \\ \mathcal{J}(\Delta b)R^T e_3 \\ \end{array}\right]
  \nonumber \\
 =&-a\Delta v^T M \Delta v
    +\frac{1}{\gamma_1}\tr\left(\Delta M \Delta\dot{ M}^{T}\right)   
 % \nonumber  \\
  +\frac{1}{2}\tr\left(\Delta M \left(\psi_1v^T  -\Delta v\psi_2^T 
     +v\psi_1^T -\psi_2\Delta v^T \right)\right) 
  \nonumber  \\
  &+\frac{1}{\gamma_2}\sum_{i=1}^6\tr\left(\Delta D_i \Delta\dot{ D}_i^{T}\right)
   +\sum_{i=1}^6 \tr\left(|v_i|\Delta D_i v\Delta v^T \right) 
  %\nonumber  \\
  +\frac{1}{\gamma_3}\Delta g \Delta\dot{ g}
   +\Delta \nu^T \Delta g R^T e_3
  \nonumber \\
  &+\frac{1}{\gamma_4}\Delta b^T \Delta\dot{ b}
   +\Delta \omega^T\mathcal{J}(\Delta b)R^T e_3
  \nonumber \\
 =&-a\Delta v^T M \Delta v
    +\frac{1}{\gamma_1}\tr\left(\Delta M \Delta\dot{ M}^{T}\right)   
  %\nonumber  \\
  +\frac{1}{2}\tr\left(\Delta M \left(\psi_1v^T  -\Delta v\psi_2^T 
     +v\psi_1^T -\psi_2\Delta v^T \right)^T\right) 
  \nonumber  \\
  &+\frac{1}{\gamma_2}\sum_{i=1}^6\tr\left(\Delta D_i \Delta\dot{ D}_i^{T}\right) 
  %\nonumber  \\
  +\sum_{i=1}^6 \tr\left(\Delta D_i \left(|v_i|\Delta v v^T\right)^T\right)
   +\frac{1}{\gamma_3}\Delta g \Delta\dot{ g}
  \nonumber \\
  &+\Delta \nu^T \Delta g R^T e_3
   +\frac{1}{\gamma_4}\Delta b^T \Delta\dot{ b}
   -\Delta b^T\mathcal{J}(\Delta \omega)R^T e_3
\label{chUV_AID.eq.UV_SE3_lyapDotLong}
\end{align}
%TM_IF_RUNNING_OUT_OF_SPACE  You could comment out the second-to-last step above

%Using (\ref{eq.deltaMdot})-(\ref{eq.deltabSE3dot}) results in
The actual parameters are constant; thus substituting 
(\ref{chUV_AID.eq.UV_SE3_MassId})-(\ref{chUV_AID.eq.UV_SE3_BuoyId}) into
(\ref{chUV_AID.eq.UV_SE3_lyapDotLong}) yields
%
\begin{equation}\label{chUV_AID.eq.UV_SE3_lyapDot}
\dot{V}(t)=-a\Delta v^T M \Delta v,
\end{equation}
%
\noindent which is negative definite in $\Delta v$ and negative
semi-definite in the error coordinates $\Delta v$, $\Delta M$, $\Delta
D_i$, $\Delta g$, and $\Delta b$.  By Lyapunov's direct method,
(\ref{chUV_AID.eq.UV_SE3_lyap}) and (\ref{chUV_AID.eq.UV_SE3_lyapDot})
imply that all error coordinates are bounded and stable.  The
structure of $\dot{V}(t)$ implies that $\Delta v \in \mathcal{L}_2$
or, equivalently, $\lim_{t\to\infty}\left( \int_{t_0}^t\Delta v^T
  \Delta v\right)^{1/2}<\infty$.  We must ensure every signal in
(\ref{chUV_AID.eq.UV_SE3_Estimator})-(\ref{chUV_AID.eq.UV_SE3_BuoyId})
is bounded.  With $v$, $u$, $\Delta v$, $\Delta M$, $\Delta D_i$,
$\Delta g$, and $\Delta b$ bounded and $M$, $D_i$, $g$, and $b$
constant, and given
(\ref{chUV_AID.eq.deltav})-(\ref{chUV_AID.eq.deltabSE3}) we can
conclude that $\hat{v}$, $\hat{M}$, $\hat{D}_i$, $\hat{g}$, and
$\hat{b}$ are bounded. All that remains is showing that $\hat{M}^{-1}$
is bounded.
%
Note for $\forall t>t_0$ the following hold:
\begin{itemize}
\item 
$\frac{1}{\gamma_2}\sum_{i=1}^6\tr\left(\Delta D_i(t) \Delta
  D_i^T(t)\right)\geq 0$
\item 
$\frac{1}{\gamma_3}\Delta g(t)^2\geq 0$
\item 
$\frac{1}{\gamma_4}\Delta b^T(t) \Delta b(t) \geq 0$
\item 
$0 \leq V(t) \leq V(t_0)$
\item 
$\Delta v(t_0)=\vec{0}$ 
\item
$ \Delta v^T(t) M \Delta v (t)\geq 0$
\item 
$\frac{1}{\gamma_1}\tr\left(\Delta M(t) \Delta M^T(t)\right)=
\frac{1}{\gamma_1}\sum_{i=1}^6|\Delta\lambda_i(t)|^2=\frac{1}{\gamma_1}\|\Delta
M(t)\|_F^2$
\end{itemize}
%
These facts can be used to show
%
\begin{align}
\frac{1}{\gamma_1}\|\Delta M(t)\|_F^2
  \leq&\frac{1}{\gamma_1}\tr\left(\Delta M(t) \Delta M^T(t)\right)
\nonumber \\
  \leq&\frac{1}{\gamma_1}\tr\left(\Delta M(t) \Delta M^T(t)\right) + \Delta v^T(t) M \Delta v (t)
%\nonumber \\
       +\frac{1}{\gamma_2}\sum_{i=1}^6\tr\left(\Delta D_i(t) \Delta D_i^{T}(t)\right)
\nonumber \\
      & +\frac{1}{\gamma_3}\Delta g(t)^2    
       +\frac{1}{\gamma_4}\Delta b^T(t) \Delta b(t) 
\nonumber \\
   \leq&2 V(t)
\nonumber \\
   \leq&2 V(t_0)
\nonumber \\
   \leq&\frac{1}{\gamma_1}\|\Delta M(t_0)\|_F^2 +
        \frac{1}{\gamma_2}\sum_{i=1}^6 \|\Delta D_i(t_0)\|_F^2 + 
%\nonumber \\
       \frac{1}{\gamma_3}\Delta g(t_0)^2+ \frac{1}{\gamma_4}\|\Delta b(t_0)\|^2,
%\nonumber
\end{align}
%
or equivalently, 
%
\begin{equation}
\|\Delta M(t)\|_F\leq\mathcal{T}(t_0)^{1/2}.
\end{equation}
%
By the Rayleigh-Ritz Theorem, $\min_{\|x\|=1}
x^T \hat{M}(t) x=\hat{\lambda}_6(t)$.  Additionally since
$\hat{M}(t)=M+\Delta M(t)$ and by assumption
$\mathcal{T}(t_0)^{1/2}+\epsilon\leq \lambda_6$, the following
inequalities hold:
%
\begin{align}
\hat{\lambda}_6(t)
 &= \min_{\|x\|=1}\left(x^T M x + x^T \Delta M(t) x\right)
\nonumber \\
 &\geq\min_{\|x\|=1}\left(x^T M x\right)-\max_{\|x\|=1}|x^T\Delta M(t)x|
\nonumber \\
 &\geq\lambda_6-\|\Delta M(t)\|_2
\nonumber \\
 &\geq\lambda_6-\left(\sum_{i=1}^6|\Delta\lambda_i(t)|^2\right)^{1/2}
\nonumber \\
 &\geq\lambda_6-\|\Delta M(t)\|_F
\nonumber \\
 &\geq\lambda_6-\mathcal{T}(t_0)^{1/2}
\nonumber \\
 &\geq\lambda_6-\left(\lambda_6-\epsilon\right)
\nonumber \\
 &\geq\epsilon
\end{align}
%
where $\epsilon$ is a finite positive scalar and we use the fact that
$\|\Delta M(t)\|_2^2=\max(|\Delta \lambda_1(t)|^2,|\Delta
\lambda_6(t)|^2)$ because the singular values of a symmetric matrix
are equal to the absolute values of its eigenvalues.
%
The above inequality guarantees that all eigenvalues of $\hat{M}(t)$
are positive and bounded away from zero for all time. 
%
The fact that all signals in (\ref{chModels.eq.UVSE3plant}) and
(\ref{chUV_AID.eq.UV_SE3_Estimator}) are bounded implies that $\dot{v}$
and $\dot{\hat{v}}$ are bounded.  
%
These bounded signals imply that $\Delta \dot{v}$ is bounded.  
%
Note that bounded $\Delta \dot{v}$ and $\Delta v \in \mathcal{L}_2$
implies% $\lim_{t\to \infty}\Delta v=\vec{0}$.
%
\begin{equation}
\lim_{t\to \infty}\Delta v=\vec{0}.
\end{equation}
%
%\noindent 
Since every signal in the parameter update equations
(\ref{chUV_AID.eq.UV_SE3_MassId})-(\ref{chUV_AID.eq.UV_SE3_BuoyId})
are bounded and $\lim_{t\to \infty}\Delta v=\vec{0}$ this implies
%$\lim_{t\to \infty}\dot{\hat{M}}=0_{6 \times 6}$, $\lim_{t\to
%  \infty}\dot{\hat{D}}_i=0_{6 \times 6}$, $\lim_{t\to
%  \infty}\dot{\hat{g}}=0$, $\lim_{t\to \infty}\dot{\hat{g}}=0$, and
%  $\lim_{t\to \infty}\dot{\hat{b}}=\vec{0}$.
%
\begin{align}
\lim_{t\to \infty}&\dot{\hat{M}}=0_{6 \times 6},
\nonumber \\
\lim_{t\to \infty}&\dot{\hat{D}}_i=0_{6 \times 6}\; \; \forall i\in\left\{1,\cdots,6\right\},
\nonumber \\
\lim_{t\to \infty}&\dot{\hat{g}}=0,\text{ and}
\nonumber \\
\lim_{t\to \infty}&\dot{\hat{b}}=\vec{0}.
%\nonumber
\end{align}
%\noindent 
The estimator's angular and linear velocities
asymptotically converge to the velocities of the actual plant, and
the estimated parameters converge to constant values.
%
Thus, the local stability of the reported \ac{AID} algorithm for
6-\ac{DOF} \ac{UV} dynamics is proven.
