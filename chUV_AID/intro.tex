
This Chapter addresses the problem of estimation of plant parameters
for 6-\ac{DOF} rigid-body \acp{UV}.
%
We report two novel \ac{AID} algorithms.  
% 
Each algorithm estimates the parameters for a rigid-body plant such
as vehicle mass with added hydrodynamic mass; quadratic drag; and 
gravitational and buoyancy parameters that arise in the
dynamic models of rigid-body \acp{UV}.
%
The first \ac{AID} algorithm identifies parameters to model 3-\ac{DOF}
\ac{UV} rotational plant dynamics; its development is a precursor to
the second \ac{AID} algorithm.
%
The second algorithm, 6-\ac{DOF} \acf{UV} \acf{AID}, identifies parameters to model
general 6-\ac{DOF} \ac{UV} motion.
% 
For both \ac{AID} algorithms a local stability proof is reported
showing velocity signal estimates converge asymptotically to the plant
velocity signals; parameter estimates are stable;
and parameter estimates converge asymptotically to values that provide
input-output model behavior identical to that of the actual plant.
%
The \ac{JHUROV} was used for comparative experimental evaluations of
both \ac{AID} algorithms.
%
Sections \ref{chUV_AID.sec.UVSO3exp} and \ref{chUV_AID.sec.UVSE3exp}
report comparisons of an \ac{AIDPM} and a \ac{LSPM} in cross-validation
experiments.
%TM_ARE_PARAM_ID_ACRO_AMBIGUIUOUS_HERE
Both models are shown to match closely the \ac{JHUROV}'s
experimentally observed input-output behavior.


As discussed in Section \ref{chSMS_ID.sec.litReview}, a rich
literature exists on the problem of model-based adaptive
trajectory-tracking control.
%
These approaches are not applicable when the plant is either
uncontrolled, under open-loop control, or under the control of a
control law other than a specific adaptive tracking controller.
%
In contrast, the \ac{AID} algorithms reported herein provide an
approach to plant parameter estimation applicable to the commonly
occurring cases of uncontrolled plants, plants under open-loop
control, and plants using control methods prescribed to meet
considerations for an application.
% TM_FIX llw says "Huh, unclear" conserning the third "commonly
% occuring case" above.  One way to make this to have GOOD EXAMPLES
% for each of the three commonly occuring cases.


Unlike \ac{LS} approaches, which requires actuator thrust, position, velocity,
and acceleration signals, \ac{AID} requires only actuator thrust,
position, and velocity signals.
%
Several parameter identification methods have been reported which do not
require direction instrumentation of acceleration.
% 
\cite{hsu&bodson&sastry&paden.icra87} reports an \ac{AID} algorithm
for \acp{OKC} that employs a low pass filter approach to the parameter
update law that does not require joint acceleration signals, and
reports a numerical simulation.
%
\ac{UV} parameter identification algorithms not requiring acceleration
signals have been reported which use adaptive methods
\cite{smallwood2003TCST} or numerical differentiation
\cite{ridao2004identification,avila2012modeling}.
% 
These reported \ac{UV} algorithms, as presented, are for decoupled
plants and have not been shown to generalize to the fully-coupled
\ac{UV} models, such as (\ref{chModels.eq.UVSO3plant}) or
(\ref{chModels.eq.UVSE3plant}).
%
To the best of our knowledge, Theorem \ref{chUV_AID.theo.UV_SE3_AID} is
the first reported adaptive method for parameter estimation of a
fully coupled 6-\ac{DOF} \ac{UV} model {\it without} the additional need to
simultaneously perform reference trajectory-tracking.


The need to dynamically estimate  the rigid-body model parameters from
input-output signals arises in a variety of vehicle dynamics and
control problems including space and air missions, where the vehicle's
mass distribution may vary as fuel or payload are expended over the
duration of a mission.
%
The issues of parameter identification are also important for
\acp{UV}; in comparison to rigid-body 6-\ac{DOF} spacecraft models
(where characterizing vehicle inertia can require 10 scalar values),
the effects of added mass, gravity, buoyancy, and drag require
additional parameter and model complexity to characterize \ac{UV}
dynamics (full characterization of these effects during dynamic \ac{UV}
operation can require hundreds of scalar parameter values).
%
In addition, for most \acp{UV} the drag parameters and mass parameters
(which include both the characteristics of the vehicle's mass and
those of the ambient fluid surrounding the vehicle) cannot be computed
analytically, and thus must be identified experimentally. 


The solution to the identification problem reported herein may prove
useful in applications with controlled or uncontrolled plants in which
reference trajectory-tracking is impractical or infeasible.
%
Of particular interest to the authors are two use case applications
common in our \ac{UV} field deployments.
%
The first is the case of vehicles for which the user does not have the
ability to specify an adaptive tracking control algorithm.
% 
This can be the case with commercially available vehicles because
often the user can not replace the controller provided by the
vehicle's manufacturer.
%
The second is the case of under-actuated vehicles.  
%
Adaptive tracking controllers require actuation in all \ac{DOF}.
%
Frequently vehicle designers utilize \ac{UV} passive stability of pitch
and roll in the design of under-actuated vehicles,
making adaptive tracking control not applicable for either control
or model identification.
%
In these examples, the ability to estimate continuously the plant
model parameters from available input-output signals may enable
improved model-based control.  % of these dynamic plants.
%  
Continuous parameter monitoring may also enable the detection of
unexpected changes that indicate system failures.



The \ac{UV} rotational dynamics \ac{AID} algorithm and its
experimental evaluation were originally reported in
2012\cite{mcfarland2012}.
%
The \ac{UV} \ac{AID} algorithm and its experimental evaluation were
originally reported in 2013\cite{mcfarland.icra2013}.
% \textit{ACM Transactions on Applied Perception} in 2010 with
% co-authors A.\ M.\ Okamura and K.\ J.\ Kuchenbecker
% \cite{Blank_TAP2010}.  Some additional material has been added here
% for clarity.
