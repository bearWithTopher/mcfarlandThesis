%leastSquares.tex
\section{\acs{UV} Least Squares Parameter Identification} 
\label{chUV_AID.sec.leastSquares}

This Section reviews the method of least squares experimental
identification of plant parameters for two \ac{UV} models.
%
%Section \ref{chUV_AID.sec.UVSO3_LS} reviews identification of the
%parameters in (\ref{chModels.eq.UVSO3plant}), which models a \ac{UV}
%subject to drag, buoyancy and external torques.
%
%Section \ref{chUV_AID.sec.UVSE3_LS} reviews identification of the
%parameters in (\ref{chModels.eq.UVSE3plant}), which models a \ac{UV}
%subject to drag, gravity, buoyancy external forces and torques.
%
%
%
%\subsection{Parameter \ac{LS} for a \ac{UV} Subject to Drag, Buoyancy
%  and External Torques}
%\label{chUV_AID.sec.UVSO3_LS}
%
%
%TM_FIX both sentences below are too wordy and cam be SIMPLIFIED
As shown in Section \ref{chModels.sec.UVSO3plant}, the \acl{UV}
rotational dynamics plant model, (\ref{chModels.eq.UVSO3plant}), can
be rewritten as
%
\begin{equation}\label{chUV_AID.eq.UVSO3_LS}
\tau(t)=\UVRreg(\omega,\dot{\omega},R) \theta_{UVR}
\end{equation}
%
\noindent where the $UVR$ subscript denotes \ac{UV} rotational dynamics,
\funDefn{\UVRreg}{\realSpace{3}\times\realSpace{3}\times\SO3}{\realSpace{3\times
    36}} is a regressor matrix, and $\theta_{UVR}\in\realSpace{36}$ is a
vector of the unique scalar parameter values in $I$, $C$, and $b$. 
%
As shown in Section \ref{chModels.sec.UVSE3plant}, the \acl{UV} plant
model, (\ref{chModels.eq.UVSE3plant}), can be rewritten as
%
\begin{equation}\label{chUV_AID.eq.UVSE3_LS}
\tau(t)=\UVreg(v,\dot{v},R) \theta_{UV}
\end{equation}
%
\noindent where the $UV$ subscript denotes 6-\ac{DOF} \ac{UV} dynamics,
\funDefn{\UVreg}{\realSpace{6}\times\realSpace{6}\times\SO3}{\realSpace{6\times
    241}} is a regressor matrix, and $\theta_{UV}\in\realSpace{241}$ is
a vector of the unique scalar parameter values in $M$, $D$, $g$, and
$b$.


The linearity of (\ref{chUV_AID.eq.UVSO3_LS}) and
(\ref{chUV_AID.eq.UVSE3_LS}) in their respective parameters allows the
use of a number of methods to estimate these parameters. A common
method is \acf{LS}.
%
\ac{LS} for (\ref{chUV_AID.eq.UVSO3_LS}) requires the signals
$\tau(t)$, $\omega(t)$, $\dot{\omega}(t)$, and $R(t)$ to be
instrumented.
%
Similarly, \ac{LS} for (\ref{chUV_AID.eq.UVSE3_LS}) requires the
signals $f(t)$, $\tau(t)$, $\nu(t)$, $\dot{\nu}(t)$, $\omega(t)$,
$\dot{\omega}(t)$, and $R(t)$ to be instrumented.
%
One method of formulating the \ac{LS} using the regressor matrix
$\UVreg$ is to employ sampled experimental data of the form 
$\{f(t_i),\;R(t_i),\;v(t_i),\;\dot{v}(t_i)\}$, $t_i\in\{1,2,\cdots n\}$.
%
From (\ref{chUV_AID.eq.UVSE3_LS}) we have
$f(t_i)=\UVreg(v(t_i),\dot{v}(t_i),R(t_i)) \theta_{UV} \quad \forall
i\in\{1,2,\cdots,n\}$.
%
Thus if we define $F=[f(t_1)^T\;f(t_2)^T\;\cdots\;f(t_n)^T]^T$ and
%
\begin{equation}
\UVregStack=\left[ \begin{array}{c}
 \UVreg(v(t_1),\dot{v}(t_1),R(t_1)) \\
 \UVreg(v(t_2),\dot{v}(t_2),R(t_2)) \\
 \vdots \\
 \UVreg(v(t_n),\dot{v}(t_n),R(t_n)) \\
 \end{array}\right],
\end{equation}
%
\noindent then the following relation holds
%
\begin{equation}
F=\UVregStack\theta_{UV}
\end{equation}
%
where $F$ and $\UVregStack$ are know and $\theta_{UV}$ is unknown.  
%
The Moore-Penrose psudo inverse of $\UVregStack$ provides the
best estimate of $\theta_{UV}$ in the least-squares sense through the
formula
%
\begin{equation}
\theta_{UV}=\UVregStack^\dagger F.
\end{equation}
%
If $\UVregStack$ is full rank, then the solution $\theta_{UV}$ will be
unique.  
%
A similar relationship can be derived for $\theta_{UVR}$.

In most plants angular acceleration, $\dot{\omega}(t)$, is not
instrumented directly and must be estimated by differentiating the
angular velocity signal, $\omega(t)$, or twice differentiating angular
position signals.  
%
The \ac{LS} algorithm was implemented using a single Moore-Penrose
pseudo inverse for Sections \ref{chUV_AID.sec.UVSO3exp} and
\ref{chUV_AID.sec.UVSE3exp}.
%
%Note that the \ac{AID} algorithm does not require access to the
%vehicle's angular acceleration, whereas \ac{LS} 
%does require instrumentation of angular acceleration.
%
In the experiments reported herein, the \ac{JHUROV} experienced angular
velocities on the order of 10$^\circ$/s.  Since the PHINS \ac{INS}
measures the vehicle's angular velocity to within
0.01$^\circ$/s, we found that numerical differentiation of the angular
velocity signals provided angular acceleration signals adequate for
\ac{LS}.
%
Inertial-grade angular acceleration sensors are not widely available;
many \ac{UV} are not equipped with angular velocity sensors accurate
enough for precise numerical differentiation.
%
\ac{AID} might be a better option over \ac{LS} for some \ac{UV} in the
field.
%
The \ac{UV} \ac{AID} algorithm reported herein
requires only the signals $u(t)$, $\tau(t)$, $R(t)$, $\nu(t)$, and
$\omega(t)$. 
%and thus may offer some advantages for plants in which
%acceleration signals are not available.
%



%\unit{10}{\Hz} is the update rate used by the \ac{JHUROV} control
%software, and is the rate both state measurements and input torques are
%being recorded.
%
%Our \ac{AID} implementation assumes the update rate of \unit{10}{\Hz}
%is fast enough to allow for a constant parameter estimate update rate
%for the \unit{0.1}{\s} interval between state measurements.


